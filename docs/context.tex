\documentclass[]{book}
\usepackage{lmodern}
\usepackage{amssymb,amsmath}
\usepackage{ifxetex,ifluatex}
\usepackage{fixltx2e} % provides \textsubscript
\ifnum 0\ifxetex 1\fi\ifluatex 1\fi=0 % if pdftex
  \usepackage[T1]{fontenc}
  \usepackage[utf8]{inputenc}
\else % if luatex or xelatex
  \ifxetex
    \usepackage{mathspec}
  \else
    \usepackage{fontspec}
  \fi
  \defaultfontfeatures{Ligatures=TeX,Scale=MatchLowercase}
\fi
% use upquote if available, for straight quotes in verbatim environments
\IfFileExists{upquote.sty}{\usepackage{upquote}}{}
% use microtype if available
\IfFileExists{microtype.sty}{%
\usepackage{microtype}
\UseMicrotypeSet[protrusion]{basicmath} % disable protrusion for tt fonts
}{}
\usepackage[margin=1in]{geometry}
\usepackage{hyperref}
\hypersetup{unicode=true,
            pdftitle={Poverty and Inequality with Complex Survey Data},
            pdfauthor={Guilherme Jacob, Anthony Damico, and Djalma Pessoa},
            pdfborder={0 0 0},
            breaklinks=true}
\urlstyle{same}  % don't use monospace font for urls
\usepackage{natbib}
\bibliographystyle{apalike}
\usepackage{color}
\usepackage{fancyvrb}
\newcommand{\VerbBar}{|}
\newcommand{\VERB}{\Verb[commandchars=\\\{\}]}
\DefineVerbatimEnvironment{Highlighting}{Verbatim}{commandchars=\\\{\}}
% Add ',fontsize=\small' for more characters per line
\usepackage{framed}
\definecolor{shadecolor}{RGB}{248,248,248}
\newenvironment{Shaded}{\begin{snugshade}}{\end{snugshade}}
\newcommand{\KeywordTok}[1]{\textcolor[rgb]{0.13,0.29,0.53}{\textbf{#1}}}
\newcommand{\DataTypeTok}[1]{\textcolor[rgb]{0.13,0.29,0.53}{#1}}
\newcommand{\DecValTok}[1]{\textcolor[rgb]{0.00,0.00,0.81}{#1}}
\newcommand{\BaseNTok}[1]{\textcolor[rgb]{0.00,0.00,0.81}{#1}}
\newcommand{\FloatTok}[1]{\textcolor[rgb]{0.00,0.00,0.81}{#1}}
\newcommand{\ConstantTok}[1]{\textcolor[rgb]{0.00,0.00,0.00}{#1}}
\newcommand{\CharTok}[1]{\textcolor[rgb]{0.31,0.60,0.02}{#1}}
\newcommand{\SpecialCharTok}[1]{\textcolor[rgb]{0.00,0.00,0.00}{#1}}
\newcommand{\StringTok}[1]{\textcolor[rgb]{0.31,0.60,0.02}{#1}}
\newcommand{\VerbatimStringTok}[1]{\textcolor[rgb]{0.31,0.60,0.02}{#1}}
\newcommand{\SpecialStringTok}[1]{\textcolor[rgb]{0.31,0.60,0.02}{#1}}
\newcommand{\ImportTok}[1]{#1}
\newcommand{\CommentTok}[1]{\textcolor[rgb]{0.56,0.35,0.01}{\textit{#1}}}
\newcommand{\DocumentationTok}[1]{\textcolor[rgb]{0.56,0.35,0.01}{\textbf{\textit{#1}}}}
\newcommand{\AnnotationTok}[1]{\textcolor[rgb]{0.56,0.35,0.01}{\textbf{\textit{#1}}}}
\newcommand{\CommentVarTok}[1]{\textcolor[rgb]{0.56,0.35,0.01}{\textbf{\textit{#1}}}}
\newcommand{\OtherTok}[1]{\textcolor[rgb]{0.56,0.35,0.01}{#1}}
\newcommand{\FunctionTok}[1]{\textcolor[rgb]{0.00,0.00,0.00}{#1}}
\newcommand{\VariableTok}[1]{\textcolor[rgb]{0.00,0.00,0.00}{#1}}
\newcommand{\ControlFlowTok}[1]{\textcolor[rgb]{0.13,0.29,0.53}{\textbf{#1}}}
\newcommand{\OperatorTok}[1]{\textcolor[rgb]{0.81,0.36,0.00}{\textbf{#1}}}
\newcommand{\BuiltInTok}[1]{#1}
\newcommand{\ExtensionTok}[1]{#1}
\newcommand{\PreprocessorTok}[1]{\textcolor[rgb]{0.56,0.35,0.01}{\textit{#1}}}
\newcommand{\AttributeTok}[1]{\textcolor[rgb]{0.77,0.63,0.00}{#1}}
\newcommand{\RegionMarkerTok}[1]{#1}
\newcommand{\InformationTok}[1]{\textcolor[rgb]{0.56,0.35,0.01}{\textbf{\textit{#1}}}}
\newcommand{\WarningTok}[1]{\textcolor[rgb]{0.56,0.35,0.01}{\textbf{\textit{#1}}}}
\newcommand{\AlertTok}[1]{\textcolor[rgb]{0.94,0.16,0.16}{#1}}
\newcommand{\ErrorTok}[1]{\textcolor[rgb]{0.64,0.00,0.00}{\textbf{#1}}}
\newcommand{\NormalTok}[1]{#1}
\usepackage{longtable,booktabs}
\usepackage{graphicx,grffile}
\makeatletter
\def\maxwidth{\ifdim\Gin@nat@width>\linewidth\linewidth\else\Gin@nat@width\fi}
\def\maxheight{\ifdim\Gin@nat@height>\textheight\textheight\else\Gin@nat@height\fi}
\makeatother
% Scale images if necessary, so that they will not overflow the page
% margins by default, and it is still possible to overwrite the defaults
% using explicit options in \includegraphics[width, height, ...]{}
\setkeys{Gin}{width=\maxwidth,height=\maxheight,keepaspectratio}
\IfFileExists{parskip.sty}{%
\usepackage{parskip}
}{% else
\setlength{\parindent}{0pt}
\setlength{\parskip}{6pt plus 2pt minus 1pt}
}
\setlength{\emergencystretch}{3em}  % prevent overfull lines
\providecommand{\tightlist}{%
  \setlength{\itemsep}{0pt}\setlength{\parskip}{0pt}}
\setcounter{secnumdepth}{5}
% Redefines (sub)paragraphs to behave more like sections
\ifx\paragraph\undefined\else
\let\oldparagraph\paragraph
\renewcommand{\paragraph}[1]{\oldparagraph{#1}\mbox{}}
\fi
\ifx\subparagraph\undefined\else
\let\oldsubparagraph\subparagraph
\renewcommand{\subparagraph}[1]{\oldsubparagraph{#1}\mbox{}}
\fi

%%% Use protect on footnotes to avoid problems with footnotes in titles
\let\rmarkdownfootnote\footnote%
\def\footnote{\protect\rmarkdownfootnote}

%%% Change title format to be more compact
\usepackage{titling}

% Create subtitle command for use in maketitle
\newcommand{\subtitle}[1]{
  \posttitle{
    \begin{center}\large#1\end{center}
    }
}

\setlength{\droptitle}{-2em}
  \title{Poverty and Inequality with Complex Survey Data}
  \pretitle{\vspace{\droptitle}\centering\huge}
  \posttitle{\par}
  \author{Guilherme Jacob, Anthony Damico, and Djalma Pessoa}
  \preauthor{\centering\large\emph}
  \postauthor{\par}
  \predate{\centering\large\emph}
  \postdate{\par}
  \date{2017-04-20}

\usepackage{booktabs}

\begin{document}
\maketitle

{
\setcounter{tocdepth}{1}
\tableofcontents
}
\chapter{Introduction}\label{introduction}

The R \texttt{convey} library estimates measures of poverty, inequality,
and wellbeing. There are two other R libraries covering this subject,
\href{https://CRAN.R-project.org/package=vardpoor}{vardpoor} and
\href{https://CRAN.R-project.org/package=laeken}{laeken}, however, only
\texttt{convey} integrates seamlessly with the
\href{https://CRAN.R-project.org/package=survey}{R survey package}.

\texttt{convey} is free and open-source software that runs inside the
\href{https://www.r-project.org/}{R environment for statistical
computing}. Anyone can review and propose changes to
\href{https://github.com/DjalmaPessoa/convey}{the source code} for this
software. Readers are welcome to
\href{https://github.com/guilhermejacob/context/}{propose changes to
this book} as well.

\section{Installation}\label{install}

In order to work with the \texttt{convey} library, you will need to have
R running on your machine. If you have never used R before, you will
need to \href{https://www.r-project.org/}{install that software} before
\texttt{convey} can be accessed. Check out
\href{http://flowingdata.com/2012/06/04/resources-for-getting-started-with-r/}{FlowingData}
for a concise list of resources for new R users. Once you have R loaded
on your machine, you can install..

\begin{itemize}
\item
  the latest released version from
  \href{https://CRAN.R-project.org/package=convey}{CRAN} with

\begin{Shaded}
\begin{Highlighting}[]
\KeywordTok{install.packages}\NormalTok{(}\StringTok{"convey"}\NormalTok{)}
\end{Highlighting}
\end{Shaded}
\item
  the latest development version from github with

\begin{Shaded}
\begin{Highlighting}[]
\NormalTok{devtools}\OperatorTok{::}\KeywordTok{install_github}\NormalTok{(}\StringTok{"djalmapessoa/convey"}\NormalTok{)}
\end{Highlighting}
\end{Shaded}
\end{itemize}

\section{Complex surveys and statistical inference}\label{survey}

In this book, we demonstrate how to measure poverty and income
concentration in a population based on microdata collected from a
complex survey sample. Most surveys administered by government agencies
or larger research organizations utilize a sampling design that violates
the assumption of simple random sampling (SRS), including:

\begin{enumerate}
\def\labelenumi{\arabic{enumi}.}
\tightlist
\item
  Different units selection probabilities;
\item
  Clustering of units;
\item
  Stratification of clusters;
\item
  Reweighting to compensate for missing values and other adjustments.
\end{enumerate}

Therefore, basic unweighted R commands such as \texttt{mean()} or
\texttt{glm()} will not properly account for the weighting nor the
measures of uncertainty (such as the confidence intervals) present in
the dataset. For some examples of publicly-available complex survey data
sets, see \href{}{http://asdfree.com}.

Unlike other software, the R \texttt{convey} package does not require
that the user specify these parameters throughout the analysis. So long
as the
\href{http://r-survey.r-forge.r-project.org/survey/html/svydesign.html}{svydesign
object} or
\href{http://r-survey.r-forge.r-project.org/survey/html/svrepdesign.html}{svrepdesign
object} has been constructed properly at the outset of the analysis, the
\texttt{convey} package will incorporate the survey design automatically
and produce statistics and variances that take the complex sample into
account.

\section{Usage Examples}\label{usage-examples}

In the following example, we've loaded the data set \texttt{eusilc} from
the R libraries
\href{https://CRAN.R-project.org/package=vardpoor}{vardpoor} and
\href{https://CRAN.R-project.org/package=laeken}{laeken}.

\begin{Shaded}
\begin{Highlighting}[]
\KeywordTok{library}\NormalTok{(vardpoor)}
\KeywordTok{data}\NormalTok{(eusilc)}
\end{Highlighting}
\end{Shaded}

Next, we create an object of class \texttt{survey.design} using the
function \texttt{svydesign} of the library survey:

\begin{Shaded}
\begin{Highlighting}[]
\KeywordTok{library}\NormalTok{(survey)}
\NormalTok{des_eusilc <-}\StringTok{ }\KeywordTok{svydesign}\NormalTok{(}\DataTypeTok{ids =} \OperatorTok{~}\NormalTok{rb030, }\DataTypeTok{strata =}\OperatorTok{~}\NormalTok{db040,  }\DataTypeTok{weights =} \OperatorTok{~}\NormalTok{rb050, }\DataTypeTok{data =}\NormalTok{ eusilc)}
\end{Highlighting}
\end{Shaded}

Right after the creation of the design object \texttt{des\_eusilc}, we
should use the function \texttt{convey\_prep} that adds an attribute to
the survey design which saves information on the design object based
upon the whole sample, needed to work with subset designs.

\begin{Shaded}
\begin{Highlighting}[]
\KeywordTok{library}\NormalTok{(convey)}
\NormalTok{des_eusilc <-}\StringTok{ }\KeywordTok{convey_prep}\NormalTok{( des_eusilc )}
\end{Highlighting}
\end{Shaded}

To estimate the at-risk-of-poverty rate, we use the function
\texttt{svyarpt}:

\begin{Shaded}
\begin{Highlighting}[]
\KeywordTok{svyarpr}\NormalTok{(}\OperatorTok{~}\NormalTok{eqIncome, }\DataTypeTok{design=}\NormalTok{des_eusilc)}
\end{Highlighting}
\end{Shaded}

\begin{verbatim}
            arpr     SE
eqIncome 0.14444 0.0028
\end{verbatim}

To estimate the at-risk-of-poverty rate across domains defined by the
variable \texttt{db040} we use:

\begin{Shaded}
\begin{Highlighting}[]
\KeywordTok{svyby}\NormalTok{(}\OperatorTok{~}\NormalTok{eqIncome, }\DataTypeTok{by =} \OperatorTok{~}\NormalTok{db040, }\DataTypeTok{design =}\NormalTok{ des_eusilc, }\DataTypeTok{FUN =}\NormalTok{ svyarpr, }\DataTypeTok{deff =} \OtherTok{FALSE}\NormalTok{)}
\end{Highlighting}
\end{Shaded}

\begin{verbatim}
                      db040  eqIncome          se
Burgenland       Burgenland 0.1953984 0.017202243
Carinthia         Carinthia 0.1308627 0.010610622
Lower Austria Lower Austria 0.1384362 0.006517660
Salzburg           Salzburg 0.1378734 0.011579280
Styria               Styria 0.1437464 0.007452360
Tyrol                 Tyrol 0.1530819 0.009880430
Upper Austria Upper Austria 0.1088977 0.005928336
Vienna               Vienna 0.1723468 0.007682826
Vorarlberg       Vorarlberg 0.1653731 0.013754670
\end{verbatim}

Using the same data set, we estimate the quintile share ratio:

\begin{Shaded}
\begin{Highlighting}[]
\CommentTok{# for the whole population}
\KeywordTok{svyqsr}\NormalTok{(}\OperatorTok{~}\NormalTok{eqIncome, }\DataTypeTok{design=}\NormalTok{des_eusilc, }\DataTypeTok{alpha1=}\NormalTok{ .}\DecValTok{20}\NormalTok{)}
\end{Highlighting}
\end{Shaded}

\begin{verbatim}
          qsr     SE
eqIncome 3.97 0.0426
\end{verbatim}

\begin{Shaded}
\begin{Highlighting}[]
\CommentTok{# for domains}
\KeywordTok{svyby}\NormalTok{(}\OperatorTok{~}\NormalTok{eqIncome, }\DataTypeTok{by =} \OperatorTok{~}\NormalTok{db040, }\DataTypeTok{design =}\NormalTok{ des_eusilc,}
  \DataTypeTok{FUN =}\NormalTok{ svyqsr, }\DataTypeTok{alpha1=}\NormalTok{ .}\DecValTok{20}\NormalTok{, }\DataTypeTok{deff =} \OtherTok{FALSE}\NormalTok{)}
\end{Highlighting}
\end{Shaded}

\begin{verbatim}
                      db040 eqIncome         se
Burgenland       Burgenland 5.008486 0.32755685
Carinthia         Carinthia 3.562404 0.10909726
Lower Austria Lower Austria 3.824539 0.08783599
Salzburg           Salzburg 3.768393 0.17015086
Styria               Styria 3.464305 0.09364800
Tyrol                 Tyrol 3.586046 0.13629739
Upper Austria Upper Austria 3.668289 0.09310624
Vienna               Vienna 4.654743 0.13135731
Vorarlberg       Vorarlberg 4.366511 0.20532075
\end{verbatim}

These functions can be used as S3 methods for the classes
\texttt{survey.design} and \texttt{svyrep.design}.

Let's create a design object of class \texttt{svyrep.design} and run the
function \texttt{convey\_prep} on it:

\begin{Shaded}
\begin{Highlighting}[]
\NormalTok{des_eusilc_rep <-}\StringTok{ }\KeywordTok{as.svrepdesign}\NormalTok{(des_eusilc, }\DataTypeTok{type =} \StringTok{"bootstrap"}\NormalTok{)}
\NormalTok{des_eusilc_rep <-}\StringTok{ }\KeywordTok{convey_prep}\NormalTok{(des_eusilc_rep) }
\end{Highlighting}
\end{Shaded}

and then use the function \texttt{svyarpr}:

\begin{Shaded}
\begin{Highlighting}[]
\KeywordTok{svyarpr}\NormalTok{(}\OperatorTok{~}\NormalTok{eqIncome, }\DataTypeTok{design=}\NormalTok{des_eusilc_rep)}
\end{Highlighting}
\end{Shaded}

\begin{verbatim}
            arpr     SE
eqIncome 0.14444 0.0024
\end{verbatim}

\begin{Shaded}
\begin{Highlighting}[]
\KeywordTok{svyby}\NormalTok{(}\OperatorTok{~}\NormalTok{eqIncome, }\DataTypeTok{by =} \OperatorTok{~}\NormalTok{db040, }\DataTypeTok{design =}\NormalTok{ des_eusilc_rep, }\DataTypeTok{FUN =}\NormalTok{ svyarpr, }\DataTypeTok{deff =} \OtherTok{FALSE}\NormalTok{)}
\end{Highlighting}
\end{Shaded}

\begin{verbatim}
                      db040  eqIncome se.eqIncome
Burgenland       Burgenland 0.1953984 0.017801952
Carinthia         Carinthia 0.1308627 0.012519104
Lower Austria Lower Austria 0.1384362 0.007868591
Salzburg           Salzburg 0.1378734 0.013794679
Styria               Styria 0.1437464 0.006967357
Tyrol                 Tyrol 0.1530819 0.010636122
Upper Austria Upper Austria 0.1088977 0.006255099
Vienna               Vienna 0.1723468 0.006764023
Vorarlberg       Vorarlberg 0.1653731 0.014675836
\end{verbatim}

The functions of the library convey are called in a similar way to the
functions in library survey.

It is also possible to deal with missing values by using the argument
\texttt{na.rm}.

\begin{Shaded}
\begin{Highlighting}[]
\CommentTok{# survey.design using a variable with missings}
\KeywordTok{svygini}\NormalTok{( }\OperatorTok{~}\StringTok{ }\NormalTok{py010n , }\DataTypeTok{design =}\NormalTok{ des_eusilc )}
\end{Highlighting}
\end{Shaded}

\begin{verbatim}
       gini SE
py010n   NA NA
\end{verbatim}

\begin{Shaded}
\begin{Highlighting}[]
\KeywordTok{svygini}\NormalTok{( }\OperatorTok{~}\StringTok{ }\NormalTok{py010n , }\DataTypeTok{design =}\NormalTok{ des_eusilc , }\DataTypeTok{na.rm =} \OtherTok{TRUE}\NormalTok{ )}
\end{Highlighting}
\end{Shaded}

\begin{verbatim}
          gini     SE
py010n 0.64606 0.0036
\end{verbatim}

\begin{Shaded}
\begin{Highlighting}[]
\CommentTok{# svyrep.design using a variable with missings}
\KeywordTok{svygini}\NormalTok{( }\OperatorTok{~}\StringTok{ }\NormalTok{py010n , }\DataTypeTok{design =}\NormalTok{ des_eusilc_rep )}
\end{Highlighting}
\end{Shaded}

\begin{verbatim}
       gini SE
py010n   NA NA
\end{verbatim}

\begin{Shaded}
\begin{Highlighting}[]
\KeywordTok{svygini}\NormalTok{( }\OperatorTok{~}\StringTok{ }\NormalTok{py010n , }\DataTypeTok{design =}\NormalTok{ des_eusilc_rep , }\DataTypeTok{na.rm =} \OtherTok{TRUE}\NormalTok{ )}
\end{Highlighting}
\end{Shaded}

\begin{verbatim}
          gini    SE
py010n 0.64606 0.004
\end{verbatim}

\section{Underlying Calculations}\label{underlying-calculations}

In what follows, we often use the linearization method as a tool to
produce an approximation for the variance of an estimator. From the
linearized variable \(z\) of an estimator \(T\), we get from the
expression \eqref{eq:var} an estimate of the variance of \(T\)

If \(T\) can be expressed as a function of the population totals
\(T = g(Y_1, Y_2, \ldots, Y_n)\), and if \(g\) is linear, the estimation
of the variance of \(T = g(Y_1, Y_2, \ldots, Y_n)\) is straightforward.
If \(g\) is not linear but is a `smooth' function, then it is possible
to approximate the variance of \(g(Y_1, Y_2, \ldots, Y_n)\) by the
variance of its first order Taylor expansion. For example, we can use
Taylor expansion to linearize the ratio of two totals. However, there
are situations where Taylor linearization cannot be immediately
possible, either because \(T\) cannot be expressed as functions of the
population totals, or because \(g\) is not a \texttt{smooth} function.
An example is the case where \(T\) is a quantile.

In these cases, it might work an alternative form of linearization of
\(T\), by \texttt{Influence\ Function}, as defined in \eqref{eq:lin},
proposed in \citep{deville1999}. Also, it coud be used replication
methods such as \texttt{bootstrap} and \texttt{jackknife}.

In the \texttt{convey} library, there are some basic functions that
produce the linearized variables needed to measure income concentration
and poverty. For example, looking at the income variable in some complex
survey dataset, the \texttt{quantile} of that income variable can be
linearized by the function \texttt{convey::svyiqalpha} and the sum total
below any quantile of the variable is linearized by the function
\texttt{convey::svyisq}.

From the linearized variables of these basic estimates, it is possible
by using rules of composition, valid for influence functions, to derive
the influence function of more complex estimates. By definition the
influence function is a Gateaux derivative and the rules rules of
composition valid for Gateaux derivatives also hold for Influence
Functions.

The following property of Gateaux derivatives was often used in the
library convey. Let \(g\) be a differentiable function of \(m\)
variables. Suppose we want to compute the influence function of the
estimator \(g(T_1, T_2,\ldots, T_m)\), knowing the Influence function of
the estimators \(T_i, i=1,\ldots, m\). Then the following holds:

\[
I(g(T_1, T_2,\ldots, T_m)) = \sum_{i=1}^m \frac{\partial g}{\partial T_i}I(T_i)
\]

In the library convey this rule is implemented by the function
\texttt{contrastinf} which uses the R function \texttt{deriv} to compute
the formal partial derivatives \(\frac{\partial g}{\partial T_i}\).

For example, suppose we want to linearize the
\texttt{Relative\ median\ poverty\ gap}(rmpg), defined as the difference
between the at-risk-of-poverty threshold (\texttt{arpt}) and the median
of incomes less than the \texttt{arpt} relative to the \texttt{arprt}:

\[
rmpg= \frac{arpt-medpoor} {arpt}
\]

where \texttt{medpoor} is the median of incomes less than \texttt{arpt}.

Suppose we know how to linearize \texttt{arpt} and \texttt{medpoor},
then by applying the function \texttt{contrastinf} with \[
g(T_1,T_2)= \frac{(T_1 - T_2)}{T_1}
\] we linearize the \texttt{rmpg}.

\section{The Variance Estimator}\label{the-variance-estimator}

Using the notation in \citep{osier2009}, the variance of the estimator
\(T(\hat{M})\) can approximated by:

\begin{equation}
Var\left[T(\hat{M})\right]\cong var\left[\sum_s w_i z_i\right]
\label{eq:var}
\end{equation}

The \texttt{linearized} variable \(z\) is given by the derivative of the
functional:

\begin{equation}
z_k=lim_{t\rightarrow0}\frac{T(M+t\delta_k)-T(M)}{t}=IT_k(M)
\label{eq:lin}
\end{equation}

where, \(\delta_k\) is the Dirac measure in \(k\): \(\delta_k(i)=1\) if
and only if \(i=k\).

This \textbf{derivative} is called \textbf{Influence Function} and was
introduced in the area of \textbf{Robust Statistics}.

\section{Influence Functions}\label{influence-functions}

Some measures of poverty and income concentration are defined by
non-differentiable functions so that it is not possible to use Taylor
linearization to estimate their variances. An alternative is to use
\textbf{Influence functions} as described in \citep{deville1999} and
\citep{osier2009}. The convey library implements this methodology to
work with \texttt{survey.design} objects and also with
\texttt{svyrep.design} objects.

Some examples of these measures are:

\begin{itemize}
\item
  At-risk-of-poverty threshold: \(arpt=.60q_{.50}\) where \(q_{.50}\) is
  the income median;
\item
  At-risk-of-poverty rate \(arpr=\frac{\sum_U 1(y_i \leq arpt)}{N}.100\)
\item
  Quintile share ratio
\end{itemize}

\(qsr=\frac{\sum_U 1(y_i>q_{.80})}{\sum_U 1(y_i\leq q_{.20})}\)

\begin{itemize}
\tightlist
\item
  Gini coefficient \(1+G=\frac{2\sum_U (r_i-1)y_i}{N\sum_Uy_i}\) where
  \(r_i\) is the rank of \(y_i\).
\end{itemize}

Note that it is not possible to use Taylor linearization for these
measures because they depend on quantiles and the Gini is defined as a
function of ranks. This could be done using the approach proposed by
Deville (1999) based upon influence functions.

Let \(U\) be a population of size \(N\) and \(M\) be a measure that
allocates mass one to the set composed by one unit, that is
\(M(i)=M_i= 1\) if \(i\in U\) and \(M(i)=0\) if \(i\notin U\)

Now, a population parameter \(\theta\) can be expressed as a functional
of \(M\) \(\theta=T(M)\)

Examples of such parameters are:

\begin{itemize}
\item
  Total: \(Y=\sum_Uy_i=\sum_U y_iM_i=\int ydM=T(M)\)
\item
  Ratio of two totals:
  \(R=\frac{Y}{X}=\frac{\int y dM}{\int x dM}=T(M)\)
\item
  Cumulative distribution function:
  \(F(x)=\frac{\sum_U 1(y_i\leq x)}{N}=\frac{\int 1(y\leq x)dM}{\int{dM}}=T(M)\)
\end{itemize}

To estimate these parameters from the sample, we replace the measure
\(M\) by the estimated measure \(\hat{M}\) defined by:
\(\hat{M}(i)=\hat{M}_i= w_i\) if \(i\in s\) and \(\hat{M}(i)=0\) if
\(i\notin s\).

The estimators of the population parameters can then be expressed as
functional of the measure \(\hat{M}\).

\begin{itemize}
\item
  Total: \(\hat{Y}=T(\hat{M})=\int yd\hat{M}=\sum_s w_iy_i\)
\item
  Ratio of totals:
  \(\hat{R}=T(\hat{M})=\frac{\int y d\hat{M}}{\int x d\hat{M}}=\frac{\sum_s w_iy_i}{\sum_s w_ix_i}\)
\item
  Cumulative distribution function:
  \(\hat{F}(x)=T(\hat{M})=\frac{\int 1(y\leq x)d\hat{M}}{\int{d\hat{M}}}=\frac{\sum_s w_i 1(y_i\leq x)}{\sum_s w_i}\)
\end{itemize}

\section{Influence Function Examples}\label{influence-function-examples}

\begin{itemize}
\item
  Total: \[
  \begin{aligned}
  IT_k(M)&=lim_{t\rightarrow 0}\frac{T(M+t\delta_k)-T(M)}{t}\\
  &=lim_{t\rightarrow 0}\frac{\int y.d(M+t\delta_k)-\int y.dM}{t}\\
  &=lim_{t\rightarrow 0}\frac{\int yd(t\delta_k)}{t}=y_k  
  \end{aligned}
  \]
\item
  Ratio of two totals: \[
  \begin{aligned}
  IR_k(M)&=I\left(\frac{U}{V}\right)_k(M)=\frac{V(M)\times IU_k(M)-U(M)\times IV_k(M)}{V(M)^2}\\
  &=\frac{X y_k-Y x_k}{X^2}=\frac{1}{X}(y_k-Rx_k)
  \end{aligned}
  \]
\end{itemize}

\section{Examples of Linearization Using the Influence
Function}\label{examples-of-linearization-using-the-influence-function}

\begin{itemize}
\tightlist
\item
  At-risk-of-poverty threshold: \[
  arpt = 0.6\times m
  \] where \(m\) is the median income.
\end{itemize}

\[
z_k= -\frac{0.6}{f(m)}\times\frac{1}{N}\times\left[I(y_k\leq m-0.5) \right]
\]

\begin{itemize}
\tightlist
\item
  At-risk-of-poverty rate:
\end{itemize}

\[
 arpr=\frac{\sum_U I(y_i \leq t)}{\sum_U w_i}.100
\] \[
z_k=\frac{1}{N}\left[I(y_k\leq t)-t\right]-\frac{0.6}{N}\times\frac{f(t)}{f(m)}\left[I(y_k\leq m)-0.5\right]
\]

where:

\(N\) - population size;

\(t\) - at-risk-of-poverty threshold;

\(y_k\) - income of person \(k\);

\(m\) - median income;

\(f\) - income density function;

\section{Replication Designs}\label{replication-designs}

All major functions in the library convey have S3 methods for the
classes: \texttt{survey.design}, \texttt{svyrep.design} and
\texttt{DBIdesign}. When the argument \texttt{design} is\\
a survey design with replicate weights created by the library survey,
convey uses the method \texttt{svrepdesign}.

Considering the remarks in \citep{W85}, p.~163, concerning the
deficiency of the \texttt{Jackknife} method in estimating the variance
of \texttt{quantiles}, we adopted the type bootstrap instead.

The function \texttt{bootVar} from the library \texttt{laeken} ,
\citep{R-laeken}, also uses the bootstrap method to estimate variances.

\section{Decomposition}\label{decomposition}

Some inequality and multidimensional poverty measures can be decomposed.
As of December 2016, the decomposition methods in \texttt{convey} are
limited to group decomposition.

For instance, the generalized entropy index can be decomposed into
between and within group components. This sheds light on a very simple
question: of the overall inequality, how much can be explained by
inequalities between groups and within groups? Since this measure is
additive decomposable, one can get estimates of the coefficients, SEs
and covariance between components. For a more practical approach, see
\citep{lima2013}.

The Alkire-Foster class of multidimensional poverty indices can be
decomposed by dimension and groups. This shows how much each group (or
dimension) contribute to the overall poverty.

This technique can help understand where and who is more affected by
inequality and poverty, contributing to more specific policy and
economic analysis.

\chapter{Poverty Indices}\label{poverty}

\section{At Risk of Poverty Ratio
(svyarpr)}\label{at-risk-of-poverty-ratio-svyarpr}

The at-risk-of-poverty rate is the share of persons with an income below
the at-risk-of-poverty threshold (\texttt{arpt}). The details of the
linearization of the \texttt{arpr} and are discussed by
\citep{deville1999} and \citep{osier2009}.

\begin{center}\rule{0.5\linewidth}{\linethickness}\end{center}

\textbf{A replication example}

The R \texttt{vardpoor} package \citep{vardpoor}, created by researchers
at the Central Statistical Bureau of Latvia, includes a arpr coefficient
calculation using the ultimate cluster method. The example below
reproduces those statistics.

Load and prepare the same data set:

\begin{Shaded}
\begin{Highlighting}[]
\CommentTok{# load the convey package}
\KeywordTok{library}\NormalTok{(convey)}

\CommentTok{# load the survey library}
\KeywordTok{library}\NormalTok{(survey)}

\CommentTok{# load the vardpoor library}
\KeywordTok{library}\NormalTok{(vardpoor)}

\CommentTok{# load the synthetic european union statistics on income & living conditions}
\KeywordTok{data}\NormalTok{(eusilc)}

\CommentTok{# make all column names lowercase}
\KeywordTok{names}\NormalTok{( eusilc ) <-}\StringTok{ }\KeywordTok{tolower}\NormalTok{( }\KeywordTok{names}\NormalTok{( eusilc ) )}

\CommentTok{# add a column with the row number}
\NormalTok{dati <-}\StringTok{ }\KeywordTok{data.table}\NormalTok{(}\DataTypeTok{IDd =} \DecValTok{1} \OperatorTok{:}\StringTok{ }\KeywordTok{nrow}\NormalTok{(eusilc), eusilc)}

\CommentTok{# calculate the arpr coefficient}
\CommentTok{# using the R vardpoor library}
\NormalTok{varpoord_arpr_calculation <-}
\StringTok{    }\KeywordTok{varpoord}\NormalTok{(}
    
        \CommentTok{# analysis variable}
        \DataTypeTok{Y =} \StringTok{"eqincome"}\NormalTok{, }
        
        \CommentTok{# weights variable}
        \DataTypeTok{w_final =} \StringTok{"rb050"}\NormalTok{,}
        
        \CommentTok{# row number variable}
        \DataTypeTok{ID_level1 =} \StringTok{"IDd"}\NormalTok{,}
        
        \CommentTok{# row number variable}
        \DataTypeTok{ID_level2 =} \StringTok{"IDd"}\NormalTok{,}
        
        \CommentTok{# strata variable}
        \DataTypeTok{H =} \StringTok{"db040"}\NormalTok{, }
        
        \DataTypeTok{N_h =} \OtherTok{NULL}\NormalTok{ ,}
        
        \CommentTok{# clustering variable}
        \DataTypeTok{PSU =} \StringTok{"rb030"}\NormalTok{, }
        
        \CommentTok{# data.table}
        \DataTypeTok{dataset =}\NormalTok{ dati, }
        
        \CommentTok{# arpr coefficient function}
        \DataTypeTok{type =} \StringTok{"linarpr"}\NormalTok{,}
      
      \CommentTok{# poverty threshold range}
      \DataTypeTok{order_quant =}\NormalTok{ 50L ,}
          
      \CommentTok{# get linearized variable}
      \DataTypeTok{outp_lin =} \OtherTok{TRUE}
        
\NormalTok{    )}
\end{Highlighting}
\end{Shaded}

\begin{verbatim}
## NULL
\end{verbatim}

\begin{Shaded}
\begin{Highlighting}[]
\CommentTok{# construct a survey.design}
\CommentTok{# using our recommended setup}
\NormalTok{des_eusilc <-}\StringTok{ }
\StringTok{    }\KeywordTok{svydesign}\NormalTok{( }
        \DataTypeTok{ids =} \OperatorTok{~}\StringTok{ }\NormalTok{rb030 , }
        \DataTypeTok{strata =} \OperatorTok{~}\StringTok{ }\NormalTok{db040 ,  }
        \DataTypeTok{weights =} \OperatorTok{~}\StringTok{ }\NormalTok{rb050 , }
        \DataTypeTok{data =}\NormalTok{ eusilc}
\NormalTok{    )}

\CommentTok{# immediately run the convey_prep function on it}
\NormalTok{des_eusilc <-}\StringTok{ }\KeywordTok{convey_prep}\NormalTok{( des_eusilc )}

\CommentTok{# coefficients do match}
\NormalTok{varpoord_arpr_calculation}\OperatorTok{$}\NormalTok{all_result}\OperatorTok{$}\NormalTok{value}
\end{Highlighting}
\end{Shaded}

\begin{verbatim}
## [1] 14.44422
\end{verbatim}

\begin{Shaded}
\begin{Highlighting}[]
\KeywordTok{coef}\NormalTok{( }\KeywordTok{svyarpr}\NormalTok{( }\OperatorTok{~}\StringTok{ }\NormalTok{eqincome , des_eusilc ) ) }\OperatorTok{*}\StringTok{ }\DecValTok{100}
\end{Highlighting}
\end{Shaded}

\begin{verbatim}
## eqincome 
## 14.44422
\end{verbatim}

\begin{Shaded}
\begin{Highlighting}[]
\CommentTok{# linearized variables do match}
\CommentTok{# vardpoor}
\NormalTok{lin_arpr_varpoord<-}\StringTok{ }\NormalTok{varpoord_arpr_calculation}\OperatorTok{$}\NormalTok{lin_out}\OperatorTok{$}\NormalTok{lin_arpr}
\CommentTok{# convey }
\NormalTok{lin_arpr_convey <-}\StringTok{ }\KeywordTok{attr}\NormalTok{(}\KeywordTok{svyarpr}\NormalTok{( }\OperatorTok{~}\StringTok{ }\NormalTok{eqincome , des_eusilc ),}\StringTok{"lin"}\NormalTok{)}

\CommentTok{# check equality}
\KeywordTok{all.equal}\NormalTok{(lin_arpr_varpoord,}\DecValTok{100}\OperatorTok{*}\NormalTok{lin_arpr_convey )}
\end{Highlighting}
\end{Shaded}

\begin{verbatim}
## [1] TRUE
\end{verbatim}

\begin{Shaded}
\begin{Highlighting}[]
\CommentTok{# variances do not match exactly}
\KeywordTok{attr}\NormalTok{( }\KeywordTok{svyarpr}\NormalTok{( }\OperatorTok{~}\StringTok{ }\NormalTok{eqincome , des_eusilc ) , }\StringTok{'var'}\NormalTok{ ) }\OperatorTok{*}\StringTok{ }\DecValTok{10000}
\end{Highlighting}
\end{Shaded}

\begin{verbatim}
##            eqincome
## eqincome 0.07599778
\end{verbatim}

\begin{Shaded}
\begin{Highlighting}[]
\NormalTok{varpoord_arpr_calculation}\OperatorTok{$}\NormalTok{all_result}\OperatorTok{$}\NormalTok{var}
\end{Highlighting}
\end{Shaded}

\begin{verbatim}
## [1] 0.07586194
\end{verbatim}

\begin{Shaded}
\begin{Highlighting}[]
\CommentTok{# standard errors do not match exactly}
\NormalTok{varpoord_arpr_calculation}\OperatorTok{$}\NormalTok{all_result}\OperatorTok{$}\NormalTok{se}
\end{Highlighting}
\end{Shaded}

\begin{verbatim}
## [1] 0.2754305
\end{verbatim}

\begin{Shaded}
\begin{Highlighting}[]
\KeywordTok{SE}\NormalTok{( }\KeywordTok{svyarpr}\NormalTok{( }\OperatorTok{~}\StringTok{ }\NormalTok{eqincome , des_eusilc ) ) }\OperatorTok{*}\StringTok{ }\DecValTok{100}
\end{Highlighting}
\end{Shaded}

\begin{verbatim}
##           eqincome
## eqincome 0.2756769
\end{verbatim}

The variance estimate is computed by using the approximation defined in
\eqref{eq:var}, where the linearized variable \(z\) is defined by
\eqref{eq:lin}. The functions \texttt{convey::svyarpr} and
\texttt{vardpoor::linarpr} produce the same linearized variable \(z\).

However, the measures of uncertainty do not line up, because
\texttt{library(vardpoor)} defaults to an ultimate cluster method that
can be replicated with an alternative setup of the
\texttt{survey.design} object.

\begin{Shaded}
\begin{Highlighting}[]
\CommentTok{# within each strata, sum up the weights}
\NormalTok{cluster_sums <-}\StringTok{ }\KeywordTok{aggregate}\NormalTok{( eusilc}\OperatorTok{$}\NormalTok{rb050 , }\KeywordTok{list}\NormalTok{( eusilc}\OperatorTok{$}\NormalTok{db040 ) , sum )}

\CommentTok{# name the within-strata sums of weights the `cluster_sum`}
\KeywordTok{names}\NormalTok{( cluster_sums ) <-}\StringTok{ }\KeywordTok{c}\NormalTok{( }\StringTok{"db040"}\NormalTok{ , }\StringTok{"cluster_sum"}\NormalTok{ )}

\CommentTok{# merge this column back onto the data.frame}
\NormalTok{eusilc <-}\StringTok{ }\KeywordTok{merge}\NormalTok{( eusilc , cluster_sums )}

\CommentTok{# construct a survey.design}
\CommentTok{# with the fpc using the cluster sum}
\NormalTok{des_eusilc_ultimate_cluster <-}\StringTok{ }
\StringTok{    }\KeywordTok{svydesign}\NormalTok{( }
        \DataTypeTok{ids =} \OperatorTok{~}\StringTok{ }\NormalTok{rb030 , }
        \DataTypeTok{strata =} \OperatorTok{~}\StringTok{ }\NormalTok{db040 ,  }
        \DataTypeTok{weights =} \OperatorTok{~}\StringTok{ }\NormalTok{rb050 , }
        \DataTypeTok{data =}\NormalTok{ eusilc , }
        \DataTypeTok{fpc =} \OperatorTok{~}\StringTok{ }\NormalTok{cluster_sum }
\NormalTok{    )}

\CommentTok{# again, immediately run the convey_prep function on the `survey.design`}
\NormalTok{des_eusilc_ultimate_cluster <-}\StringTok{ }\KeywordTok{convey_prep}\NormalTok{( des_eusilc_ultimate_cluster )}

\CommentTok{# matches}
\KeywordTok{attr}\NormalTok{( }\KeywordTok{svyarpr}\NormalTok{( }\OperatorTok{~}\StringTok{ }\NormalTok{eqincome , des_eusilc_ultimate_cluster ) , }\StringTok{'var'}\NormalTok{ ) }\OperatorTok{*}\StringTok{ }\DecValTok{10000}
\end{Highlighting}
\end{Shaded}

\begin{verbatim}
##            eqincome
## eqincome 0.07586194
\end{verbatim}

\begin{Shaded}
\begin{Highlighting}[]
\NormalTok{varpoord_arpr_calculation}\OperatorTok{$}\NormalTok{all_result}\OperatorTok{$}\NormalTok{var}
\end{Highlighting}
\end{Shaded}

\begin{verbatim}
## [1] 0.07586194
\end{verbatim}

\begin{Shaded}
\begin{Highlighting}[]
\CommentTok{# matches}
\NormalTok{varpoord_arpr_calculation}\OperatorTok{$}\NormalTok{all_result}\OperatorTok{$}\NormalTok{se}
\end{Highlighting}
\end{Shaded}

\begin{verbatim}
## [1] 0.2754305
\end{verbatim}

\begin{Shaded}
\begin{Highlighting}[]
\KeywordTok{SE}\NormalTok{( }\KeywordTok{svyarpr}\NormalTok{( }\OperatorTok{~}\StringTok{ }\NormalTok{eqincome , des_eusilc_ultimate_cluster ) ) }\OperatorTok{*}\StringTok{ }\DecValTok{100}
\end{Highlighting}
\end{Shaded}

\begin{verbatim}
##           eqincome
## eqincome 0.2754305
\end{verbatim}

For additional usage examples of \texttt{svyarpr}, type
\texttt{?convey::svyarpr} in the R console.

\section{At Risk of Poverty Threshold
(svyarpt)}\label{at-risk-of-poverty-threshold-svyarpt}

The at-risk-of-poverty-theshold is \(0.6\) times the median income for
the entire population.

\[
arpt = 0.6 \times median(y),
\] where, \(y\) is the income variable and \texttt{median} is estimated
for the whole population. The details of the linearization of the
\texttt{arpt} are discussed by \citep{deville1999} and
\citep{osier2009}.

\begin{center}\rule{0.5\linewidth}{\linethickness}\end{center}

\textbf{A replication example}

The R \texttt{vardpoor} package \citep{vardpoor}, created by researchers
at the Central Statistical Bureau of Latvia, includes a arpt coefficient
calculation using the ultimate cluster method. The example below
reproduces those statistics.

Load and prepare the same data set:

\begin{Shaded}
\begin{Highlighting}[]
\CommentTok{# load the convey package}
\KeywordTok{library}\NormalTok{(convey)}

\CommentTok{# load the survey library}
\KeywordTok{library}\NormalTok{(survey)}

\CommentTok{# load the vardpoor library}
\KeywordTok{library}\NormalTok{(vardpoor)}

\CommentTok{# load the synthetic european union statistics on income & living conditions}
\KeywordTok{data}\NormalTok{(eusilc)}

\CommentTok{# make all column names lowercase}
\KeywordTok{names}\NormalTok{( eusilc ) <-}\StringTok{ }\KeywordTok{tolower}\NormalTok{( }\KeywordTok{names}\NormalTok{( eusilc ) )}

\CommentTok{# add a column with the row number}
\NormalTok{dati <-}\StringTok{ }\KeywordTok{data.table}\NormalTok{(}\DataTypeTok{IDd =} \DecValTok{1} \OperatorTok{:}\StringTok{ }\KeywordTok{nrow}\NormalTok{(eusilc), eusilc)}

\CommentTok{# calculate the arpt coefficient}
\CommentTok{# using the R vardpoor library}
\NormalTok{varpoord_arpt_calculation <-}
\StringTok{    }\KeywordTok{varpoord}\NormalTok{(}
    
        \CommentTok{# analysis variable}
        \DataTypeTok{Y =} \StringTok{"eqincome"}\NormalTok{, }
        
        \CommentTok{# weights variable}
        \DataTypeTok{w_final =} \StringTok{"rb050"}\NormalTok{,}
        
        \CommentTok{# row number variable}
        \DataTypeTok{ID_level1 =} \StringTok{"IDd"}\NormalTok{,}
        
        \CommentTok{# row number variable}
        \DataTypeTok{ID_level2 =} \StringTok{"IDd"}\NormalTok{,}
        
        \CommentTok{# strata variable}
        \DataTypeTok{H =} \StringTok{"db040"}\NormalTok{, }
        
        \DataTypeTok{N_h =} \OtherTok{NULL}\NormalTok{ ,}
        
        \CommentTok{# clustering variable}
        \DataTypeTok{PSU =} \StringTok{"rb030"}\NormalTok{, }
        
        \CommentTok{# data.table}
        \DataTypeTok{dataset =}\NormalTok{ dati, }
        
        \CommentTok{# arpt coefficient function}
        \DataTypeTok{type =} \StringTok{"linarpt"}\NormalTok{,}
      
      \CommentTok{# poverty threshold range}
      \DataTypeTok{order_quant =}\NormalTok{ 50L ,}
      
        \CommentTok{# get linearized variable}
      \DataTypeTok{outp_lin =} \OtherTok{TRUE}
\NormalTok{    )}
\end{Highlighting}
\end{Shaded}

\begin{verbatim}
## NULL
\end{verbatim}

\begin{Shaded}
\begin{Highlighting}[]
\CommentTok{# construct a survey.design}
\CommentTok{# using our recommended setup}
\NormalTok{des_eusilc <-}\StringTok{ }
\StringTok{    }\KeywordTok{svydesign}\NormalTok{( }
        \DataTypeTok{ids =} \OperatorTok{~}\StringTok{ }\NormalTok{rb030 , }
        \DataTypeTok{strata =} \OperatorTok{~}\StringTok{ }\NormalTok{db040 ,  }
        \DataTypeTok{weights =} \OperatorTok{~}\StringTok{ }\NormalTok{rb050 , }
        \DataTypeTok{data =}\NormalTok{ eusilc}
\NormalTok{    )}

\CommentTok{# immediately run the convey_prep function on it}
\NormalTok{des_eusilc <-}\StringTok{ }\KeywordTok{convey_prep}\NormalTok{( des_eusilc )}

\CommentTok{# coefficients do match}
\NormalTok{varpoord_arpt_calculation}\OperatorTok{$}\NormalTok{all_result}\OperatorTok{$}\NormalTok{value}
\end{Highlighting}
\end{Shaded}

\begin{verbatim}
## [1] 10859.24
\end{verbatim}

\begin{Shaded}
\begin{Highlighting}[]
\KeywordTok{coef}\NormalTok{( }\KeywordTok{svyarpt}\NormalTok{( }\OperatorTok{~}\StringTok{ }\NormalTok{eqincome , des_eusilc ) )}
\end{Highlighting}
\end{Shaded}

\begin{verbatim}
## eqincome 
## 10859.24
\end{verbatim}

\begin{Shaded}
\begin{Highlighting}[]
\CommentTok{# linearized variables do match}
\CommentTok{# vardpoor}
\NormalTok{lin_arpt_varpoord<-}\StringTok{ }\NormalTok{varpoord_arpt_calculation}\OperatorTok{$}\NormalTok{lin_out}\OperatorTok{$}\NormalTok{lin_arpt}
\CommentTok{# convey }
\NormalTok{lin_arpt_convey <-}\StringTok{ }\KeywordTok{attr}\NormalTok{(}\KeywordTok{svyarpt}\NormalTok{( }\OperatorTok{~}\StringTok{ }\NormalTok{eqincome , des_eusilc ),}\StringTok{"lin"}\NormalTok{)}

\CommentTok{# check equality}
\KeywordTok{all.equal}\NormalTok{(lin_arpt_varpoord, lin_arpt_convey )}
\end{Highlighting}
\end{Shaded}

\begin{verbatim}
## [1] TRUE
\end{verbatim}

\begin{Shaded}
\begin{Highlighting}[]
\CommentTok{# variances do not match exactly}
\KeywordTok{attr}\NormalTok{( }\KeywordTok{svyarpt}\NormalTok{( }\OperatorTok{~}\StringTok{ }\NormalTok{eqincome , des_eusilc ) , }\StringTok{'var'}\NormalTok{ )}
\end{Highlighting}
\end{Shaded}

\begin{verbatim}
##          eqincome
## eqincome 2564.027
\end{verbatim}

\begin{Shaded}
\begin{Highlighting}[]
\NormalTok{varpoord_arpt_calculation}\OperatorTok{$}\NormalTok{all_result}\OperatorTok{$}\NormalTok{var}
\end{Highlighting}
\end{Shaded}

\begin{verbatim}
## [1] 2559.442
\end{verbatim}

\begin{Shaded}
\begin{Highlighting}[]
\CommentTok{# standard errors do not match exactly}
\NormalTok{varpoord_arpt_calculation}\OperatorTok{$}\NormalTok{all_result}\OperatorTok{$}\NormalTok{se}
\end{Highlighting}
\end{Shaded}

\begin{verbatim}
## [1] 50.59093
\end{verbatim}

\begin{Shaded}
\begin{Highlighting}[]
\KeywordTok{SE}\NormalTok{( }\KeywordTok{svyarpt}\NormalTok{( }\OperatorTok{~}\StringTok{ }\NormalTok{eqincome , des_eusilc ) )}
\end{Highlighting}
\end{Shaded}

\begin{verbatim}
##          eqincome
## eqincome 50.63622
\end{verbatim}

The variance estimate is computed by using the approximation defined in
\eqref{eq:var}, where the linearized variable \(z\) is defined by
\eqref{eq:lin}. The functions \texttt{convey::svyarpt} and
\texttt{vardpoor::linarpt} produce the same linearized variable \(z\).

However, the measures of uncertainty do not line up, because
\texttt{library(vardpoor)} defaults to an ultimate cluster method that
can be replicated with an alternative setup of the
\texttt{survey.design} object.

\begin{Shaded}
\begin{Highlighting}[]
\CommentTok{# within each strata, sum up the weights}
\NormalTok{cluster_sums <-}\StringTok{ }\KeywordTok{aggregate}\NormalTok{( eusilc}\OperatorTok{$}\NormalTok{rb050 , }\KeywordTok{list}\NormalTok{( eusilc}\OperatorTok{$}\NormalTok{db040 ) , sum )}

\CommentTok{# name the within-strata sums of weights the `cluster_sum`}
\KeywordTok{names}\NormalTok{( cluster_sums ) <-}\StringTok{ }\KeywordTok{c}\NormalTok{( }\StringTok{"db040"}\NormalTok{ , }\StringTok{"cluster_sum"}\NormalTok{ )}

\CommentTok{# merge this column back onto the data.frame}
\NormalTok{eusilc <-}\StringTok{ }\KeywordTok{merge}\NormalTok{( eusilc , cluster_sums )}

\CommentTok{# construct a survey.design}
\CommentTok{# with the fpc using the cluster sum}
\NormalTok{des_eusilc_ultimate_cluster <-}\StringTok{ }
\StringTok{    }\KeywordTok{svydesign}\NormalTok{( }
        \DataTypeTok{ids =} \OperatorTok{~}\StringTok{ }\NormalTok{rb030 , }
        \DataTypeTok{strata =} \OperatorTok{~}\StringTok{ }\NormalTok{db040 ,  }
        \DataTypeTok{weights =} \OperatorTok{~}\StringTok{ }\NormalTok{rb050 , }
        \DataTypeTok{data =}\NormalTok{ eusilc , }
        \DataTypeTok{fpc =} \OperatorTok{~}\StringTok{ }\NormalTok{cluster_sum }
\NormalTok{    )}

\CommentTok{# again, immediately run the convey_prep function on the `survey.design`}
\NormalTok{des_eusilc_ultimate_cluster <-}\StringTok{ }\KeywordTok{convey_prep}\NormalTok{( des_eusilc_ultimate_cluster )}

\CommentTok{# matches}
\KeywordTok{attr}\NormalTok{( }\KeywordTok{svyarpt}\NormalTok{( }\OperatorTok{~}\StringTok{ }\NormalTok{eqincome , des_eusilc_ultimate_cluster ) , }\StringTok{'var'}\NormalTok{ )}
\end{Highlighting}
\end{Shaded}

\begin{verbatim}
##          eqincome
## eqincome 2559.442
\end{verbatim}

\begin{Shaded}
\begin{Highlighting}[]
\NormalTok{varpoord_arpt_calculation}\OperatorTok{$}\NormalTok{all_result}\OperatorTok{$}\NormalTok{var}
\end{Highlighting}
\end{Shaded}

\begin{verbatim}
## [1] 2559.442
\end{verbatim}

\begin{Shaded}
\begin{Highlighting}[]
\CommentTok{# matches}
\NormalTok{varpoord_arpt_calculation}\OperatorTok{$}\NormalTok{all_result}\OperatorTok{$}\NormalTok{se}
\end{Highlighting}
\end{Shaded}

\begin{verbatim}
## [1] 50.59093
\end{verbatim}

\begin{Shaded}
\begin{Highlighting}[]
\KeywordTok{SE}\NormalTok{( }\KeywordTok{svyarpt}\NormalTok{( }\OperatorTok{~}\StringTok{ }\NormalTok{eqincome , des_eusilc_ultimate_cluster ) )}
\end{Highlighting}
\end{Shaded}

\begin{verbatim}
##          eqincome
## eqincome 50.59093
\end{verbatim}

For additional usage examples of \texttt{svyarpt}, type
\texttt{?convey::svyarpt} in the R console.

\section{Relative Median Income Ratio
(svyrmir)}\label{relative-median-income-ratio-svyrmir}

The relative median income ratio (rmir) is the ratio of the median
income of people aged above a value (65) to the median of people aged
below the same value.

\[
rmir = \frac{median\{y_i; age_i >65 \}}{median\{y_i; age_i \leq 65 \}}
\]

The details of the linearization of the \texttt{rmir} and are discussed
by \citep{deville1999} and \citep{osier2009}.

\begin{center}\rule{0.5\linewidth}{\linethickness}\end{center}

\textbf{A replication example}

The R \texttt{vardpoor} package \citep{vardpoor}, created by researchers
at the Central Statistical Bureau of Latvia, includes a rmir coefficient
calculation using the ultimate cluster method. The example below
reproduces those statistics.

Load and prepare the same data set:

\begin{Shaded}
\begin{Highlighting}[]
\CommentTok{# load the convey package}
\KeywordTok{library}\NormalTok{(convey)}

\CommentTok{# load the survey library}
\KeywordTok{library}\NormalTok{(survey)}

\CommentTok{# load the vardpoor library}
\KeywordTok{library}\NormalTok{(vardpoor)}

\CommentTok{# load the synthetic european union statistics on income & living conditions}
\KeywordTok{data}\NormalTok{(eusilc)}

\CommentTok{# make all column names lowercase}
\KeywordTok{names}\NormalTok{( eusilc ) <-}\StringTok{ }\KeywordTok{tolower}\NormalTok{( }\KeywordTok{names}\NormalTok{( eusilc ) )}

\CommentTok{# add a column with the row number}
\NormalTok{dati <-}\StringTok{ }\KeywordTok{data.table}\NormalTok{(}\DataTypeTok{IDd =} \DecValTok{1} \OperatorTok{:}\StringTok{ }\KeywordTok{nrow}\NormalTok{(eusilc), eusilc)}

\CommentTok{# calculate the rmir coefficient}
\CommentTok{# using the R vardpoor library}
\NormalTok{varpoord_rmir_calculation <-}
\StringTok{    }\KeywordTok{varpoord}\NormalTok{(}
    
        \CommentTok{# analysis variable}
        \DataTypeTok{Y =} \StringTok{"eqincome"}\NormalTok{, }
        
        \CommentTok{# weights variable}
        \DataTypeTok{w_final =} \StringTok{"rb050"}\NormalTok{,}
        
        \CommentTok{# row number variable}
        \DataTypeTok{ID_level1 =} \StringTok{"IDd"}\NormalTok{,}
        
        \CommentTok{# row number variable}
        \DataTypeTok{ID_level2 =} \StringTok{"IDd"}\NormalTok{,}
        
        \CommentTok{# strata variable}
        \DataTypeTok{H =} \StringTok{"db040"}\NormalTok{, }
        
        \DataTypeTok{N_h =} \OtherTok{NULL}\NormalTok{ ,}
        
        \CommentTok{# clustering variable}
        \DataTypeTok{PSU =} \StringTok{"rb030"}\NormalTok{, }
        
        \CommentTok{# data.table}
        \DataTypeTok{dataset =}\NormalTok{ dati,}
      
      \CommentTok{# age variable}
      \DataTypeTok{age =} \StringTok{"age"}\NormalTok{,}
        
        \CommentTok{# rmir coefficient function}
        \DataTypeTok{type =} \StringTok{"linrmir"}\NormalTok{,}
      
      \CommentTok{# poverty threshold range}
      \DataTypeTok{order_quant =}\NormalTok{ 50L ,}
      
      \CommentTok{# get linearized variable}
      \DataTypeTok{outp_lin =} \OtherTok{TRUE}
        
\NormalTok{    )}
\end{Highlighting}
\end{Shaded}

\begin{verbatim}
## NULL
\end{verbatim}

\begin{Shaded}
\begin{Highlighting}[]
\CommentTok{# construct a survey.design}
\CommentTok{# using our recommended setup}
\NormalTok{des_eusilc <-}\StringTok{ }
\StringTok{    }\KeywordTok{svydesign}\NormalTok{( }
        \DataTypeTok{ids =} \OperatorTok{~}\StringTok{ }\NormalTok{rb030 , }
        \DataTypeTok{strata =} \OperatorTok{~}\StringTok{ }\NormalTok{db040 ,  }
        \DataTypeTok{weights =} \OperatorTok{~}\StringTok{ }\NormalTok{rb050 , }
        \DataTypeTok{data =}\NormalTok{ eusilc}
\NormalTok{    )}

\CommentTok{# immediately run the convey_prep function on it}
\NormalTok{des_eusilc <-}\StringTok{ }\KeywordTok{convey_prep}\NormalTok{( des_eusilc )}

\CommentTok{# coefficients do match}
\NormalTok{varpoord_rmir_calculation}\OperatorTok{$}\NormalTok{all_result}\OperatorTok{$}\NormalTok{value}
\end{Highlighting}
\end{Shaded}

\begin{verbatim}
## [1] 0.9330361
\end{verbatim}

\begin{Shaded}
\begin{Highlighting}[]
\KeywordTok{coef}\NormalTok{( }\KeywordTok{svyrmir}\NormalTok{( }\OperatorTok{~}\StringTok{ }\NormalTok{eqincome , des_eusilc, }\DataTypeTok{age =} \OperatorTok{~}\NormalTok{age ) ) }
\end{Highlighting}
\end{Shaded}

\begin{verbatim}
##  eqincome 
## 0.9330361
\end{verbatim}

\begin{Shaded}
\begin{Highlighting}[]
\CommentTok{# linearized variables do match}
\CommentTok{# vardpoor}
\NormalTok{lin_rmir_varpoord<-}\StringTok{ }\NormalTok{varpoord_rmir_calculation}\OperatorTok{$}\NormalTok{lin_out}\OperatorTok{$}\NormalTok{lin_rmir}
\CommentTok{# convey }
\NormalTok{lin_rmir_convey <-}\StringTok{ }\KeywordTok{attr}\NormalTok{(}\KeywordTok{svyrmir}\NormalTok{( }\OperatorTok{~}\StringTok{ }\NormalTok{eqincome , des_eusilc, }\DataTypeTok{age =} \OperatorTok{~}\NormalTok{age ),}\StringTok{"lin"}\NormalTok{)}

\CommentTok{# check equality}
\KeywordTok{all.equal}\NormalTok{(lin_rmir_varpoord, lin_rmir_convey[,}\DecValTok{1}\NormalTok{] )}
\end{Highlighting}
\end{Shaded}

\begin{verbatim}
## [1] TRUE
\end{verbatim}

\begin{Shaded}
\begin{Highlighting}[]
\CommentTok{# variances do not match exactly}
\KeywordTok{attr}\NormalTok{( }\KeywordTok{svyrmir}\NormalTok{( }\OperatorTok{~}\StringTok{ }\NormalTok{eqincome , des_eusilc, }\DataTypeTok{age =} \OperatorTok{~}\NormalTok{age ) , }\StringTok{'var'}\NormalTok{ ) }
\end{Highlighting}
\end{Shaded}

\begin{verbatim}
##             eqincome
## eqincome 0.000127444
\end{verbatim}

\begin{Shaded}
\begin{Highlighting}[]
\NormalTok{varpoord_rmir_calculation}\OperatorTok{$}\NormalTok{all_result}\OperatorTok{$}\NormalTok{var}
\end{Highlighting}
\end{Shaded}

\begin{verbatim}
## [1] 0.0001272137
\end{verbatim}

\begin{Shaded}
\begin{Highlighting}[]
\CommentTok{# standard errors do not match exactly}
\NormalTok{varpoord_rmir_calculation}\OperatorTok{$}\NormalTok{all_result}\OperatorTok{$}\NormalTok{se}
\end{Highlighting}
\end{Shaded}

\begin{verbatim}
## [1] 0.0112789
\end{verbatim}

\begin{Shaded}
\begin{Highlighting}[]
\KeywordTok{SE}\NormalTok{( }\KeywordTok{svyrmir}\NormalTok{( }\OperatorTok{~}\StringTok{ }\NormalTok{eqincome , des_eusilc , }\DataTypeTok{age =} \OperatorTok{~}\NormalTok{age) ) }
\end{Highlighting}
\end{Shaded}

\begin{verbatim}
##            eqincome
## eqincome 0.01128911
\end{verbatim}

The variance estimate is computed by using the approximation defined in
\eqref{eq:var}, where the linearized variable \(z\) is defined by
\eqref{eq:lin}. The functions \texttt{convey::svyrmir} and
\texttt{vardpoor::linrmir} produce the same linearized variable \(z\).

However, the measures of uncertainty do not line up, because
\texttt{library(vardpoor)} defaults to an ultimate cluster method that
can be replicated with an alternative setup of the
\texttt{survey.design} object.

\begin{Shaded}
\begin{Highlighting}[]
\CommentTok{# within each strata, sum up the weights}
\NormalTok{cluster_sums <-}\StringTok{ }\KeywordTok{aggregate}\NormalTok{( eusilc}\OperatorTok{$}\NormalTok{rb050 , }\KeywordTok{list}\NormalTok{( eusilc}\OperatorTok{$}\NormalTok{db040 ) , sum )}

\CommentTok{# name the within-strata sums of weights the `cluster_sum`}
\KeywordTok{names}\NormalTok{( cluster_sums ) <-}\StringTok{ }\KeywordTok{c}\NormalTok{( }\StringTok{"db040"}\NormalTok{ , }\StringTok{"cluster_sum"}\NormalTok{ )}

\CommentTok{# merge this column back onto the data.frame}
\NormalTok{eusilc <-}\StringTok{ }\KeywordTok{merge}\NormalTok{( eusilc , cluster_sums )}

\CommentTok{# construct a survey.design}
\CommentTok{# with the fpc using the cluster sum}
\NormalTok{des_eusilc_ultimate_cluster <-}\StringTok{ }
\StringTok{    }\KeywordTok{svydesign}\NormalTok{( }
        \DataTypeTok{ids =} \OperatorTok{~}\StringTok{ }\NormalTok{rb030 , }
        \DataTypeTok{strata =} \OperatorTok{~}\StringTok{ }\NormalTok{db040 ,  }
        \DataTypeTok{weights =} \OperatorTok{~}\StringTok{ }\NormalTok{rb050 , }
        \DataTypeTok{data =}\NormalTok{ eusilc , }
        \DataTypeTok{fpc =} \OperatorTok{~}\StringTok{ }\NormalTok{cluster_sum }
\NormalTok{    )}

\CommentTok{# again, immediately run the convey_prep function on the `survey.design`}
\NormalTok{des_eusilc_ultimate_cluster <-}\StringTok{ }\KeywordTok{convey_prep}\NormalTok{( des_eusilc_ultimate_cluster )}

\CommentTok{# matches}
\KeywordTok{attr}\NormalTok{( }\KeywordTok{svyrmir}\NormalTok{( }\OperatorTok{~}\StringTok{ }\NormalTok{eqincome , des_eusilc_ultimate_cluster , }\DataTypeTok{age =} \OperatorTok{~}\NormalTok{age ) , }\StringTok{'var'}\NormalTok{ ) }
\end{Highlighting}
\end{Shaded}

\begin{verbatim}
##              eqincome
## eqincome 0.0001272137
\end{verbatim}

\begin{Shaded}
\begin{Highlighting}[]
\NormalTok{varpoord_rmir_calculation}\OperatorTok{$}\NormalTok{all_result}\OperatorTok{$}\NormalTok{var}
\end{Highlighting}
\end{Shaded}

\begin{verbatim}
## [1] 0.0001272137
\end{verbatim}

\begin{Shaded}
\begin{Highlighting}[]
\CommentTok{# matches}
\NormalTok{varpoord_rmir_calculation}\OperatorTok{$}\NormalTok{all_result}\OperatorTok{$}\NormalTok{se}
\end{Highlighting}
\end{Shaded}

\begin{verbatim}
## [1] 0.0112789
\end{verbatim}

\begin{Shaded}
\begin{Highlighting}[]
\KeywordTok{SE}\NormalTok{( }\KeywordTok{svyrmir}\NormalTok{( }\OperatorTok{~}\StringTok{ }\NormalTok{eqincome , des_eusilc_ultimate_cluster, }\DataTypeTok{age =} \OperatorTok{~}\NormalTok{age ) ) }
\end{Highlighting}
\end{Shaded}

\begin{verbatim}
##           eqincome
## eqincome 0.0112789
\end{verbatim}

For additional usage examples of \texttt{svyrmir}, type
\texttt{?convey::svyrmir} in the R console.

\section{Relative Median Poverty Gap
(svyrmpg)}\label{relative-median-poverty-gap-svyrmpg}

The relative median poverty gap (\texttt{rmpg}) is the relative
difference between the median income of people having income below the
\texttt{arpt} and the \texttt{arpt} itself:

\[
rmpg = \frac{median\{y_i, y_i<arpt\}-arpt}{arpt}
\] The details of the linearization of the \texttt{rmpg} are discussed
by \citep{deville1999} and \citep{osier2009}.

\begin{center}\rule{0.5\linewidth}{\linethickness}\end{center}

\textbf{A replication example}

The R \texttt{vardpoor} package \citep{vardpoor}, created by researchers
at the Central Statistical Bureau of Latvia, includes a rmpg coefficient
calculation using the ultimate cluster method. The example below
reproduces those statistics.

Load and prepare the same data set:

\begin{Shaded}
\begin{Highlighting}[]
\CommentTok{# load the convey package}
\KeywordTok{library}\NormalTok{(convey)}

\CommentTok{# load the survey library}
\KeywordTok{library}\NormalTok{(survey)}

\CommentTok{# load the vardpoor library}
\KeywordTok{library}\NormalTok{(vardpoor)}

\CommentTok{# load the synthetic european union statistics on income & living conditions}
\KeywordTok{data}\NormalTok{(eusilc)}

\CommentTok{# make all column names lowercase}
\KeywordTok{names}\NormalTok{( eusilc ) <-}\StringTok{ }\KeywordTok{tolower}\NormalTok{( }\KeywordTok{names}\NormalTok{( eusilc ) )}

\CommentTok{# add a column with the row number}
\NormalTok{dati <-}\StringTok{ }\KeywordTok{data.table}\NormalTok{(}\DataTypeTok{IDd =} \DecValTok{1} \OperatorTok{:}\StringTok{ }\KeywordTok{nrow}\NormalTok{(eusilc), eusilc)}

\CommentTok{# calculate the rmpg coefficient}
\CommentTok{# using the R vardpoor library}
\NormalTok{varpoord_rmpg_calculation <-}
\StringTok{    }\KeywordTok{varpoord}\NormalTok{(}
    
        \CommentTok{# analysis variable}
        \DataTypeTok{Y =} \StringTok{"eqincome"}\NormalTok{, }
        
        \CommentTok{# weights variable}
        \DataTypeTok{w_final =} \StringTok{"rb050"}\NormalTok{,}
        
        \CommentTok{# row number variable}
        \DataTypeTok{ID_level1 =} \StringTok{"IDd"}\NormalTok{,}

        \CommentTok{# row number variable}
        \DataTypeTok{ID_level2 =} \StringTok{"IDd"}\NormalTok{,}
                
        \CommentTok{# strata variable}
        \DataTypeTok{H =} \StringTok{"db040"}\NormalTok{, }
        
        \DataTypeTok{N_h =} \OtherTok{NULL}\NormalTok{ ,}
        
        \CommentTok{# clustering variable}
        \DataTypeTok{PSU =} \StringTok{"rb030"}\NormalTok{, }
        
        \CommentTok{# data.table}
        \DataTypeTok{dataset =}\NormalTok{ dati, }
        
        \CommentTok{# rmpg coefficient function}
        \DataTypeTok{type =} \StringTok{"linrmpg"}\NormalTok{,}
      
      \CommentTok{# poverty threshold range}
      \DataTypeTok{order_quant =}\NormalTok{ 50L ,}
      
      \CommentTok{# get linearized variable}
      \DataTypeTok{outp_lin =} \OtherTok{TRUE}
        
\NormalTok{    )}
\end{Highlighting}
\end{Shaded}

\begin{verbatim}
## NULL
\end{verbatim}

\begin{Shaded}
\begin{Highlighting}[]
\CommentTok{# construct a survey.design}
\CommentTok{# using our recommended setup}
\NormalTok{des_eusilc <-}\StringTok{ }
\StringTok{    }\KeywordTok{svydesign}\NormalTok{( }
        \DataTypeTok{ids =} \OperatorTok{~}\StringTok{ }\NormalTok{rb030 , }
        \DataTypeTok{strata =} \OperatorTok{~}\StringTok{ }\NormalTok{db040 ,  }
        \DataTypeTok{weights =} \OperatorTok{~}\StringTok{ }\NormalTok{rb050 , }
        \DataTypeTok{data =}\NormalTok{ eusilc}
\NormalTok{    )}

\CommentTok{# immediately run the convey_prep function on it}
\NormalTok{des_eusilc <-}\StringTok{ }\KeywordTok{convey_prep}\NormalTok{( des_eusilc )}

\CommentTok{# coefficients do match}
\NormalTok{varpoord_rmpg_calculation}\OperatorTok{$}\NormalTok{all_result}\OperatorTok{$}\NormalTok{value}
\end{Highlighting}
\end{Shaded}

\begin{verbatim}
## [1] 18.9286
\end{verbatim}

\begin{Shaded}
\begin{Highlighting}[]
\KeywordTok{coef}\NormalTok{( }\KeywordTok{svyrmpg}\NormalTok{( }\OperatorTok{~}\StringTok{ }\NormalTok{eqincome , des_eusilc ) ) }\OperatorTok{*}\StringTok{ }\DecValTok{100}
\end{Highlighting}
\end{Shaded}

\begin{verbatim}
## eqincome 
##  18.9286
\end{verbatim}

\begin{Shaded}
\begin{Highlighting}[]
\CommentTok{# linearized variables do match}
\CommentTok{# vardpoor}
\NormalTok{lin_rmpg_varpoord<-}\StringTok{ }\NormalTok{varpoord_rmpg_calculation}\OperatorTok{$}\NormalTok{lin_out}\OperatorTok{$}\NormalTok{lin_rmpg}
\CommentTok{# convey }
\NormalTok{lin_rmpg_convey <-}\StringTok{ }\KeywordTok{attr}\NormalTok{(}\KeywordTok{svyrmpg}\NormalTok{( }\OperatorTok{~}\StringTok{ }\NormalTok{eqincome , des_eusilc ),}\StringTok{"lin"}\NormalTok{)}

\CommentTok{# check equality}
\KeywordTok{all.equal}\NormalTok{(lin_rmpg_varpoord, }\DecValTok{100}\OperatorTok{*}\NormalTok{lin_rmpg_convey[,}\DecValTok{1}\NormalTok{] )}
\end{Highlighting}
\end{Shaded}

\begin{verbatim}
## [1] TRUE
\end{verbatim}

\begin{Shaded}
\begin{Highlighting}[]
\CommentTok{# variances do not match exactly}
\KeywordTok{attr}\NormalTok{( }\KeywordTok{svyrmpg}\NormalTok{( }\OperatorTok{~}\StringTok{ }\NormalTok{eqincome , des_eusilc ) , }\StringTok{'var'}\NormalTok{ ) }\OperatorTok{*}\StringTok{ }\DecValTok{10000}
\end{Highlighting}
\end{Shaded}

\begin{verbatim}
##          eqincome
## eqincome 0.332234
\end{verbatim}

\begin{Shaded}
\begin{Highlighting}[]
\NormalTok{varpoord_rmpg_calculation}\OperatorTok{$}\NormalTok{all_result}\OperatorTok{$}\NormalTok{var}
\end{Highlighting}
\end{Shaded}

\begin{verbatim}
## [1] 0.3316454
\end{verbatim}

\begin{Shaded}
\begin{Highlighting}[]
\CommentTok{# standard errors do not match exactly}
\NormalTok{varpoord_rmpg_calculation}\OperatorTok{$}\NormalTok{all_result}\OperatorTok{$}\NormalTok{se}
\end{Highlighting}
\end{Shaded}

\begin{verbatim}
## [1] 0.5758866
\end{verbatim}

\begin{Shaded}
\begin{Highlighting}[]
\KeywordTok{SE}\NormalTok{( }\KeywordTok{svyrmpg}\NormalTok{( }\OperatorTok{~}\StringTok{ }\NormalTok{eqincome , des_eusilc ) ) }\OperatorTok{*}\StringTok{ }\DecValTok{100}
\end{Highlighting}
\end{Shaded}

\begin{verbatim}
##           eqincome
## eqincome 0.5763974
\end{verbatim}

The variance estimate is computed by using the approximation defined in
\eqref{eq:var}, where the linearized variable \(z\) is defined by
\eqref{eq:lin}. The functions \texttt{convey::svyrmpg} and
\texttt{vardpoor::linrmpg} produce the same linearized variable \(z\).

However, the measures of uncertainty do not line up, because
\texttt{library(vardpoor)} defaults to an ultimate cluster method that
can be replicated with an alternative setup of the
\texttt{survey.design} object.

\begin{Shaded}
\begin{Highlighting}[]
\CommentTok{# within each strata, sum up the weights}
\NormalTok{cluster_sums <-}\StringTok{ }\KeywordTok{aggregate}\NormalTok{( eusilc}\OperatorTok{$}\NormalTok{rb050 , }\KeywordTok{list}\NormalTok{( eusilc}\OperatorTok{$}\NormalTok{db040 ) , sum )}

\CommentTok{# name the within-strata sums of weights the `cluster_sum`}
\KeywordTok{names}\NormalTok{( cluster_sums ) <-}\StringTok{ }\KeywordTok{c}\NormalTok{( }\StringTok{"db040"}\NormalTok{ , }\StringTok{"cluster_sum"}\NormalTok{ )}

\CommentTok{# merge this column back onto the data.frame}
\NormalTok{eusilc <-}\StringTok{ }\KeywordTok{merge}\NormalTok{( eusilc , cluster_sums )}

\CommentTok{# construct a survey.design}
\CommentTok{# with the fpc using the cluster sum}
\NormalTok{des_eusilc_ultimate_cluster <-}\StringTok{ }
\StringTok{    }\KeywordTok{svydesign}\NormalTok{( }
        \DataTypeTok{ids =} \OperatorTok{~}\StringTok{ }\NormalTok{rb030 , }
        \DataTypeTok{strata =} \OperatorTok{~}\StringTok{ }\NormalTok{db040 ,  }
        \DataTypeTok{weights =} \OperatorTok{~}\StringTok{ }\NormalTok{rb050 , }
        \DataTypeTok{data =}\NormalTok{ eusilc , }
        \DataTypeTok{fpc =} \OperatorTok{~}\StringTok{ }\NormalTok{cluster_sum }
\NormalTok{    )}

\CommentTok{# again, immediately run the convey_prep function on the `survey.design`}
\NormalTok{des_eusilc_ultimate_cluster <-}\StringTok{ }\KeywordTok{convey_prep}\NormalTok{( des_eusilc_ultimate_cluster )}

\CommentTok{# matches}
\KeywordTok{attr}\NormalTok{( }\KeywordTok{svyrmpg}\NormalTok{( }\OperatorTok{~}\StringTok{ }\NormalTok{eqincome , des_eusilc_ultimate_cluster ) , }\StringTok{'var'}\NormalTok{ ) }\OperatorTok{*}\StringTok{ }\DecValTok{10000}
\end{Highlighting}
\end{Shaded}

\begin{verbatim}
##           eqincome
## eqincome 0.3316454
\end{verbatim}

\begin{Shaded}
\begin{Highlighting}[]
\NormalTok{varpoord_rmpg_calculation}\OperatorTok{$}\NormalTok{all_result}\OperatorTok{$}\NormalTok{var}
\end{Highlighting}
\end{Shaded}

\begin{verbatim}
## [1] 0.3316454
\end{verbatim}

\begin{Shaded}
\begin{Highlighting}[]
\CommentTok{# matches}
\NormalTok{varpoord_rmpg_calculation}\OperatorTok{$}\NormalTok{all_result}\OperatorTok{$}\NormalTok{se}
\end{Highlighting}
\end{Shaded}

\begin{verbatim}
## [1] 0.5758866
\end{verbatim}

\begin{Shaded}
\begin{Highlighting}[]
\KeywordTok{SE}\NormalTok{( }\KeywordTok{svyrmpg}\NormalTok{( }\OperatorTok{~}\StringTok{ }\NormalTok{eqincome , des_eusilc_ultimate_cluster ) ) }\OperatorTok{*}\StringTok{ }\DecValTok{100}
\end{Highlighting}
\end{Shaded}

\begin{verbatim}
##           eqincome
## eqincome 0.5758866
\end{verbatim}

For additional usage examples of \texttt{svyrmpg}, type
\texttt{?convey::svyrmpg} in the R console.

\section{Median Income Below the At Risk of Poverty Threshold
(svypoormed)}\label{median-income-below-the-at-risk-of-poverty-threshold-svypoormed}

Median income below the at-risk-of-poverty- threshold (poormed) is
median of incomes of people having the income below the \texttt{arpt}:

\[
poormed = median\{y_i; y_i< arpt\}
\] The details of the linearization of the \texttt{poormed} are
discussed by \citep{deville1999} and \citep{osier2009}.

\begin{center}\rule{0.5\linewidth}{\linethickness}\end{center}

\textbf{A replication example}

The R \texttt{vardpoor} package \citep{vardpoor}, created by researchers
at the Central Statistical Bureau of Latvia, includes a poormed
coefficient calculation using the ultimate cluster method. The example
below reproduces those statistics.

Load and prepare the same data set:

\begin{Shaded}
\begin{Highlighting}[]
\CommentTok{# load the convey package}
\KeywordTok{library}\NormalTok{(convey)}

\CommentTok{# load the survey library}
\KeywordTok{library}\NormalTok{(survey)}

\CommentTok{# load the vardpoor library}
\KeywordTok{library}\NormalTok{(vardpoor)}

\CommentTok{# load the synthetic european union statistics on income & living conditions}
\KeywordTok{data}\NormalTok{(eusilc)}

\CommentTok{# make all column names lowercase}
\KeywordTok{names}\NormalTok{( eusilc ) <-}\StringTok{ }\KeywordTok{tolower}\NormalTok{( }\KeywordTok{names}\NormalTok{( eusilc ) )}

\CommentTok{# add a column with the row number}
\NormalTok{dati <-}\StringTok{ }\KeywordTok{data.table}\NormalTok{(}\DataTypeTok{IDd =} \DecValTok{1} \OperatorTok{:}\StringTok{ }\KeywordTok{nrow}\NormalTok{(eusilc), eusilc)}

\CommentTok{# calculate the poormed coefficient}
\CommentTok{# using the R vardpoor library}
\NormalTok{varpoord_poormed_calculation <-}
\StringTok{    }\KeywordTok{varpoord}\NormalTok{(}
    
        \CommentTok{# analysis variable}
        \DataTypeTok{Y =} \StringTok{"eqincome"}\NormalTok{, }
        
        \CommentTok{# weights variable}
        \DataTypeTok{w_final =} \StringTok{"rb050"}\NormalTok{,}
        
        \CommentTok{# row number variable}
        \DataTypeTok{ID_level1 =} \StringTok{"IDd"}\NormalTok{,}

        \CommentTok{# row number variable}
        \DataTypeTok{ID_level2 =} \StringTok{"IDd"}\NormalTok{,}
                
        \CommentTok{# strata variable}
        \DataTypeTok{H =} \StringTok{"db040"}\NormalTok{, }
        
        \DataTypeTok{N_h =} \OtherTok{NULL}\NormalTok{ ,}
        
        \CommentTok{# clustering variable}
        \DataTypeTok{PSU =} \StringTok{"rb030"}\NormalTok{, }
        
        \CommentTok{# data.table}
        \DataTypeTok{dataset =}\NormalTok{ dati, }
        
        \CommentTok{# poormed coefficient function}
        \DataTypeTok{type =} \StringTok{"linpoormed"}\NormalTok{,}
      
      \CommentTok{# poverty threshold range}
      \DataTypeTok{order_quant =}\NormalTok{ 50L ,}
      
      \CommentTok{# get linearized variable}
      \DataTypeTok{outp_lin =} \OtherTok{TRUE}
        
\NormalTok{    )}
\end{Highlighting}
\end{Shaded}

\begin{verbatim}
## NULL
\end{verbatim}

\begin{Shaded}
\begin{Highlighting}[]
\CommentTok{# construct a survey.design}
\CommentTok{# using our recommended setup}
\NormalTok{des_eusilc <-}\StringTok{ }
\StringTok{    }\KeywordTok{svydesign}\NormalTok{( }
        \DataTypeTok{ids =} \OperatorTok{~}\StringTok{ }\NormalTok{rb030 , }
        \DataTypeTok{strata =} \OperatorTok{~}\StringTok{ }\NormalTok{db040 ,  }
        \DataTypeTok{weights =} \OperatorTok{~}\StringTok{ }\NormalTok{rb050 , }
        \DataTypeTok{data =}\NormalTok{ eusilc}
\NormalTok{    )}

\CommentTok{# immediately run the convey_prep function on it}
\NormalTok{des_eusilc <-}\StringTok{ }\KeywordTok{convey_prep}\NormalTok{( des_eusilc )}

\CommentTok{# coefficients do match}
\NormalTok{varpoord_poormed_calculation}\OperatorTok{$}\NormalTok{all_result}\OperatorTok{$}\NormalTok{value}
\end{Highlighting}
\end{Shaded}

\begin{verbatim}
## [1] 8803.735
\end{verbatim}

\begin{Shaded}
\begin{Highlighting}[]
\KeywordTok{coef}\NormalTok{( }\KeywordTok{svypoormed}\NormalTok{( }\OperatorTok{~}\StringTok{ }\NormalTok{eqincome , des_eusilc ) )}
\end{Highlighting}
\end{Shaded}

\begin{verbatim}
## eqincome 
## 8803.735
\end{verbatim}

\begin{Shaded}
\begin{Highlighting}[]
\CommentTok{# linearized variables do match}
\CommentTok{# vardpoor}
\NormalTok{lin_poormed_varpoord<-}\StringTok{ }\NormalTok{varpoord_poormed_calculation}\OperatorTok{$}\NormalTok{lin_out}\OperatorTok{$}\NormalTok{lin_poormed}
\CommentTok{# convey }
\NormalTok{lin_poormed_convey <-}\StringTok{ }\KeywordTok{attr}\NormalTok{(}\KeywordTok{svypoormed}\NormalTok{( }\OperatorTok{~}\StringTok{ }\NormalTok{eqincome , des_eusilc ),}\StringTok{"lin"}\NormalTok{)}

\CommentTok{# check equality}
\KeywordTok{all.equal}\NormalTok{(lin_poormed_varpoord, lin_poormed_convey )}
\end{Highlighting}
\end{Shaded}

\begin{verbatim}
## [1] TRUE
\end{verbatim}

\begin{Shaded}
\begin{Highlighting}[]
\CommentTok{# variances do not match exactly}
\KeywordTok{attr}\NormalTok{( }\KeywordTok{svypoormed}\NormalTok{( }\OperatorTok{~}\StringTok{ }\NormalTok{eqincome , des_eusilc ) , }\StringTok{'var'}\NormalTok{ )}
\end{Highlighting}
\end{Shaded}

\begin{verbatim}
##          eqincome
## eqincome  5311.47
\end{verbatim}

\begin{Shaded}
\begin{Highlighting}[]
\NormalTok{varpoord_poormed_calculation}\OperatorTok{$}\NormalTok{all_result}\OperatorTok{$}\NormalTok{var}
\end{Highlighting}
\end{Shaded}

\begin{verbatim}
## [1] 5302.086
\end{verbatim}

\begin{Shaded}
\begin{Highlighting}[]
\CommentTok{# standard errors do not match exactly}
\NormalTok{varpoord_poormed_calculation}\OperatorTok{$}\NormalTok{all_result}\OperatorTok{$}\NormalTok{se}
\end{Highlighting}
\end{Shaded}

\begin{verbatim}
## [1] 72.81542
\end{verbatim}

\begin{Shaded}
\begin{Highlighting}[]
\KeywordTok{SE}\NormalTok{( }\KeywordTok{svypoormed}\NormalTok{( }\OperatorTok{~}\StringTok{ }\NormalTok{eqincome , des_eusilc ) )}
\end{Highlighting}
\end{Shaded}

\begin{verbatim}
##          eqincome
## eqincome 72.87983
\end{verbatim}

The variance estimate is computed by using the approximation defined in
\eqref{eq:var}, where the linearized variable \(z\) is defined by
\eqref{eq:lin}. The functions \texttt{convey::svypoormed} and
\texttt{vardpoor::linpoormed} produce the same linearized variable
\(z\).

However, the measures of uncertainty do not line up, because
\texttt{library(vardpoor)} defaults to an ultimate cluster method that
can be replicated with an alternative setup of the
\texttt{survey.design} object.

\begin{Shaded}
\begin{Highlighting}[]
\CommentTok{# within each strata, sum up the weights}
\NormalTok{cluster_sums <-}\StringTok{ }\KeywordTok{aggregate}\NormalTok{( eusilc}\OperatorTok{$}\NormalTok{rb050 , }\KeywordTok{list}\NormalTok{( eusilc}\OperatorTok{$}\NormalTok{db040 ) , sum )}

\CommentTok{# name the within-strata sums of weights the `cluster_sum`}
\KeywordTok{names}\NormalTok{( cluster_sums ) <-}\StringTok{ }\KeywordTok{c}\NormalTok{( }\StringTok{"db040"}\NormalTok{ , }\StringTok{"cluster_sum"}\NormalTok{ )}

\CommentTok{# merge this column back onto the data.frame}
\NormalTok{eusilc <-}\StringTok{ }\KeywordTok{merge}\NormalTok{( eusilc , cluster_sums )}

\CommentTok{# construct a survey.design}
\CommentTok{# with the fpc using the cluster sum}
\NormalTok{des_eusilc_ultimate_cluster <-}\StringTok{ }
\StringTok{    }\KeywordTok{svydesign}\NormalTok{( }
        \DataTypeTok{ids =} \OperatorTok{~}\StringTok{ }\NormalTok{rb030 , }
        \DataTypeTok{strata =} \OperatorTok{~}\StringTok{ }\NormalTok{db040 ,  }
        \DataTypeTok{weights =} \OperatorTok{~}\StringTok{ }\NormalTok{rb050 , }
        \DataTypeTok{data =}\NormalTok{ eusilc , }
        \DataTypeTok{fpc =} \OperatorTok{~}\StringTok{ }\NormalTok{cluster_sum }
\NormalTok{    )}

\CommentTok{# again, immediately run the convey_prep function on the `survey.design`}
\NormalTok{des_eusilc_ultimate_cluster <-}\StringTok{ }\KeywordTok{convey_prep}\NormalTok{( des_eusilc_ultimate_cluster )}

\CommentTok{# matches}
\KeywordTok{attr}\NormalTok{( }\KeywordTok{svypoormed}\NormalTok{( }\OperatorTok{~}\StringTok{ }\NormalTok{eqincome , des_eusilc_ultimate_cluster ) , }\StringTok{'var'}\NormalTok{ )}
\end{Highlighting}
\end{Shaded}

\begin{verbatim}
##          eqincome
## eqincome 5302.086
\end{verbatim}

\begin{Shaded}
\begin{Highlighting}[]
\NormalTok{varpoord_poormed_calculation}\OperatorTok{$}\NormalTok{all_result}\OperatorTok{$}\NormalTok{var}
\end{Highlighting}
\end{Shaded}

\begin{verbatim}
## [1] 5302.086
\end{verbatim}

\begin{Shaded}
\begin{Highlighting}[]
\CommentTok{# matches}
\NormalTok{varpoord_poormed_calculation}\OperatorTok{$}\NormalTok{all_result}\OperatorTok{$}\NormalTok{se}
\end{Highlighting}
\end{Shaded}

\begin{verbatim}
## [1] 72.81542
\end{verbatim}

\begin{Shaded}
\begin{Highlighting}[]
\KeywordTok{SE}\NormalTok{( }\KeywordTok{svypoormed}\NormalTok{( }\OperatorTok{~}\StringTok{ }\NormalTok{eqincome , des_eusilc_ultimate_cluster ) )}
\end{Highlighting}
\end{Shaded}

\begin{verbatim}
##          eqincome
## eqincome 72.81542
\end{verbatim}

For additional usage examples of \texttt{svypoormed}, type
\texttt{?convey::svypoormed} in the R console.

\section{Foster-Greer-Thorbecke class
(svyfgt)}\label{foster-greer-thorbecke-class-svyfgt}

\citep{foster1984} proposed a family of indicators to measure poverty.
This class of \(FGT\) measures, can be defined as

\[
p=\frac{1}{N}\sum_{k\in U}h(y_{k},\theta ), 
\]

where

\[
h(y_{k},\theta )=\left[ \frac{(\theta -y_{k})}{\theta }\right] ^{\gamma
}\delta \left\{ y_{k}\leq \theta \right\} , 
\]

where: \(\theta\) is the poverty threshold; \(\delta\) the indicator
function that assigns value \(1\) if the condition
\(\{y_{k}\leq \theta \}\) is satisfied and \(0\) otherwise, and
\(\gamma\) is a non-negative constant.

If \(\gamma =0\), the \texttt{FGT\ index} equals the poverty headcount
ratio, which accounts for the spread of poverty. If \(\gamma =1\), FGT
is the mean of the normalized income shortfall of the poor. By doing so,
the measure takes into account both the spread and the intensity of
poverty. When \(\gamma =2\), the relative weight of larger shortfalls
increases even more, which yields a measure that accounts for poverty
severity, i.e., the inequality among the poor. This way, a transfer from
a poor person to an even poorer person would reduce the FGT(2),
\citep{foster1984}.

The poverty measure FGT is implemented in the library convey by the
function \texttt{svyfgt}. The argument \texttt{thresh\_type} of this
function defines the type of poverty threshold adopted. There are three
possible choices:

\begin{enumerate}
\def\labelenumi{\arabic{enumi}.}
\tightlist
\item
  \texttt{abs} -- fixed and given by the argument thresh\_value
\item
  \texttt{relq} -- a proportion of a quantile fixed by the argument
  \texttt{proportion} and the quantile is defined by the argument
  \texttt{order}.
\item
  \texttt{relm} -- a proportion of the mean fixed the argument
  \texttt{proportion}
\end{enumerate}

The quantile and the mean involved in the definition of the threshold
are estimated for the whole population. When \(\gamma=0\) and
\(\theta= .6*MED\) the measure is equal to the indicator \texttt{arpr}
computed by the function \texttt{svyarpr}. The linearization of the FGT
is presented in \citep{berger2003}.

Next, we give some examples of the function \texttt{svyfgt} to estimate
the values of the FGT poverty index.

Consider first the poverty threshold fixed (\(\gamma=0\)) in the value
\(10000\). The headcount ratio (FGT0) is

\begin{Shaded}
\begin{Highlighting}[]
\KeywordTok{svyfgt}\NormalTok{(}\OperatorTok{~}\NormalTok{eqincome, des_eusilc, }\DataTypeTok{g=}\DecValTok{0}\NormalTok{, }\DataTypeTok{abs_thresh=}\DecValTok{10000}\NormalTok{)}
\end{Highlighting}
\end{Shaded}

\begin{verbatim}
            fgt0     SE
eqincome 0.11444 0.0027
\end{verbatim}

The poverty gap (FGT1) (\(\gamma=1\)) index for the poverty threshold
fixed at the same value is

\begin{Shaded}
\begin{Highlighting}[]
\KeywordTok{svyfgt}\NormalTok{(}\OperatorTok{~}\NormalTok{eqincome, des_eusilc, }\DataTypeTok{g=}\DecValTok{1}\NormalTok{, }\DataTypeTok{abs_thresh=}\DecValTok{10000}\NormalTok{)}
\end{Highlighting}
\end{Shaded}

\begin{verbatim}
             fgt1     SE
eqincome 0.032085 0.0011
\end{verbatim}

To estimate the FGT0 with the poverty threshold fixed at \(0.6* MED\) we
fix the argument type\_thresh=``relq'' and use the default values for
\texttt{percent} and \texttt{order}:

\begin{Shaded}
\begin{Highlighting}[]
\KeywordTok{svyfgt}\NormalTok{(}\OperatorTok{~}\NormalTok{eqincome, des_eusilc, }\DataTypeTok{g=}\DecValTok{0}\NormalTok{, }\DataTypeTok{type_thresh=} \StringTok{"relq"}\NormalTok{)}
\end{Highlighting}
\end{Shaded}

\begin{verbatim}
            fgt0     SE
eqincome 0.14444 0.0028
\end{verbatim}

that matches the estimate obtained by

\begin{Shaded}
\begin{Highlighting}[]
\KeywordTok{svyarpr}\NormalTok{(}\OperatorTok{~}\NormalTok{eqincome, }\DataTypeTok{design=}\NormalTok{des_eusilc, .}\DecValTok{5}\NormalTok{, .}\DecValTok{6}\NormalTok{)}
\end{Highlighting}
\end{Shaded}

\begin{verbatim}
            arpr     SE
eqincome 0.14444 0.0028
\end{verbatim}

To estimate the poverty gap(FGT1) with the poverty threshold equal to
\(0.6*MEAN\) we use:

\begin{Shaded}
\begin{Highlighting}[]
\KeywordTok{svyfgt}\NormalTok{(}\OperatorTok{~}\NormalTok{eqincome, des_eusilc, }\DataTypeTok{g=}\DecValTok{1}\NormalTok{, }\DataTypeTok{type_thresh=} \StringTok{"relm"}\NormalTok{)}
\end{Highlighting}
\end{Shaded}

\begin{verbatim}
             fgt1     SE
eqincome 0.051187 0.0011
\end{verbatim}

\begin{center}\rule{0.5\linewidth}{\linethickness}\end{center}

\textbf{A replication example}

In July 2006, \citep{jenkins2006} presented at the North American Stata
Users' Group Meetings on the stata Atkinson Index command. The example
below reproduces those statistics.

In order to match the results in \citep{jenkins2006} using the
\texttt{svyfgt} function from the convey library, the poverty threshold
was considered absolute despite being directly estimated from the survey
sample. This effectively treats the variance of the estimated poverty
threshold as zero; \texttt{svyfgt} does not account for the uncertainty
of the poverty threshold when the level has been stated as absolute with
the \texttt{abs\_thresh=} parameter. In general, we would instead
recommend using either \texttt{relq} or \texttt{relm} in the
\texttt{type\_thresh=} parameter in order to account for the added
uncertainty of the poverty threshold calculation. This example serves
only to show that \texttt{svyfgt} behaves properly as compared to other
software.

Load and prepare the same data set:

\begin{Shaded}
\begin{Highlighting}[]
\CommentTok{# load the convey package}
\KeywordTok{library}\NormalTok{(convey)}

\CommentTok{# load the survey library}
\KeywordTok{library}\NormalTok{(survey)}

\CommentTok{# load the foreign library}
\KeywordTok{library}\NormalTok{(foreign)}

\CommentTok{# create a temporary file on the local disk}
\NormalTok{tf <-}\StringTok{ }\KeywordTok{tempfile}\NormalTok{()}

\CommentTok{# store the location of the presentation file}
\NormalTok{presentation_zip <-}\StringTok{ "http://repec.org/nasug2006/nasug2006_jenkins.zip"}

\CommentTok{# download jenkins' presentation to the temporary file}
\KeywordTok{download.file}\NormalTok{( presentation_zip , tf , }\DataTypeTok{mode =} \StringTok{'wb'}\NormalTok{ )}

\CommentTok{# unzip the contents of the archive}
\NormalTok{presentation_files <-}\StringTok{ }\KeywordTok{unzip}\NormalTok{( tf , }\DataTypeTok{exdir =} \KeywordTok{tempdir}\NormalTok{() )}

\CommentTok{# load the institute for fiscal studies' 1981, 1985, and 1991 data.frame objects}
\NormalTok{x81 <-}\StringTok{ }\KeywordTok{read.dta}\NormalTok{( }\KeywordTok{grep}\NormalTok{( }\StringTok{"ifs81"}\NormalTok{ , presentation_files , }\DataTypeTok{value =} \OtherTok{TRUE}\NormalTok{ ) )}
\NormalTok{x85 <-}\StringTok{ }\KeywordTok{read.dta}\NormalTok{( }\KeywordTok{grep}\NormalTok{( }\StringTok{"ifs85"}\NormalTok{ , presentation_files , }\DataTypeTok{value =} \OtherTok{TRUE}\NormalTok{ ) )}
\NormalTok{x91 <-}\StringTok{ }\KeywordTok{read.dta}\NormalTok{( }\KeywordTok{grep}\NormalTok{( }\StringTok{"ifs91"}\NormalTok{ , presentation_files , }\DataTypeTok{value =} \OtherTok{TRUE}\NormalTok{ ) )}

\CommentTok{# }\AlertTok{NOTE}\CommentTok{: we recommend using ?convey::svyarpt rather than this unweighted calculation #}

\CommentTok{# calculate 60% of the unweighted median income in 1981}
\NormalTok{unwtd_arpt81 <-}\StringTok{ }\KeywordTok{quantile}\NormalTok{( x81}\OperatorTok{$}\NormalTok{eybhc0 , }\FloatTok{0.5}\NormalTok{ ) }\OperatorTok{*}\StringTok{ }\NormalTok{.}\DecValTok{6}

\CommentTok{# calculate 60% of the unweighted median income in 1985}
\NormalTok{unwtd_arpt85 <-}\StringTok{ }\KeywordTok{quantile}\NormalTok{( x85}\OperatorTok{$}\NormalTok{eybhc0 , }\FloatTok{0.5}\NormalTok{ ) }\OperatorTok{*}\StringTok{ }\NormalTok{.}\DecValTok{6}

\CommentTok{# calculate 60% of the unweighted median income in 1991}
\NormalTok{unwtd_arpt91 <-}\StringTok{ }\KeywordTok{quantile}\NormalTok{( x91}\OperatorTok{$}\NormalTok{eybhc0 , }\FloatTok{0.5}\NormalTok{ ) }\OperatorTok{*}\StringTok{ }\NormalTok{.}\DecValTok{6}

\CommentTok{# stack each of these three years of data into a single data.frame}
\NormalTok{x <-}\StringTok{ }\KeywordTok{rbind}\NormalTok{( x81 , x85 , x91 )}
\end{Highlighting}
\end{Shaded}

Replicate the author's survey design statement from stata code..

\begin{verbatim}
. ge poor = (year==1981)*(x < $z_81) + (year==1985)*(x < $z_85) +  (year==1991)*(x < $z_91)
. * account for clustering within HHs 
. svyset hrn [pweight = wgt]
\end{verbatim}

.. into R code:

\begin{Shaded}
\begin{Highlighting}[]
\CommentTok{# initiate a linearized survey design object}
\NormalTok{y <-}\StringTok{ }\KeywordTok{svydesign}\NormalTok{( }\OperatorTok{~}\StringTok{ }\NormalTok{hrn , }\DataTypeTok{data =}\NormalTok{ x , }\DataTypeTok{weights =} \OperatorTok{~}\StringTok{ }\NormalTok{wgt )}

\CommentTok{# immediately run the `convey_prep` function on the survey design}
\NormalTok{z <-}\StringTok{ }\KeywordTok{convey_prep}\NormalTok{( y )}
\end{Highlighting}
\end{Shaded}

Replicate the author's headcount ratio results with stata..

\begin{verbatim}
. svy: mean poor if year == 1981
(running mean on estimation sample)

Survey: Mean estimation

Number of strata =       1          Number of obs    =    9772
Number of PSUs   =    7476          Population size  = 5.5e+07
                                    Design df        =    7475

--------------------------------------------------------------
             |             Linearized
             |       Mean   Std. Err.     [95% Conf. Interval]
-------------+------------------------------------------------
        poor |   .1410125   .0044859       .132219     .149806
--------------------------------------------------------------

. svy: mean poor if year == 1985
(running mean on estimation sample)

Survey: Mean estimation

Number of strata =       1          Number of obs    =    8991
Number of PSUs   =    6972          Population size  = 5.5e+07
                                    Design df        =    6971

--------------------------------------------------------------
             |             Linearized
             |       Mean   Std. Err.     [95% Conf. Interval]
-------------+------------------------------------------------
        poor |    .137645   .0046531      .1285235    .1467665
--------------------------------------------------------------

. svy: mean poor if year == 1991
(running mean on estimation sample)

Survey: Mean estimation

Number of strata =       1          Number of obs    =    6468
Number of PSUs   =    5254          Population size  = 5.6e+07
                                    Design df        =    5253

--------------------------------------------------------------
             |             Linearized
             |       Mean   Std. Err.     [95% Conf. Interval]
-------------+------------------------------------------------
        poor |   .2021312   .0062077      .1899615    .2143009
--------------------------------------------------------------
\end{verbatim}

..using R code:

\begin{Shaded}
\begin{Highlighting}[]
\NormalTok{headcount_}\DecValTok{81}\NormalTok{ <-}\StringTok{ }
\StringTok{    }\KeywordTok{svyfgt}\NormalTok{( }
        \OperatorTok{~}\StringTok{ }\NormalTok{eybhc0 , }
        \KeywordTok{subset}\NormalTok{( z , year }\OperatorTok{==}\StringTok{ }\DecValTok{1981}\NormalTok{ ) , }
        \DataTypeTok{g =} \DecValTok{0}\NormalTok{ , }
        \DataTypeTok{abs_thresh =}\NormalTok{ unwtd_arpt81}
\NormalTok{    )}

\NormalTok{headcount_}\DecValTok{81}
\end{Highlighting}
\end{Shaded}

\begin{verbatim}
##           fgt0     SE
## eybhc0 0.14101 0.0045
\end{verbatim}

\begin{Shaded}
\begin{Highlighting}[]
\KeywordTok{confint}\NormalTok{( headcount_}\DecValTok{81}\NormalTok{ , }\DataTypeTok{df =} \KeywordTok{degf}\NormalTok{( }\KeywordTok{subset}\NormalTok{( z , year }\OperatorTok{==}\StringTok{ }\DecValTok{1981}\NormalTok{ ) ) )}
\end{Highlighting}
\end{Shaded}

\begin{verbatim}
##            2.5 %    97.5 %
## eybhc0 0.1322193 0.1498057
\end{verbatim}

\begin{Shaded}
\begin{Highlighting}[]
\NormalTok{headcount_}\DecValTok{85}\NormalTok{ <-}\StringTok{ }
\StringTok{    }\KeywordTok{svyfgt}\NormalTok{( }
        \OperatorTok{~}\StringTok{ }\NormalTok{eybhc0 , }
        \KeywordTok{subset}\NormalTok{( z , year }\OperatorTok{==}\StringTok{ }\DecValTok{1985}\NormalTok{ ) , }
        \DataTypeTok{g =} \DecValTok{0}\NormalTok{ , }
        \DataTypeTok{abs_thresh =}\NormalTok{ unwtd_arpt85 }
\NormalTok{    )}
    
\NormalTok{headcount_}\DecValTok{85}
\end{Highlighting}
\end{Shaded}

\begin{verbatim}
##           fgt0     SE
## eybhc0 0.13764 0.0047
\end{verbatim}

\begin{Shaded}
\begin{Highlighting}[]
\KeywordTok{confint}\NormalTok{( headcount_}\DecValTok{85}\NormalTok{ , }\DataTypeTok{df =} \KeywordTok{degf}\NormalTok{( }\KeywordTok{subset}\NormalTok{( z , year }\OperatorTok{==}\StringTok{ }\DecValTok{1985}\NormalTok{ ) ) )}
\end{Highlighting}
\end{Shaded}

\begin{verbatim}
##            2.5 %    97.5 %
## eybhc0 0.1285239 0.1467661
\end{verbatim}

\begin{Shaded}
\begin{Highlighting}[]
\NormalTok{headcount_}\DecValTok{91}\NormalTok{ <-}\StringTok{ }
\StringTok{    }\KeywordTok{svyfgt}\NormalTok{( }
        \OperatorTok{~}\StringTok{ }\NormalTok{eybhc0 , }
        \KeywordTok{subset}\NormalTok{( z , year }\OperatorTok{==}\StringTok{ }\DecValTok{1991}\NormalTok{ ) , }
        \DataTypeTok{g =} \DecValTok{0}\NormalTok{ , }
        \DataTypeTok{abs_thresh =}\NormalTok{ unwtd_arpt91 }
\NormalTok{    )}

\NormalTok{headcount_}\DecValTok{91}
\end{Highlighting}
\end{Shaded}

\begin{verbatim}
##           fgt0     SE
## eybhc0 0.20213 0.0062
\end{verbatim}

\begin{Shaded}
\begin{Highlighting}[]
\KeywordTok{confint}\NormalTok{( headcount_}\DecValTok{91}\NormalTok{ , }\DataTypeTok{df =} \KeywordTok{degf}\NormalTok{( }\KeywordTok{subset}\NormalTok{( z , year }\OperatorTok{==}\StringTok{ }\DecValTok{1991}\NormalTok{ ) ) )}
\end{Highlighting}
\end{Shaded}

\begin{verbatim}
##            2.5 % 97.5 %
## eybhc0 0.1899624 0.2143
\end{verbatim}

Confirm this replication applies for the normalized poverty gap as well,
comparing stata code..

\begin{verbatim}
. ge ngap = poor*($z_81- x)/$z_81 if year == 1981

. svy: mean ngap if year == 1981
(running mean on estimation sample)

Survey: Mean estimation

Number of strata =       1          Number of obs    =    9772
Number of PSUs   =    7476          Population size  = 5.5e+07
                                    Design df        =    7475

--------------------------------------------------------------
             |             Linearized
             |       Mean   Std. Err.     [95% Conf. Interval]
-------------+------------------------------------------------
        ngap |   .0271577   .0013502      .0245109    .0298044
--------------------------------------------------------------
\end{verbatim}

..to R code:

\begin{Shaded}
\begin{Highlighting}[]
\NormalTok{norm_pov_}\DecValTok{81}\NormalTok{ <-}\StringTok{ }
\StringTok{    }\KeywordTok{svyfgt}\NormalTok{( }
        \OperatorTok{~}\StringTok{ }\NormalTok{eybhc0 , }
        \KeywordTok{subset}\NormalTok{( z , year }\OperatorTok{==}\StringTok{ }\DecValTok{1981}\NormalTok{ ) , }
        \DataTypeTok{g =} \DecValTok{1}\NormalTok{ , }
        \DataTypeTok{abs_thresh =}\NormalTok{ unwtd_arpt81}
\NormalTok{    )}
    
\NormalTok{norm_pov_}\DecValTok{81}
\end{Highlighting}
\end{Shaded}

\begin{verbatim}
##            fgt1     SE
## eybhc0 0.027158 0.0014
\end{verbatim}

\begin{Shaded}
\begin{Highlighting}[]
\KeywordTok{confint}\NormalTok{( norm_pov_}\DecValTok{81}\NormalTok{ , }\DataTypeTok{df =} \KeywordTok{degf}\NormalTok{( }\KeywordTok{subset}\NormalTok{( z , year }\OperatorTok{==}\StringTok{ }\DecValTok{1981}\NormalTok{ ) ) )}
\end{Highlighting}
\end{Shaded}

\begin{verbatim}
##             2.5 %     97.5 %
## eybhc0 0.02451106 0.02980428
\end{verbatim}

For additional usage examples of \texttt{svyfgt}, type
\texttt{?convey::svyfgt} in the R console.

\chapter{Inequality Measurement}\label{inequality}

Another problem faced by societies is inequality. Economic inequality
can have several different meanings: income, education, resources,
opportunities, wellbeing, etc. Usually, studies on economic inequality
focus on income distribution.

Most inequality data comes from censuses and household surveys.
Therefore, in order to produce reliable estimates from this samples,
appropriate procedures are necessary.

This chapter presents brief presentations on inequality measures, also
providing replication examples if possible. It starts with an initial
attempt to measure the inequality between two groups of a population;
then, it presents ideas of overall inequality indices, moving from the
quintile share ratio to the Lorenz curve and measures derived from it;
finally, it discusses the concept of entropy and presents inequality
measures based on it.

\section{The Gender Pay Gap (svygpg)}\label{the-gender-pay-gap-svygpg}

Although the \(GPG\) is not an inequality measure in the usual sense, it
can still be an useful instrument to evaluate the discrimination among
men and women. Put simply, it expresses the relative difference between
the average hourly earnings of men and women, presenting it as a
percentage of the average of hourly earnings of men.

In mathematical terms, this index can be described as,

\[ \widehat{GPG} = \frac{ \widehat{\bar{Y}}_{male} - \widehat{\bar{Y}}_{female} }{ \widehat{\bar{Y}}_{male} } \],

which is precisely the estimator used in the package. As we can see from
the formula, if there is no difference among classes, \(GPG = 0\). Else,
if \(GPG > 0\), it means that the average hourly income received by
women are \(GPG\) percent smaller than men's. For negative \(GPG\), it
means that women's hourly earnings are \(GPG\) percent larger than
men's. In other words, the larger the \(GPG\), larger is the shortfall
of women's hourly earnings.

We can also develop a more straightforward idea: for every \$1 raise in
men's hourly earnings, women's hourly earnings are expected to increase
\$\((1-GPG)\). For instance, assuming \(GPG = 0.8\), for every \$1.00
increase in men's average hourly earnings, women's hourly earnings would
increase only \$0.20.

The details of the linearization of the \texttt{GPG} are discussed by
\citep{deville1999} and \citep{osier2009}.

\begin{center}\rule{0.5\linewidth}{\linethickness}\end{center}

\textbf{A replication example}

The R \texttt{vardpoor} package \citep{vardpoor}, created by researchers
at the Central Statistical Bureau of Latvia, includes a gpg coefficient
calculation using the ultimate cluster method. The example below
reproduces those statistics.

Load and prepare the same data set:

\begin{Shaded}
\begin{Highlighting}[]
\CommentTok{# load the convey package}
\KeywordTok{library}\NormalTok{(convey)}

\CommentTok{# load the survey library}
\KeywordTok{library}\NormalTok{(survey)}

\CommentTok{# load the vardpoor library}
\KeywordTok{library}\NormalTok{(vardpoor)}

\CommentTok{# load the synthetic european union statistics on income & living conditions}
\KeywordTok{data}\NormalTok{(eusilc)}

\CommentTok{# make all column names lowercase}
\KeywordTok{names}\NormalTok{( eusilc ) <-}\StringTok{ }\KeywordTok{tolower}\NormalTok{( }\KeywordTok{names}\NormalTok{( eusilc ) )}

\CommentTok{# coerce the gender variable to numeric 1 or 2}
\NormalTok{eusilc}\OperatorTok{$}\NormalTok{one_two <-}\StringTok{ }\KeywordTok{as.numeric}\NormalTok{( eusilc}\OperatorTok{$}\NormalTok{rb090 }\OperatorTok{==}\StringTok{ "female"}\NormalTok{ ) }\OperatorTok{+}\StringTok{ }\DecValTok{1}

\CommentTok{# add a column with the row number}
\NormalTok{dati <-}\StringTok{ }\KeywordTok{data.table}\NormalTok{(}\DataTypeTok{IDd =} \DecValTok{1} \OperatorTok{:}\StringTok{ }\KeywordTok{nrow}\NormalTok{(eusilc), eusilc)}

\CommentTok{# calculate the gpg coefficient}
\CommentTok{# using the R vardpoor library}
\NormalTok{varpoord_gpg_calculation <-}
\StringTok{    }\KeywordTok{varpoord}\NormalTok{(}
    
        \CommentTok{# analysis variable}
        \DataTypeTok{Y =} \StringTok{"eqincome"}\NormalTok{, }
        
        \CommentTok{# weights variable}
        \DataTypeTok{w_final =} \StringTok{"rb050"}\NormalTok{,}
        
        \CommentTok{# row number variable}
        \DataTypeTok{ID_level1 =} \StringTok{"IDd"}\NormalTok{,}
        
        \CommentTok{# row number variable}
        \DataTypeTok{ID_level2 =} \StringTok{"IDd"}\NormalTok{,}
        
        \CommentTok{# strata variable}
        \DataTypeTok{H =} \StringTok{"db040"}\NormalTok{, }
        
        \DataTypeTok{N_h =} \OtherTok{NULL}\NormalTok{ ,}
        
        \CommentTok{# clustering variable}
        \DataTypeTok{PSU =} \StringTok{"rb030"}\NormalTok{, }
        
        \CommentTok{# data.table}
        \DataTypeTok{dataset =}\NormalTok{ dati, }
        
        \CommentTok{# gpg coefficient function}
        \DataTypeTok{type =} \StringTok{"lingpg"}\NormalTok{ ,}
        
        \CommentTok{# gender variable}
        \DataTypeTok{gender =} \StringTok{"one_two"}\NormalTok{,}
      
      \CommentTok{# poverty threshold range}
      \DataTypeTok{order_quant =}\NormalTok{ 50L ,}
      
      \CommentTok{# get linearized variable}
      \DataTypeTok{outp_lin =} \OtherTok{TRUE}
\NormalTok{    )}
\end{Highlighting}
\end{Shaded}

\begin{verbatim}
## NULL
\end{verbatim}

\begin{Shaded}
\begin{Highlighting}[]
\CommentTok{# construct a survey.design}
\CommentTok{# using our recommended setup}
\NormalTok{des_eusilc <-}\StringTok{ }
\StringTok{    }\KeywordTok{svydesign}\NormalTok{( }
        \DataTypeTok{ids =} \OperatorTok{~}\StringTok{ }\NormalTok{rb030 , }
        \DataTypeTok{strata =} \OperatorTok{~}\StringTok{ }\NormalTok{db040 ,  }
        \DataTypeTok{weights =} \OperatorTok{~}\StringTok{ }\NormalTok{rb050 , }
        \DataTypeTok{data =}\NormalTok{ eusilc}
\NormalTok{    )}

\CommentTok{# immediately run the convey_prep function on it}
\NormalTok{des_eusilc <-}\StringTok{ }\KeywordTok{convey_prep}\NormalTok{( des_eusilc )}

\CommentTok{# coefficients do match}
\NormalTok{varpoord_gpg_calculation}\OperatorTok{$}\NormalTok{all_result}\OperatorTok{$}\NormalTok{value}
\end{Highlighting}
\end{Shaded}

\begin{verbatim}
## [1] 7.645389
\end{verbatim}

\begin{Shaded}
\begin{Highlighting}[]
\KeywordTok{coef}\NormalTok{( }\KeywordTok{svygpg}\NormalTok{( }\OperatorTok{~}\StringTok{ }\NormalTok{eqincome , des_eusilc , }\DataTypeTok{sex =} \OperatorTok{~}\StringTok{ }\NormalTok{rb090 ) ) }\OperatorTok{*}\StringTok{ }\DecValTok{100}
\end{Highlighting}
\end{Shaded}

\begin{verbatim}
## eqincome 
## 7.645389
\end{verbatim}

\begin{Shaded}
\begin{Highlighting}[]
\CommentTok{# linearized variables do match}
\CommentTok{# vardpoor}
\NormalTok{lin_gpg_varpoord<-}\StringTok{ }\NormalTok{varpoord_gpg_calculation}\OperatorTok{$}\NormalTok{lin_out}\OperatorTok{$}\NormalTok{lin_gpg}
\CommentTok{# convey }
\NormalTok{lin_gpg_convey <-}\StringTok{ }\KeywordTok{attr}\NormalTok{(}\KeywordTok{svygpg}\NormalTok{( }\OperatorTok{~}\StringTok{ }\NormalTok{eqincome , des_eusilc, }\DataTypeTok{sex =} \OperatorTok{~}\StringTok{ }\NormalTok{rb090 ),}\StringTok{"lin"}\NormalTok{)}

\CommentTok{# check equality}
\KeywordTok{all.equal}\NormalTok{(lin_gpg_varpoord,}\DecValTok{100}\OperatorTok{*}\NormalTok{lin_gpg_convey[,}\DecValTok{1}\NormalTok{] )}
\end{Highlighting}
\end{Shaded}

\begin{verbatim}
## [1] TRUE
\end{verbatim}

\begin{Shaded}
\begin{Highlighting}[]
\CommentTok{# variances do not match exactly}
\KeywordTok{attr}\NormalTok{( }\KeywordTok{svygpg}\NormalTok{( }\OperatorTok{~}\StringTok{ }\NormalTok{eqincome , des_eusilc , }\DataTypeTok{sex =} \OperatorTok{~}\StringTok{ }\NormalTok{rb090 ) , }\StringTok{'var'}\NormalTok{ ) }\OperatorTok{*}\StringTok{ }\DecValTok{10000}
\end{Highlighting}
\end{Shaded}

\begin{verbatim}
##           eqincome
## eqincome 0.6493911
\end{verbatim}

\begin{Shaded}
\begin{Highlighting}[]
\NormalTok{varpoord_gpg_calculation}\OperatorTok{$}\NormalTok{all_result}\OperatorTok{$}\NormalTok{var}
\end{Highlighting}
\end{Shaded}

\begin{verbatim}
## [1] 0.6482346
\end{verbatim}

\begin{Shaded}
\begin{Highlighting}[]
\CommentTok{# standard errors do not match exactly}
\NormalTok{varpoord_gpg_calculation}\OperatorTok{$}\NormalTok{all_result}\OperatorTok{$}\NormalTok{se}
\end{Highlighting}
\end{Shaded}

\begin{verbatim}
## [1] 0.8051301
\end{verbatim}

\begin{Shaded}
\begin{Highlighting}[]
\KeywordTok{SE}\NormalTok{( }\KeywordTok{svygpg}\NormalTok{( }\OperatorTok{~}\StringTok{ }\NormalTok{eqincome , des_eusilc , }\DataTypeTok{sex =} \OperatorTok{~}\StringTok{ }\NormalTok{rb090 ) ) }\OperatorTok{*}\StringTok{ }\DecValTok{100}
\end{Highlighting}
\end{Shaded}

\begin{verbatim}
##           eqincome
## eqincome 0.8058481
\end{verbatim}

The variance estimate is computed by using the approximation defined in
\eqref{eq:var}, where the linearized variable \(z\) is defined by
\eqref{eq:lin}. The functions \texttt{convey::svygpg} and
\texttt{vardpoor::lingpg} produce the same linearized variable \(z\).

However, the measures of uncertainty do not line up, because
\texttt{library(vardpoor)} defaults to an ultimate cluster method that
can be replicated with an alternative setup of the
\texttt{survey.design} object.

\begin{Shaded}
\begin{Highlighting}[]
\CommentTok{# within each strata, sum up the weights}
\NormalTok{cluster_sums <-}\StringTok{ }\KeywordTok{aggregate}\NormalTok{( eusilc}\OperatorTok{$}\NormalTok{rb050 , }\KeywordTok{list}\NormalTok{( eusilc}\OperatorTok{$}\NormalTok{db040 ) , sum )}

\CommentTok{# name the within-strata sums of weights the `cluster_sum`}
\KeywordTok{names}\NormalTok{( cluster_sums ) <-}\StringTok{ }\KeywordTok{c}\NormalTok{( }\StringTok{"db040"}\NormalTok{ , }\StringTok{"cluster_sum"}\NormalTok{ )}

\CommentTok{# merge this column back onto the data.frame}
\NormalTok{eusilc <-}\StringTok{ }\KeywordTok{merge}\NormalTok{( eusilc , cluster_sums )}

\CommentTok{# construct a survey.design}
\CommentTok{# with the fpc using the cluster sum}
\NormalTok{des_eusilc_ultimate_cluster <-}\StringTok{ }
\StringTok{    }\KeywordTok{svydesign}\NormalTok{( }
        \DataTypeTok{ids =} \OperatorTok{~}\StringTok{ }\NormalTok{rb030 , }
        \DataTypeTok{strata =} \OperatorTok{~}\StringTok{ }\NormalTok{db040 ,  }
        \DataTypeTok{weights =} \OperatorTok{~}\StringTok{ }\NormalTok{rb050 , }
        \DataTypeTok{data =}\NormalTok{ eusilc , }
        \DataTypeTok{fpc =} \OperatorTok{~}\StringTok{ }\NormalTok{cluster_sum }
\NormalTok{    )}

\CommentTok{# again, immediately run the convey_prep function on the `survey.design`}
\NormalTok{des_eusilc_ultimate_cluster <-}\StringTok{ }\KeywordTok{convey_prep}\NormalTok{( des_eusilc_ultimate_cluster )}

\CommentTok{# matches}
\KeywordTok{attr}\NormalTok{( }\KeywordTok{svygpg}\NormalTok{( }\OperatorTok{~}\StringTok{ }\NormalTok{eqincome , des_eusilc_ultimate_cluster , }\DataTypeTok{sex =} \OperatorTok{~}\StringTok{ }\NormalTok{rb090 ) , }\StringTok{'var'}\NormalTok{ ) }\OperatorTok{*}\StringTok{ }\DecValTok{10000}
\end{Highlighting}
\end{Shaded}

\begin{verbatim}
##           eqincome
## eqincome 0.6482346
\end{verbatim}

\begin{Shaded}
\begin{Highlighting}[]
\NormalTok{varpoord_gpg_calculation}\OperatorTok{$}\NormalTok{all_result}\OperatorTok{$}\NormalTok{var}
\end{Highlighting}
\end{Shaded}

\begin{verbatim}
## [1] 0.6482346
\end{verbatim}

\begin{Shaded}
\begin{Highlighting}[]
\CommentTok{# matches}
\NormalTok{varpoord_gpg_calculation}\OperatorTok{$}\NormalTok{all_result}\OperatorTok{$}\NormalTok{se}
\end{Highlighting}
\end{Shaded}

\begin{verbatim}
## [1] 0.8051301
\end{verbatim}

\begin{Shaded}
\begin{Highlighting}[]
\KeywordTok{SE}\NormalTok{( }\KeywordTok{svygpg}\NormalTok{( }\OperatorTok{~}\StringTok{ }\NormalTok{eqincome , des_eusilc_ultimate_cluster , }\DataTypeTok{sex =} \OperatorTok{~}\StringTok{ }\NormalTok{rb090 ) ) }\OperatorTok{*}\StringTok{ }\DecValTok{100}
\end{Highlighting}
\end{Shaded}

\begin{verbatim}
##           eqincome
## eqincome 0.8051301
\end{verbatim}

For additional usage examples of \texttt{svygpg}, type
\texttt{?convey::svygpg} in the R console.

\section{Quintile Share Ratio
(svyqsr)}\label{quintile-share-ratio-svyqsr}

Unlike the previous measure, the quintile share ratio is an inequality
measure in itself, depending only of the income distribution to evaluate
the degree of inequality. By definition, it can be described as the
ratio between the income share held by the richest 20\% and the poorest
20\% of the population.

In plain terms, it expresses how many times the wealthier part of the
population are richer than the poorest part. For instance, a \(QSR = 4\)
implies that the upper class owns 4 times as much of the total income as
the poor.

The quintile share ratio can be modified to a more general function of
fractile share ratios. For instance, the Palma index is defined as the
ratio between the share of the 10\% richest over the share held by the
poorest 40\% and \citep{cobham2015} presents interesting arguments for
its usage.

The details of the linearization of the \texttt{QSR} are discussed by
\citep{deville1999} and \citep{osier2009}.

\begin{center}\rule{0.5\linewidth}{\linethickness}\end{center}

\textbf{A replication example}

The R \texttt{vardpoor} package \citep{vardpoor}, created by researchers
at the Central Statistical Bureau of Latvia, includes a qsr coefficient
calculation using the ultimate cluster method. The example below
reproduces those statistics.

Load and prepare the same data set:

\begin{Shaded}
\begin{Highlighting}[]
\CommentTok{# load the convey package}
\KeywordTok{library}\NormalTok{(convey)}

\CommentTok{# load the survey library}
\KeywordTok{library}\NormalTok{(survey)}

\CommentTok{# load the vardpoor library}
\KeywordTok{library}\NormalTok{(vardpoor)}

\CommentTok{# load the synthetic european union statistics on income & living conditions}
\KeywordTok{data}\NormalTok{(eusilc)}

\CommentTok{# make all column names lowercase}
\KeywordTok{names}\NormalTok{( eusilc ) <-}\StringTok{ }\KeywordTok{tolower}\NormalTok{( }\KeywordTok{names}\NormalTok{( eusilc ) )}

\CommentTok{# add a column with the row number}
\NormalTok{dati <-}\StringTok{ }\KeywordTok{data.table}\NormalTok{(}\DataTypeTok{IDd =} \DecValTok{1} \OperatorTok{:}\StringTok{ }\KeywordTok{nrow}\NormalTok{(eusilc), eusilc)}

\CommentTok{# calculate the qsr coefficient}
\CommentTok{# using the R vardpoor library}
\NormalTok{varpoord_qsr_calculation <-}
\StringTok{    }\KeywordTok{varpoord}\NormalTok{(}
    
        \CommentTok{# analysis variable}
        \DataTypeTok{Y =} \StringTok{"eqincome"}\NormalTok{, }
        
        \CommentTok{# weights variable}
        \DataTypeTok{w_final =} \StringTok{"rb050"}\NormalTok{,}
        
        \CommentTok{# row number variable}
        \DataTypeTok{ID_level1 =} \StringTok{"IDd"}\NormalTok{,}
        
        \CommentTok{# row number variable}
        \DataTypeTok{ID_level2 =} \StringTok{"IDd"}\NormalTok{,}
        
        \CommentTok{# strata variable}
        \DataTypeTok{H =} \StringTok{"db040"}\NormalTok{, }
        
        \DataTypeTok{N_h =} \OtherTok{NULL}\NormalTok{ ,}
        
        \CommentTok{# clustering variable}
        \DataTypeTok{PSU =} \StringTok{"rb030"}\NormalTok{, }
        
        \CommentTok{# data.table}
        \DataTypeTok{dataset =}\NormalTok{ dati, }
        
        \CommentTok{# qsr coefficient function}
        \DataTypeTok{type =} \StringTok{"linqsr"}\NormalTok{,}
      
      \CommentTok{# poverty threshold range}
      \DataTypeTok{order_quant =}\NormalTok{ 50L ,}
      
      \CommentTok{# get linearized variable}
      \DataTypeTok{outp_lin =} \OtherTok{TRUE}
        
\NormalTok{    )}
\end{Highlighting}
\end{Shaded}

\begin{verbatim}
## NULL
\end{verbatim}

\begin{Shaded}
\begin{Highlighting}[]
\CommentTok{# construct a survey.design}
\CommentTok{# using our recommended setup}
\NormalTok{des_eusilc <-}\StringTok{ }
\StringTok{    }\KeywordTok{svydesign}\NormalTok{( }
        \DataTypeTok{ids =} \OperatorTok{~}\StringTok{ }\NormalTok{rb030 , }
        \DataTypeTok{strata =} \OperatorTok{~}\StringTok{ }\NormalTok{db040 ,  }
        \DataTypeTok{weights =} \OperatorTok{~}\StringTok{ }\NormalTok{rb050 , }
        \DataTypeTok{data =}\NormalTok{ eusilc}
\NormalTok{    )}

\CommentTok{# immediately run the convey_prep function on it}
\NormalTok{des_eusilc <-}\StringTok{ }\KeywordTok{convey_prep}\NormalTok{( des_eusilc )}

\CommentTok{# coefficients do match}
\NormalTok{varpoord_qsr_calculation}\OperatorTok{$}\NormalTok{all_result}\OperatorTok{$}\NormalTok{value}
\end{Highlighting}
\end{Shaded}

\begin{verbatim}
## [1] 3.970004
\end{verbatim}

\begin{Shaded}
\begin{Highlighting}[]
\KeywordTok{coef}\NormalTok{( }\KeywordTok{svyqsr}\NormalTok{( }\OperatorTok{~}\StringTok{ }\NormalTok{eqincome , des_eusilc ) )}
\end{Highlighting}
\end{Shaded}

\begin{verbatim}
## eqincome 
## 3.970004
\end{verbatim}

\begin{Shaded}
\begin{Highlighting}[]
\CommentTok{# linearized variables do match}
\CommentTok{# vardpoor}
\NormalTok{lin_qsr_varpoord<-}\StringTok{ }\NormalTok{varpoord_qsr_calculation}\OperatorTok{$}\NormalTok{lin_out}\OperatorTok{$}\NormalTok{lin_qsr}
\CommentTok{# convey }
\NormalTok{lin_qsr_convey <-}\StringTok{ }\KeywordTok{attr}\NormalTok{(}\KeywordTok{svyqsr}\NormalTok{( }\OperatorTok{~}\StringTok{ }\NormalTok{eqincome , des_eusilc ),}\StringTok{"lin"}\NormalTok{)}

\CommentTok{# check equality}
\KeywordTok{all.equal}\NormalTok{(lin_qsr_varpoord, lin_qsr_convey )}
\end{Highlighting}
\end{Shaded}

\begin{verbatim}
## [1] TRUE
\end{verbatim}

\begin{Shaded}
\begin{Highlighting}[]
\CommentTok{# variances do not match exactly}
\KeywordTok{attr}\NormalTok{( }\KeywordTok{svyqsr}\NormalTok{( }\OperatorTok{~}\StringTok{ }\NormalTok{eqincome , des_eusilc ) , }\StringTok{'var'}\NormalTok{ )}
\end{Highlighting}
\end{Shaded}

\begin{verbatim}
##             eqincome
## eqincome 0.001810537
\end{verbatim}

\begin{Shaded}
\begin{Highlighting}[]
\NormalTok{varpoord_qsr_calculation}\OperatorTok{$}\NormalTok{all_result}\OperatorTok{$}\NormalTok{var}
\end{Highlighting}
\end{Shaded}

\begin{verbatim}
## [1] 0.001807323
\end{verbatim}

\begin{Shaded}
\begin{Highlighting}[]
\CommentTok{# standard errors do not match exactly}
\NormalTok{varpoord_qsr_calculation}\OperatorTok{$}\NormalTok{all_result}\OperatorTok{$}\NormalTok{se}
\end{Highlighting}
\end{Shaded}

\begin{verbatim}
## [1] 0.04251263
\end{verbatim}

\begin{Shaded}
\begin{Highlighting}[]
\KeywordTok{SE}\NormalTok{( }\KeywordTok{svyqsr}\NormalTok{( }\OperatorTok{~}\StringTok{ }\NormalTok{eqincome , des_eusilc ) )}
\end{Highlighting}
\end{Shaded}

\begin{verbatim}
##            eqincome
## eqincome 0.04255041
\end{verbatim}

The variance estimate is computed by using the approximation defined in
\eqref{eq:var}, where the linearized variable \(z\) is defined by
\eqref{eq:lin}. The functions \texttt{convey::svygpg} and
\texttt{vardpoor::lingpg} produce the same linearized variable \(z\).

However, the measures of uncertainty do not line up, because
\texttt{library(vardpoor)} defaults to an ultimate cluster method that
can be replicated with an alternative setup of the
\texttt{survey.design} object.

\begin{Shaded}
\begin{Highlighting}[]
\CommentTok{# within each strata, sum up the weights}
\NormalTok{cluster_sums <-}\StringTok{ }\KeywordTok{aggregate}\NormalTok{( eusilc}\OperatorTok{$}\NormalTok{rb050 , }\KeywordTok{list}\NormalTok{( eusilc}\OperatorTok{$}\NormalTok{db040 ) , sum )}

\CommentTok{# name the within-strata sums of weights the `cluster_sum`}
\KeywordTok{names}\NormalTok{( cluster_sums ) <-}\StringTok{ }\KeywordTok{c}\NormalTok{( }\StringTok{"db040"}\NormalTok{ , }\StringTok{"cluster_sum"}\NormalTok{ )}

\CommentTok{# merge this column back onto the data.frame}
\NormalTok{eusilc <-}\StringTok{ }\KeywordTok{merge}\NormalTok{( eusilc , cluster_sums )}

\CommentTok{# construct a survey.design}
\CommentTok{# with the fpc using the cluster sum}
\NormalTok{des_eusilc_ultimate_cluster <-}\StringTok{ }
\StringTok{    }\KeywordTok{svydesign}\NormalTok{( }
        \DataTypeTok{ids =} \OperatorTok{~}\StringTok{ }\NormalTok{rb030 , }
        \DataTypeTok{strata =} \OperatorTok{~}\StringTok{ }\NormalTok{db040 ,  }
        \DataTypeTok{weights =} \OperatorTok{~}\StringTok{ }\NormalTok{rb050 , }
        \DataTypeTok{data =}\NormalTok{ eusilc , }
        \DataTypeTok{fpc =} \OperatorTok{~}\StringTok{ }\NormalTok{cluster_sum }
\NormalTok{    )}

\CommentTok{# again, immediately run the convey_prep function on the `survey.design`}
\NormalTok{des_eusilc_ultimate_cluster <-}\StringTok{ }\KeywordTok{convey_prep}\NormalTok{( des_eusilc_ultimate_cluster )}

\CommentTok{# matches}
\KeywordTok{attr}\NormalTok{( }\KeywordTok{svyqsr}\NormalTok{( }\OperatorTok{~}\StringTok{ }\NormalTok{eqincome , des_eusilc_ultimate_cluster ) , }\StringTok{'var'}\NormalTok{ )}
\end{Highlighting}
\end{Shaded}

\begin{verbatim}
##             eqincome
## eqincome 0.001807323
\end{verbatim}

\begin{Shaded}
\begin{Highlighting}[]
\NormalTok{varpoord_qsr_calculation}\OperatorTok{$}\NormalTok{all_result}\OperatorTok{$}\NormalTok{var}
\end{Highlighting}
\end{Shaded}

\begin{verbatim}
## [1] 0.001807323
\end{verbatim}

\begin{Shaded}
\begin{Highlighting}[]
\CommentTok{# matches}
\NormalTok{varpoord_qsr_calculation}\OperatorTok{$}\NormalTok{all_result}\OperatorTok{$}\NormalTok{se}
\end{Highlighting}
\end{Shaded}

\begin{verbatim}
## [1] 0.04251263
\end{verbatim}

\begin{Shaded}
\begin{Highlighting}[]
\KeywordTok{SE}\NormalTok{( }\KeywordTok{svyqsr}\NormalTok{( }\OperatorTok{~}\StringTok{ }\NormalTok{eqincome , des_eusilc_ultimate_cluster ) )}
\end{Highlighting}
\end{Shaded}

\begin{verbatim}
##            eqincome
## eqincome 0.04251263
\end{verbatim}

For additional usage examples of \texttt{svyqsr}, type
\texttt{?convey::svyqsr} in the R console.

\section{Lorenz Curve (svylorenz)}\label{lorenz-curve-svylorenz}

Though not an inequality measure in itself, the Lorenz curve is a
classic instrument of distribution analysis. Basically, it is a function
that associates a cumulative share of the population to the share of the
total income it owns. In mathematical terms,

\[
L(p) = \frac{\int_{-\infty}^{Q_p}yf(y)dy}{\int_{-\infty}^{+\infty}yf(y)dy}
\]

where \(Q_p\) is the quantile \(p\) of the population.

The two extreme distributive cases are

\begin{itemize}
\tightlist
\item
  Perfect equality:

  \begin{itemize}
  \tightlist
  \item
    Every individual has the same income;
  \item
    Every share of the population has the same share of the income;
  \item
    Therefore, the reference curve is
    \[L(p) = p \text{ } \forall p \in [0,1] \text{.}\]
  \end{itemize}
\item
  Perfect inequality:

  \begin{itemize}
  \tightlist
  \item
    One individual concentrates all of society's income, while the other
    individuals have zero income;
  \item
    Therefore, the reference curve is
  \end{itemize}
\end{itemize}

\[
L(p)=
\begin{cases}
0, &\forall p < 1 \\
1, &\text{if } p = 1 \text{.}
\end{cases}
\]

In order to evaluate the degree of inequality in a society, the analyst
looks at the distance between the real curve and those two reference
curves.

The estimator of this function was derived by \citep{kovacevic1997}:

\[
L(p) = \frac{ \sum_{i \in S} w_i \cdot y_i \cdot \delta \{ y_i \le \widehat{Q}_p \}}{\widehat{Y}}, \text{ } 0 \le p \le 1.
\]

Yet, this formula is used to calculate specific points of the curve and
their respective SEs. The formula to plot an approximation of the
continuous empirical curve comes from \citep{lerman1989}.

\begin{center}\rule{0.5\linewidth}{\linethickness}\end{center}

\textbf{A replication example}

In October 2016, \citep{jann2016} released a pre-publication working
paper to estimate lorenz and concentration curves using stata. The
example below reproduces the statistics presented in his section 4.1.

\begin{Shaded}
\begin{Highlighting}[]
\CommentTok{# load the convey package}
\KeywordTok{library}\NormalTok{(convey)}

\CommentTok{# load the survey library}
\KeywordTok{library}\NormalTok{(survey)}

\CommentTok{# load the stata-style webuse library}
\KeywordTok{library}\NormalTok{(webuse)}

\CommentTok{# load the NLSW 1988 data}
\KeywordTok{webuse}\NormalTok{(}\StringTok{"nlsw88"}\NormalTok{)}

\CommentTok{# coerce that `tbl_df` to a standard R `data.frame`}
\NormalTok{nlsw88 <-}\StringTok{ }\KeywordTok{data.frame}\NormalTok{( nlsw88 )}

\CommentTok{# initiate a linearized survey design object}
\NormalTok{des_nlsw88 <-}\StringTok{ }\KeywordTok{svydesign}\NormalTok{( }\DataTypeTok{ids =} \OperatorTok{~}\DecValTok{1}\NormalTok{ , }\DataTypeTok{data =}\NormalTok{ nlsw88 )}
\end{Highlighting}
\end{Shaded}

\begin{verbatim}
## Warning in svydesign.default(ids = ~1, data = nlsw88): No weights or
## probabilities supplied, assuming equal probability
\end{verbatim}

\begin{Shaded}
\begin{Highlighting}[]
\CommentTok{# immediately run the `convey_prep` function on the survey design}
\NormalTok{des_nlsw88 <-}\StringTok{ }\KeywordTok{convey_prep}\NormalTok{(des_nlsw88)}

\CommentTok{# estimates lorenz curve}
\NormalTok{result.lin <-}\StringTok{ }\KeywordTok{svylorenz}\NormalTok{( }\OperatorTok{~}\NormalTok{wage, des_nlsw88, }\DataTypeTok{quantiles =} \KeywordTok{seq}\NormalTok{( }\DecValTok{0}\NormalTok{, }\DecValTok{1}\NormalTok{, .}\DecValTok{05}\NormalTok{ ), }\DataTypeTok{na.rm =} \OtherTok{TRUE}\NormalTok{ )}
\end{Highlighting}
\end{Shaded}

\includegraphics{context_files/figure-latex/unnamed-chunk-36-1.pdf}

\begin{Shaded}
\begin{Highlighting}[]
\CommentTok{# note: most survey commands in R use Inf degrees of freedom by default}
\CommentTok{# stata generally uses the degrees of freedom of the survey design.}
\CommentTok{# therefore, while this extended syntax serves to prove a precise replication of stata}
\CommentTok{# it is generally not necessary.}
\NormalTok{section_four_one <-}
\StringTok{    }\KeywordTok{data.frame}\NormalTok{( }
        \DataTypeTok{estimate =} \KeywordTok{coef}\NormalTok{( result.lin ) , }
        \DataTypeTok{standard_error =} \KeywordTok{SE}\NormalTok{( result.lin ) , }
        \DataTypeTok{ci_lower_bound =} 
            \KeywordTok{coef}\NormalTok{( result.lin ) }\OperatorTok{+}\StringTok{ }
\StringTok{            }\KeywordTok{SE}\NormalTok{( result.lin ) }\OperatorTok{*}\StringTok{ }
\StringTok{            }\KeywordTok{qt}\NormalTok{( }\FloatTok{0.025}\NormalTok{ , }\KeywordTok{degf}\NormalTok{( }\KeywordTok{subset}\NormalTok{( des_nlsw88 , }\OperatorTok{!}\KeywordTok{is.na}\NormalTok{( wage ) ) ) ) ,}
        \DataTypeTok{ci_upper_bound =} 
            \KeywordTok{coef}\NormalTok{( result.lin ) }\OperatorTok{+}\StringTok{ }
\StringTok{            }\KeywordTok{SE}\NormalTok{( result.lin ) }\OperatorTok{*}\StringTok{ }
\StringTok{            }\KeywordTok{qt}\NormalTok{( }\FloatTok{0.975}\NormalTok{ , }\KeywordTok{degf}\NormalTok{( }\KeywordTok{subset}\NormalTok{( des_nlsw88 , }\OperatorTok{!}\KeywordTok{is.na}\NormalTok{( wage ) ) ) )}
\NormalTok{    )}
    

\NormalTok{knitr}\OperatorTok{::}\KeywordTok{kable}\NormalTok{(}
\NormalTok{  section_four_one , }\DataTypeTok{caption =} \StringTok{'Here is a nice table!'}\NormalTok{,}
  \DataTypeTok{booktabs =} \OtherTok{TRUE}
\NormalTok{)}
\end{Highlighting}
\end{Shaded}

\begin{table}

\caption{\label{tab:unnamed-chunk-36}Here is a nice table!}
\centering
\begin{tabular}[t]{lrrrr}
\toprule
  & estimate & standard\_error & ci\_lower\_bound & ci\_upper\_bound\\
\midrule
0 & 0.0000000 & 0.0000000 & 0.0000000 & 0.0000000\\
0.05 & 0.0151060 & 0.0004159 & 0.0142904 & 0.0159216\\
0.1 & 0.0342651 & 0.0007021 & 0.0328882 & 0.0356420\\
0.15 & 0.0558635 & 0.0010096 & 0.0538836 & 0.0578434\\
0.2 & 0.0801846 & 0.0014032 & 0.0774329 & 0.0829363\\
\addlinespace
0.25 & 0.1067687 & 0.0017315 & 0.1033732 & 0.1101642\\
0.3 & 0.1356307 & 0.0021301 & 0.1314535 & 0.1398078\\
0.35 & 0.1670287 & 0.0025182 & 0.1620903 & 0.1719670\\
0.4 & 0.2005501 & 0.0029161 & 0.1948315 & 0.2062687\\
0.45 & 0.2369209 & 0.0033267 & 0.2303971 & 0.2434447\\
\addlinespace
0.5 & 0.2759734 & 0.0037423 & 0.2686347 & 0.2833121\\
0.55 & 0.3180215 & 0.0041626 & 0.3098585 & 0.3261844\\
0.6 & 0.3633071 & 0.0045833 & 0.3543192 & 0.3722950\\
0.65 & 0.4125183 & 0.0050056 & 0.4027021 & 0.4223345\\
0.7 & 0.4657641 & 0.0054137 & 0.4551478 & 0.4763804\\
\addlinespace
0.75 & 0.5241784 & 0.0058003 & 0.5128039 & 0.5355529\\
0.8 & 0.5880894 & 0.0062464 & 0.5758401 & 0.6003388\\
0.85 & 0.6577051 & 0.0066148 & 0.6447333 & 0.6706769\\
0.9 & 0.7346412 & 0.0068289 & 0.7212497 & 0.7480328\\
0.95 & 0.8265786 & 0.0062686 & 0.8142857 & 0.8388715\\
1 & 1.0000000 & 0.0000000 & 1.0000000 & 1.0000000\\
\bottomrule
\end{tabular}
\end{table}

For additional usage examples of \texttt{svylorenz}, type
\texttt{?convey::svylorenz} in the R console.

\section{Gini index (svygini)}\label{gini-index-svygini}

The Gini index is an attempt to express the inequality presented in the
Lorenz curve as a single number. In essence, it is twice the area
between the equality curve and the real Lorenz curve. Put simply:

\[
\begin{aligned}
G &= 2 \bigg( \int_{0}^{1} pdp - \int_{0}^{1} L(p)dp \bigg) \\
\therefore G &= 1 - 2 \int_{0}^{1} L(p)dp
\end{aligned}
\]

where \(G=0\) in case of perfect equality and \(G = 1\) in the case of
perfect inequality.

The estimator proposed by \citep{osier2009} is defined as:

\[
\widehat{G} = \frac{ 2 \sum_{i \in S} w_i r_i y_i - \sum_{i \in S} w_i y_i }{ \hat{Y} }
\]

The linearized formula of \(\widehat{G}\) is used to calculate the SE.

\begin{center}\rule{0.5\linewidth}{\linethickness}\end{center}

\textbf{A replication example}

The R \texttt{vardpoor} package \citep{vardpoor}, created by researchers
at the Central Statistical Bureau of Latvia, includes a gini coefficient
calculation using the ultimate cluster method. The example below
reproduces those statistics.

Load and prepare the same data set:

\begin{Shaded}
\begin{Highlighting}[]
\CommentTok{# load the convey package}
\KeywordTok{library}\NormalTok{(convey)}

\CommentTok{# load the survey library}
\KeywordTok{library}\NormalTok{(survey)}

\CommentTok{# load the vardpoor library}
\KeywordTok{library}\NormalTok{(vardpoor)}

\CommentTok{# load the synthetic european union statistics on income & living conditions}
\KeywordTok{data}\NormalTok{(eusilc)}

\CommentTok{# make all column names lowercase}
\KeywordTok{names}\NormalTok{( eusilc ) <-}\StringTok{ }\KeywordTok{tolower}\NormalTok{( }\KeywordTok{names}\NormalTok{( eusilc ) )}

\CommentTok{# add a column with the row number}
\NormalTok{dati <-}\StringTok{ }\KeywordTok{data.table}\NormalTok{(}\DataTypeTok{IDd =} \DecValTok{1} \OperatorTok{:}\StringTok{ }\KeywordTok{nrow}\NormalTok{(eusilc), eusilc)}

\CommentTok{# calculate the gini coefficient}
\CommentTok{# using the R vardpoor library}
\NormalTok{varpoord_gini_calculation <-}
\StringTok{    }\KeywordTok{varpoord}\NormalTok{(}
    
        \CommentTok{# analysis variable}
        \DataTypeTok{Y =} \StringTok{"eqincome"}\NormalTok{, }
        
        \CommentTok{# weights variable}
        \DataTypeTok{w_final =} \StringTok{"rb050"}\NormalTok{,}
        
        \CommentTok{# row number variable}
        \DataTypeTok{ID_level1 =} \StringTok{"IDd"}\NormalTok{,}
        
        \CommentTok{# row number variable}
        \DataTypeTok{ID_level2 =} \StringTok{"IDd"}\NormalTok{,}
        
        \CommentTok{# strata variable}
        \DataTypeTok{H =} \StringTok{"db040"}\NormalTok{, }
        
        \DataTypeTok{N_h =} \OtherTok{NULL}\NormalTok{ ,}
        
        \CommentTok{# clustering variable}
        \DataTypeTok{PSU =} \StringTok{"rb030"}\NormalTok{, }
        
        \CommentTok{# data.table}
        \DataTypeTok{dataset =}\NormalTok{ dati, }
        
        \CommentTok{# gini coefficient function}
        \DataTypeTok{type =} \StringTok{"lingini"}\NormalTok{,}
      
      \CommentTok{# poverty threshold range}
      \DataTypeTok{order_quant =}\NormalTok{ 50L ,}
      
      \CommentTok{# get linearized variable}
      \DataTypeTok{outp_lin =} \OtherTok{TRUE}
        
\NormalTok{    )}
\end{Highlighting}
\end{Shaded}

\begin{verbatim}
## NULL
\end{verbatim}

\begin{Shaded}
\begin{Highlighting}[]
\CommentTok{# construct a survey.design}
\CommentTok{# using our recommended setup}
\NormalTok{des_eusilc <-}\StringTok{ }
\StringTok{    }\KeywordTok{svydesign}\NormalTok{( }
        \DataTypeTok{ids =} \OperatorTok{~}\StringTok{ }\NormalTok{rb030 , }
        \DataTypeTok{strata =} \OperatorTok{~}\StringTok{ }\NormalTok{db040 ,  }
        \DataTypeTok{weights =} \OperatorTok{~}\StringTok{ }\NormalTok{rb050 , }
        \DataTypeTok{data =}\NormalTok{ eusilc}
\NormalTok{    )}

\CommentTok{# immediately run the convey_prep function on it}
\NormalTok{des_eusilc <-}\StringTok{ }\KeywordTok{convey_prep}\NormalTok{( des_eusilc )}

\CommentTok{# coefficients do match}
\NormalTok{varpoord_gini_calculation}\OperatorTok{$}\NormalTok{all_result}\OperatorTok{$}\NormalTok{value}
\end{Highlighting}
\end{Shaded}

\begin{verbatim}
## [1] 26.49652
\end{verbatim}

\begin{Shaded}
\begin{Highlighting}[]
\KeywordTok{coef}\NormalTok{( }\KeywordTok{svygini}\NormalTok{( }\OperatorTok{~}\StringTok{ }\NormalTok{eqincome , des_eusilc ) ) }\OperatorTok{*}\StringTok{ }\DecValTok{100}
\end{Highlighting}
\end{Shaded}

\begin{verbatim}
## eqincome 
## 26.49652
\end{verbatim}

\begin{Shaded}
\begin{Highlighting}[]
\CommentTok{# linearized variables do match}
\CommentTok{# varpoord}
\NormalTok{lin_gini_varpoord<-}\StringTok{ }\NormalTok{varpoord_gini_calculation}\OperatorTok{$}\NormalTok{lin_out}\OperatorTok{$}\NormalTok{lin_gini}
\CommentTok{# convey }
\NormalTok{lin_gini_convey <-}\StringTok{ }\KeywordTok{attr}\NormalTok{(}\KeywordTok{svygini}\NormalTok{( }\OperatorTok{~}\StringTok{ }\NormalTok{eqincome , des_eusilc ),}\StringTok{"lin"}\NormalTok{)}

\CommentTok{# check equality}
\KeywordTok{all.equal}\NormalTok{(lin_gini_varpoord,}\DecValTok{100}\OperatorTok{*}\NormalTok{lin_gini_convey )}
\end{Highlighting}
\end{Shaded}

\begin{verbatim}
## [1] TRUE
\end{verbatim}

\begin{Shaded}
\begin{Highlighting}[]
\CommentTok{# variances do not match exactly}
\KeywordTok{attr}\NormalTok{( }\KeywordTok{svygini}\NormalTok{( }\OperatorTok{~}\StringTok{ }\NormalTok{eqincome , des_eusilc ) , }\StringTok{'var'}\NormalTok{ ) }\OperatorTok{*}\StringTok{ }\DecValTok{10000}
\end{Highlighting}
\end{Shaded}

\begin{verbatim}
##            eqincome
## eqincome 0.03790739
\end{verbatim}

\begin{Shaded}
\begin{Highlighting}[]
\NormalTok{varpoord_gini_calculation}\OperatorTok{$}\NormalTok{all_result}\OperatorTok{$}\NormalTok{var}
\end{Highlighting}
\end{Shaded}

\begin{verbatim}
## [1] 0.03783931
\end{verbatim}

\begin{Shaded}
\begin{Highlighting}[]
\CommentTok{# standard errors do not match exactly}
\NormalTok{varpoord_gini_calculation}\OperatorTok{$}\NormalTok{all_result}\OperatorTok{$}\NormalTok{se}
\end{Highlighting}
\end{Shaded}

\begin{verbatim}
## [1] 0.1945233
\end{verbatim}

\begin{Shaded}
\begin{Highlighting}[]
\KeywordTok{SE}\NormalTok{( }\KeywordTok{svygini}\NormalTok{( }\OperatorTok{~}\StringTok{ }\NormalTok{eqincome , des_eusilc ) ) }\OperatorTok{*}\StringTok{ }\DecValTok{100}
\end{Highlighting}
\end{Shaded}

\begin{verbatim}
##           eqincome
## eqincome 0.1946982
\end{verbatim}

The variance estimate is computed by using the approximation defined in
\eqref{eq:var}, where the linearized variable \(z\) is defined by
\eqref{eq:lin}. The functions \texttt{convey::svygini} and
\texttt{vardpoor::lingini} produce the same linearized variable \(z\).

However, the measures of uncertainty do not line up, because
\texttt{library(vardpoor)} defaults to an ultimate cluster method that
can be replicated with an alternative setup of the
\texttt{survey.design} object.

\begin{Shaded}
\begin{Highlighting}[]
\CommentTok{# within each strata, sum up the weights}
\NormalTok{cluster_sums <-}\StringTok{ }\KeywordTok{aggregate}\NormalTok{( eusilc}\OperatorTok{$}\NormalTok{rb050 , }\KeywordTok{list}\NormalTok{( eusilc}\OperatorTok{$}\NormalTok{db040 ) , sum )}

\CommentTok{# name the within-strata sums of weights the `cluster_sum`}
\KeywordTok{names}\NormalTok{( cluster_sums ) <-}\StringTok{ }\KeywordTok{c}\NormalTok{( }\StringTok{"db040"}\NormalTok{ , }\StringTok{"cluster_sum"}\NormalTok{ )}

\CommentTok{# merge this column back onto the data.frame}
\NormalTok{eusilc <-}\StringTok{ }\KeywordTok{merge}\NormalTok{( eusilc , cluster_sums )}

\CommentTok{# construct a survey.design}
\CommentTok{# with the fpc using the cluster sum}
\NormalTok{des_eusilc_ultimate_cluster <-}\StringTok{ }
\StringTok{    }\KeywordTok{svydesign}\NormalTok{( }
        \DataTypeTok{ids =} \OperatorTok{~}\StringTok{ }\NormalTok{rb030 , }
        \DataTypeTok{strata =} \OperatorTok{~}\StringTok{ }\NormalTok{db040 ,  }
        \DataTypeTok{weights =} \OperatorTok{~}\StringTok{ }\NormalTok{rb050 , }
        \DataTypeTok{data =}\NormalTok{ eusilc , }
        \DataTypeTok{fpc =} \OperatorTok{~}\StringTok{ }\NormalTok{cluster_sum }
\NormalTok{    )}

\CommentTok{# again, immediately run the convey_prep function on the `survey.design`}
\NormalTok{des_eusilc_ultimate_cluster <-}\StringTok{ }\KeywordTok{convey_prep}\NormalTok{( des_eusilc_ultimate_cluster )}

\CommentTok{# matches}
\KeywordTok{attr}\NormalTok{( }\KeywordTok{svygini}\NormalTok{( }\OperatorTok{~}\StringTok{ }\NormalTok{eqincome , des_eusilc_ultimate_cluster ) , }\StringTok{'var'}\NormalTok{ ) }\OperatorTok{*}\StringTok{ }\DecValTok{10000}
\end{Highlighting}
\end{Shaded}

\begin{verbatim}
##            eqincome
## eqincome 0.03783931
\end{verbatim}

\begin{Shaded}
\begin{Highlighting}[]
\NormalTok{varpoord_gini_calculation}\OperatorTok{$}\NormalTok{all_result}\OperatorTok{$}\NormalTok{var}
\end{Highlighting}
\end{Shaded}

\begin{verbatim}
## [1] 0.03783931
\end{verbatim}

\begin{Shaded}
\begin{Highlighting}[]
\CommentTok{# matches}
\NormalTok{varpoord_gini_calculation}\OperatorTok{$}\NormalTok{all_result}\OperatorTok{$}\NormalTok{se}
\end{Highlighting}
\end{Shaded}

\begin{verbatim}
## [1] 0.1945233
\end{verbatim}

\begin{Shaded}
\begin{Highlighting}[]
\KeywordTok{SE}\NormalTok{( }\KeywordTok{svygini}\NormalTok{( }\OperatorTok{~}\StringTok{ }\NormalTok{eqincome , des_eusilc_ultimate_cluster ) ) }\OperatorTok{*}\StringTok{ }\DecValTok{100}
\end{Highlighting}
\end{Shaded}

\begin{verbatim}
##           eqincome
## eqincome 0.1945233
\end{verbatim}

For additional usage examples of \texttt{svygini}, type
\texttt{?convey::svygini} in the R console.

\section{Amato index (svyamato)}\label{amato-index-svyamato}

The Amato index is also based on the Lorenz curve, but instead of
focusing on the area of the curve, it focuses on its length.
\citep{arnold2012} proposes a formula not directly based in the Lorenz
curve, which \citep{barabesi2016} uses to present the following
estimator:

\[
\widehat{A} = \sum_{i \in S} w_i \bigg[ \frac{1}{\widehat{N}^2} + \frac{y_i^2}{\widehat{Y}^2} \bigg]^{\frac{1}{2}} \text{,}
\]

which also generates the linearized formula for SE estimation.

The minimum value \(A\) assumes is \(\sqrt{2}\) and the maximum is
\(2\). In order to get a measure in the interval \([0,1]\), the
standardized Amato index \(\widetilde{A}\) can be defined as:

\[
\widetilde{A} = \frac{ A - \sqrt{2} }{2 - \sqrt{2} } \text{ .}
\]

For additional usage examples of \texttt{svyamato}, type
\texttt{?convey::svyamato} in the R console.

\section{Zenga Index and Curve (svyzenga,
svyzengacurve)}\label{zenga-index-and-curve-svyzenga-svyzengacurve}

The Zenga index and its curve were proposed in \citep{zenga2007}. As
\citep{polisicchio2011} noticed, this curve derives directly from the
Lorenz curve, and can be defined as:

\[
Z(p) = 1 - \frac{L(p)}{p} \cdot \frac{1 - p}{1 - L(p)}.
\]

In the \texttt{convey} library, an experimental estimator based on the
Lorenz curve is used:

\[
\widehat{Z(p)} = \frac{ p \widehat{Y} - \widehat{\widetilde{Y}}(p) }{p \big[ \widehat{Y} - \widehat{\widetilde{Y}}(p) \big] }.
\]

In turn, the Zenga index derives from this curve and is defined as:

\[
Z = \int_0^1 Z(p)dp.
\]

However, its estimators were proposed by \citep{langel2012} and
\citep{barabesi2016}. In this library, the latter is used and is defined
as:

\[
\widehat{Z} = 1 - \sum_{i \in S} w_i \bigg[ \frac{ ( \widehat{N} - \widehat{H}_{y_i} ) ( \widehat{Y} -\widehat{K}_{y_i} ) }
{ \widehat{N} \cdot \widehat{H}_{y_i} \cdot \widehat{K}_{y_i} } \bigg]
\]

where \(\widehat{N}\) is the population total, \(\widehat{Y}\) is the
total income, \(\widehat{H}_{y_i}\) is the sum of incomes below or equal
to \(y_i\) and \(\widehat{N}_{y_i}\) is the sum of incomes greater or
equal to \(y_i\).

For additional usage examples of \texttt{svyzenga} or
\texttt{svyzengacurve}, type \texttt{?convey::svyzenga} or
\texttt{?convey::svyzengacurve} in the R console.

\section{Entropy-based Measures}\label{entropy-based-measures}

Entropy is a concept derived from information theory, meaning the
expected amount of information given the occurrence of an event.
Following \citep{shannon1948}, given an event \(y\) with probability
density function \(f(\cdot)\), the information content given the
occurrence of \(y\) can be defined as \(g(f(y)) \colon= - \log f(y)\).
Therefore, the expected information or, put simply, the \emph{entropy}
is

\[
H(f) \colon = -E \big[ \log f(y) \big] = - \int_{-\infty}^{\infty} f(y) \log f(y) dy
\]

Assuming a discrete distribution, with \(p_k\) as the probability of
occurring event \(k \in K\), the entropy formula takes the form:

\[
H = - \sum_{k \in K} p_k \log p_k \text{.}
\]

The main idea behind it is that the expected amount of information of an
event is inversely proportional to the probability of its occurrence. In
other words, the information derived from the observation of a rare
event is higher than of the information of more probable events.

Using the intuition presented in \citep{cowell2009}, substituting the
density function by the income share of an individual
\(s(q) = {F}^{-1}(q) / \int_{0}^{1} F^{-1}(t)dt = y/\mu\), the entropy
function becomes the Theil inequality index

\[
I_{Theil} = \int_{0}^{\infty} \frac{y}{\mu} \log \bigg( \frac{y}{\mu} \bigg) dF(y) = -H(s)
\]

Therefore, the entropy-based inequality measure increases as a person's
income \(y\) deviates from the mean \(\mu\). This is the basic idea
behind entropy-based inequality measures.

\section{Generalized Entropy and Decomposition (svygei,
svygeidec)}\label{generalized-entropy-and-decomposition-svygei-svygeidec}

Using a generalization of the information function, now defined as
\(g(f) = \frac{1}{\alpha-1} [ 1 - f^{\alpha - 1} ]\), the
\(\alpha\)-class entropy is \[
H_\alpha(f) = \frac{1}{\alpha - 1} \bigg[ 1 - \int_{-\infty}^{\infty} f(y)^{ \alpha - 1} f(y) dy \bigg] \text{.}
\]

This relates to a class of inequality measures, the Generalized entropy
indices, defined as:

\[
GE_\alpha = \frac{1}{\alpha^2 - \alpha} \int_{0}^\infty \bigg[ \bigg( \frac{y}{\mu} \bigg)^\alpha - 1 \bigg]dF(x) = - \frac{-H_\alpha(s) }{ \alpha } \text{.}
\]

The parameter \(\alpha\) also has an economic interpretation: as
\(\alpha\) increases, the influence of top incomes upon the index
increases. In some cases, this measure takes special forms, such as mean
log deviation and the aforementioned Theil index.

In order to estimate it, \citep{biewen2003} proposed the following:

\[
GE_\alpha =
\begin{cases}
( \alpha^2 - \alpha)^{-1} \big[ U_0^{\alpha - 1} U_1^{-\alpha} U_\alpha -1 \big], & \text{if } \alpha \in \mathbb{R} \setminus \{0,1\} \\
- T_0 U_0^{-1} + \log ( U_1 / U_0 ), &\text{if } \alpha \rightarrow 0 \\
T_1 U_1^{-1} - \log ( U_1 / U_0 ), & \text{if } \alpha \rightarrow 1
\end{cases}
\]

where \(U_\gamma = \sum_{i \in S} w_i \cdot y_i^\gamma\) and
\(T_\gamma = \sum_{i \in S} w_i \cdot y_i^\gamma \cdot \log y_i\). since
those are all functions of totals, the linearization of the indices are
easily achieved using the theorems described in \citep{deville1999}.

This class also has several desirable properties, such as additive
decomposition. The additive decomposition allows to compare the effects
of inequality within and between population groups on the population
inequality. Put simply, an additive decomposable index allows for:

\[
I_{Total} = I_{Between} + I_{Within} \text{.}
\]

\begin{center}\rule{0.5\linewidth}{\linethickness}\end{center}

\textbf{A replication example}

In July 2006, \citep{jenkins2006} presented at the North American Stata
Users' Group Meetings on the stata Generalized Entropy Index command.
The example below reproduces those statistics.

Load and prepare the same data set:

\begin{Shaded}
\begin{Highlighting}[]
\CommentTok{# load the convey package}
\KeywordTok{library}\NormalTok{(convey)}

\CommentTok{# load the survey library}
\KeywordTok{library}\NormalTok{(survey)}

\CommentTok{# load the foreign library}
\KeywordTok{library}\NormalTok{(foreign)}

\CommentTok{# create a temporary file on the local disk}
\NormalTok{tf <-}\StringTok{ }\KeywordTok{tempfile}\NormalTok{()}

\CommentTok{# store the location of the presentation file}
\NormalTok{presentation_zip <-}\StringTok{ "http://repec.org/nasug2006/nasug2006_jenkins.zip"}

\CommentTok{# download jenkins' presentation to the temporary file}
\KeywordTok{download.file}\NormalTok{( presentation_zip , tf , }\DataTypeTok{mode =} \StringTok{'wb'}\NormalTok{ )}

\CommentTok{# unzip the contents of the archive}
\NormalTok{presentation_files <-}\StringTok{ }\KeywordTok{unzip}\NormalTok{( tf , }\DataTypeTok{exdir =} \KeywordTok{tempdir}\NormalTok{() )}

\CommentTok{# load the institute for fiscal studies' 1981, 1985, and 1991 data.frame objects}
\NormalTok{x81 <-}\StringTok{ }\KeywordTok{read.dta}\NormalTok{( }\KeywordTok{grep}\NormalTok{( }\StringTok{"ifs81"}\NormalTok{ , presentation_files , }\DataTypeTok{value =} \OtherTok{TRUE}\NormalTok{ ) )}
\NormalTok{x85 <-}\StringTok{ }\KeywordTok{read.dta}\NormalTok{( }\KeywordTok{grep}\NormalTok{( }\StringTok{"ifs85"}\NormalTok{ , presentation_files , }\DataTypeTok{value =} \OtherTok{TRUE}\NormalTok{ ) )}
\NormalTok{x91 <-}\StringTok{ }\KeywordTok{read.dta}\NormalTok{( }\KeywordTok{grep}\NormalTok{( }\StringTok{"ifs91"}\NormalTok{ , presentation_files , }\DataTypeTok{value =} \OtherTok{TRUE}\NormalTok{ ) )}

\CommentTok{# stack each of these three years of data into a single data.frame}
\NormalTok{x <-}\StringTok{ }\KeywordTok{rbind}\NormalTok{( x81 , x85 , x91 )}
\end{Highlighting}
\end{Shaded}

Replicate the author's survey design statement from stata code..

\begin{verbatim}
. * account for clustering within HHs 
. version 8: svyset [pweight = wgt], psu(hrn)
pweight is wgt
psu is hrn
construct an
\end{verbatim}

.. into R code:

\begin{Shaded}
\begin{Highlighting}[]
\CommentTok{# initiate a linearized survey design object}
\NormalTok{y <-}\StringTok{ }\KeywordTok{svydesign}\NormalTok{( }\OperatorTok{~}\StringTok{ }\NormalTok{hrn , }\DataTypeTok{data =}\NormalTok{ x , }\DataTypeTok{weights =} \OperatorTok{~}\StringTok{ }\NormalTok{wgt )}

\CommentTok{# immediately run the `convey_prep` function on the survey design}
\NormalTok{z <-}\StringTok{ }\KeywordTok{convey_prep}\NormalTok{( y )}
\end{Highlighting}
\end{Shaded}

Replicate the author's subset statement and each of his svygei results..

\begin{verbatim}
. svygei x if year == 1981
 
Warning: x has 20 values = 0. Not used in calculations

Complex survey estimates of Generalized Entropy inequality indices
 
pweight: wgt                                   Number of obs    = 9752
Strata: <one>                                  Number of strata = 1
PSU: hrn                                       Number of PSUs   = 7459
                                               Population size  = 54766261
---------------------------------------------------------------------------
Index    |  Estimate   Std. Err.      z      P>|z|     [95% Conf. Interval]
---------+-----------------------------------------------------------------
GE(-1)   |  .1902062   .02474921     7.69    0.000      .1416987   .2387138
MLD      |  .1142851   .00275138    41.54    0.000      .1088925   .1196777
Theil    |  .1116923   .00226489    49.31    0.000      .1072532   .1161314
GE(2)    |   .128793   .00330774    38.94    0.000      .1223099    .135276
GE(3)    |  .1739994   .00662015    26.28    0.000      .1610242   .1869747
---------------------------------------------------------------------------
\end{verbatim}

..using R code:

\begin{Shaded}
\begin{Highlighting}[]
\NormalTok{z81 <-}\StringTok{ }\KeywordTok{subset}\NormalTok{( z , year }\OperatorTok{==}\StringTok{ }\DecValTok{1981}\NormalTok{ )}

\KeywordTok{svygei}\NormalTok{( }\OperatorTok{~}\StringTok{ }\NormalTok{eybhc0 , }\KeywordTok{subset}\NormalTok{( z81 , eybhc0 }\OperatorTok{>}\StringTok{ }\DecValTok{0}\NormalTok{ ) , }\DataTypeTok{epsilon =} \OperatorTok{-}\DecValTok{1}\NormalTok{ )}
\end{Highlighting}
\end{Shaded}

\begin{verbatim}
##            gei     SE
## eybhc0 0.19021 0.0247
\end{verbatim}

\begin{Shaded}
\begin{Highlighting}[]
\KeywordTok{svygei}\NormalTok{( }\OperatorTok{~}\StringTok{ }\NormalTok{eybhc0 , }\KeywordTok{subset}\NormalTok{( z81 , eybhc0 }\OperatorTok{>}\StringTok{ }\DecValTok{0}\NormalTok{ ) , }\DataTypeTok{epsilon =} \DecValTok{0}\NormalTok{ )}
\end{Highlighting}
\end{Shaded}

\begin{verbatim}
##            gei     SE
## eybhc0 0.11429 0.0028
\end{verbatim}

\begin{Shaded}
\begin{Highlighting}[]
\KeywordTok{svygei}\NormalTok{( }\OperatorTok{~}\StringTok{ }\NormalTok{eybhc0 , }\KeywordTok{subset}\NormalTok{( z81 , eybhc0 }\OperatorTok{>}\StringTok{ }\DecValTok{0}\NormalTok{ ) )}
\end{Highlighting}
\end{Shaded}

\begin{verbatim}
##            gei     SE
## eybhc0 0.11169 0.0023
\end{verbatim}

\begin{Shaded}
\begin{Highlighting}[]
\KeywordTok{svygei}\NormalTok{( }\OperatorTok{~}\StringTok{ }\NormalTok{eybhc0 , }\KeywordTok{subset}\NormalTok{( z81 , eybhc0 }\OperatorTok{>}\StringTok{ }\DecValTok{0}\NormalTok{ ) , }\DataTypeTok{epsilon =} \DecValTok{2}\NormalTok{ )}
\end{Highlighting}
\end{Shaded}

\begin{verbatim}
##            gei     SE
## eybhc0 0.12879 0.0033
\end{verbatim}

\begin{Shaded}
\begin{Highlighting}[]
\KeywordTok{svygei}\NormalTok{( }\OperatorTok{~}\StringTok{ }\NormalTok{eybhc0 , }\KeywordTok{subset}\NormalTok{( z81 , eybhc0 }\OperatorTok{>}\StringTok{ }\DecValTok{0}\NormalTok{ ) , }\DataTypeTok{epsilon =} \DecValTok{3}\NormalTok{ )}
\end{Highlighting}
\end{Shaded}

\begin{verbatim}
##          gei     SE
## eybhc0 0.174 0.0066
\end{verbatim}

Confirm this replication applies for subsetted objects as well. Compare
stata output..

\begin{verbatim}
. svygei x if year == 1985 & x >= 1

Complex survey estimates of Generalized Entropy inequality indices
 
pweight: wgt                                   Number of obs    = 8969
Strata: <one>                                  Number of strata = 1
PSU: hrn                                       Number of PSUs   = 6950
                                               Population size  = 55042871
---------------------------------------------------------------------------
Index    |  Estimate   Std. Err.      z      P>|z|     [95% Conf. Interval]
---------+-----------------------------------------------------------------
GE(-1)   |  .1602358   .00936931    17.10    0.000      .1418723   .1785993
MLD      |   .127616   .00332187    38.42    0.000      .1211052   .1341267
Theil    |  .1337177   .00406302    32.91    0.000      .1257543    .141681
GE(2)    |  .1676393   .00730057    22.96    0.000      .1533304   .1819481
GE(3)    |  .2609507   .01850689    14.10    0.000      .2246779   .2972235
---------------------------------------------------------------------------
\end{verbatim}

..to R code:

\begin{Shaded}
\begin{Highlighting}[]
\NormalTok{z85 <-}\StringTok{ }\KeywordTok{subset}\NormalTok{( z , year }\OperatorTok{==}\StringTok{ }\DecValTok{1985}\NormalTok{ )}

\KeywordTok{svygei}\NormalTok{( }\OperatorTok{~}\StringTok{ }\NormalTok{eybhc0 , }\KeywordTok{subset}\NormalTok{( z85 , eybhc0 }\OperatorTok{>}\StringTok{ }\DecValTok{1}\NormalTok{ ) , }\DataTypeTok{epsilon =} \OperatorTok{-}\DecValTok{1}\NormalTok{ )}
\end{Highlighting}
\end{Shaded}

\begin{verbatim}
##            gei     SE
## eybhc0 0.16024 0.0094
\end{verbatim}

\begin{Shaded}
\begin{Highlighting}[]
\KeywordTok{svygei}\NormalTok{( }\OperatorTok{~}\StringTok{ }\NormalTok{eybhc0 , }\KeywordTok{subset}\NormalTok{( z85 , eybhc0 }\OperatorTok{>}\StringTok{ }\DecValTok{1}\NormalTok{ ) , }\DataTypeTok{epsilon =} \DecValTok{0}\NormalTok{ )}
\end{Highlighting}
\end{Shaded}

\begin{verbatim}
##            gei     SE
## eybhc0 0.12762 0.0033
\end{verbatim}

\begin{Shaded}
\begin{Highlighting}[]
\KeywordTok{svygei}\NormalTok{( }\OperatorTok{~}\StringTok{ }\NormalTok{eybhc0 , }\KeywordTok{subset}\NormalTok{( z85 , eybhc0 }\OperatorTok{>}\StringTok{ }\DecValTok{1}\NormalTok{ ) )}
\end{Highlighting}
\end{Shaded}

\begin{verbatim}
##            gei     SE
## eybhc0 0.13372 0.0041
\end{verbatim}

\begin{Shaded}
\begin{Highlighting}[]
\KeywordTok{svygei}\NormalTok{( }\OperatorTok{~}\StringTok{ }\NormalTok{eybhc0 , }\KeywordTok{subset}\NormalTok{( z85 , eybhc0 }\OperatorTok{>}\StringTok{ }\DecValTok{1}\NormalTok{ ) , }\DataTypeTok{epsilon =} \DecValTok{2}\NormalTok{ )}
\end{Highlighting}
\end{Shaded}

\begin{verbatim}
##            gei     SE
## eybhc0 0.16764 0.0073
\end{verbatim}

\begin{Shaded}
\begin{Highlighting}[]
\KeywordTok{svygei}\NormalTok{( }\OperatorTok{~}\StringTok{ }\NormalTok{eybhc0 , }\KeywordTok{subset}\NormalTok{( z85 , eybhc0 }\OperatorTok{>}\StringTok{ }\DecValTok{1}\NormalTok{ ) , }\DataTypeTok{epsilon =} \DecValTok{3}\NormalTok{ )}
\end{Highlighting}
\end{Shaded}

\begin{verbatim}
##            gei     SE
## eybhc0 0.26095 0.0185
\end{verbatim}

Replicate the author's decomposition by population subgroup (work
status) shown on PDF page 57..

\begin{Shaded}
\begin{Highlighting}[]
\CommentTok{# define work status (PDF page 22)}
\NormalTok{z <-}\StringTok{ }\KeywordTok{update}\NormalTok{( z , }\DataTypeTok{wkstatus =} \KeywordTok{c}\NormalTok{( }\DecValTok{1}\NormalTok{ , }\DecValTok{1}\NormalTok{ , }\DecValTok{1}\NormalTok{ , }\DecValTok{1}\NormalTok{ , }\DecValTok{2}\NormalTok{ , }\DecValTok{3}\NormalTok{ , }\DecValTok{2}\NormalTok{ , }\DecValTok{2}\NormalTok{ )[ }\KeywordTok{as.numeric}\NormalTok{( esbu ) ] )}
\NormalTok{z <-}\StringTok{ }\KeywordTok{update}\NormalTok{( z , }\KeywordTok{factor}\NormalTok{( wkstatus , }\DataTypeTok{labels =} \KeywordTok{c}\NormalTok{( }\StringTok{"1+ ft working"}\NormalTok{ , }\StringTok{"no ft working"}\NormalTok{ , }\StringTok{"elderly"}\NormalTok{ ) ) )}

\CommentTok{# subset to 1991 and remove records with zero income}
\NormalTok{z91 <-}\StringTok{ }\KeywordTok{subset}\NormalTok{( z , year }\OperatorTok{==}\StringTok{ }\DecValTok{1991} \OperatorTok{&}\StringTok{ }\NormalTok{eybhc0 }\OperatorTok{>}\StringTok{ }\DecValTok{0}\NormalTok{ )}

\CommentTok{# population share}
\KeywordTok{svymean}\NormalTok{( }\OperatorTok{~}\NormalTok{wkstatus, z91 )}
\end{Highlighting}
\end{Shaded}

\begin{verbatim}
##            mean     SE
## wkstatus 1.5594 0.0099
\end{verbatim}

\begin{Shaded}
\begin{Highlighting}[]
\CommentTok{# mean}
\KeywordTok{svyby}\NormalTok{( }\OperatorTok{~}\NormalTok{eybhc0, }\OperatorTok{~}\NormalTok{wkstatus, z91, svymean )}
\end{Highlighting}
\end{Shaded}

\begin{verbatim}
##   wkstatus   eybhc0       se
## 1        1 278.8040 3.703790
## 2        2 151.6317 3.153968
## 3        3 176.6045 4.661740
\end{verbatim}

\begin{Shaded}
\begin{Highlighting}[]
\CommentTok{# subgroup indices: ge_k}
\KeywordTok{svyby}\NormalTok{( }\OperatorTok{~}\StringTok{ }\NormalTok{eybhc0 , }\OperatorTok{~}\NormalTok{wkstatus , z91 , svygei , }\DataTypeTok{epsilon =} \OperatorTok{-}\DecValTok{1}\NormalTok{ )}
\end{Highlighting}
\end{Shaded}

\begin{verbatim}
##   wkstatus     eybhc0          se
## 1        1  0.2300708  0.02853959
## 2        2 10.9231761 10.65482557
## 3        3  0.1932164  0.02571991
\end{verbatim}

\begin{Shaded}
\begin{Highlighting}[]
\KeywordTok{svyby}\NormalTok{( }\OperatorTok{~}\StringTok{ }\NormalTok{eybhc0 , }\OperatorTok{~}\NormalTok{wkstatus , z91 , svygei , }\DataTypeTok{epsilon =} \DecValTok{0}\NormalTok{ )}
\end{Highlighting}
\end{Shaded}

\begin{verbatim}
##   wkstatus    eybhc0          se
## 1        1 0.1536921 0.006955506
## 2        2 0.1836835 0.014740510
## 3        3 0.1653658 0.016409770
\end{verbatim}

\begin{Shaded}
\begin{Highlighting}[]
\KeywordTok{svyby}\NormalTok{( }\OperatorTok{~}\StringTok{ }\NormalTok{eybhc0 , }\OperatorTok{~}\NormalTok{wkstatus , z91 , svygei , }\DataTypeTok{epsilon =} \DecValTok{1}\NormalTok{ )}
\end{Highlighting}
\end{Shaded}

\begin{verbatim}
##   wkstatus    eybhc0          se
## 1        1 0.1598558 0.008327994
## 2        2 0.1889909 0.016766120
## 3        3 0.2023862 0.027787224
\end{verbatim}

\begin{Shaded}
\begin{Highlighting}[]
\KeywordTok{svyby}\NormalTok{( }\OperatorTok{~}\StringTok{ }\NormalTok{eybhc0 , }\OperatorTok{~}\NormalTok{wkstatus , z91 , svygei , }\DataTypeTok{epsilon =} \DecValTok{2}\NormalTok{ )}
\end{Highlighting}
\end{Shaded}

\begin{verbatim}
##   wkstatus    eybhc0         se
## 1        1 0.2130664 0.01546521
## 2        2 0.2846345 0.06016394
## 3        3 0.3465088 0.07362898
\end{verbatim}

\begin{Shaded}
\begin{Highlighting}[]
\CommentTok{# GE decomposition}
\KeywordTok{svygeidec}\NormalTok{( }\OperatorTok{~}\NormalTok{eybhc0, }\OperatorTok{~}\NormalTok{wkstatus, z91, }\DataTypeTok{epsilon =} \OperatorTok{-}\DecValTok{1}\NormalTok{ )}
\end{Highlighting}
\end{Shaded}

\begin{verbatim}
##       total within between
## coef 3.6829 3.6466  0.0363
## SE   3.3999 3.3993  0.0541
\end{verbatim}

\begin{Shaded}
\begin{Highlighting}[]
\KeywordTok{svygeidec}\NormalTok{( }\OperatorTok{~}\NormalTok{eybhc0, }\OperatorTok{~}\NormalTok{wkstatus, z91, }\DataTypeTok{epsilon =} \DecValTok{0}\NormalTok{ )}
\end{Highlighting}
\end{Shaded}

\begin{verbatim}
##          total    within between
## coef 0.1952363 0.1619352  0.0333
## SE   0.0064615 0.0062209  0.0027
\end{verbatim}

\begin{Shaded}
\begin{Highlighting}[]
\KeywordTok{svygeidec}\NormalTok{( }\OperatorTok{~}\NormalTok{eybhc0, }\OperatorTok{~}\NormalTok{wkstatus, z91, }\DataTypeTok{epsilon =} \DecValTok{1}\NormalTok{ )}
\end{Highlighting}
\end{Shaded}

\begin{verbatim}
##          total    within between
## coef 0.2003897 0.1693958  0.0310
## SE   0.0079299 0.0044132  0.0073
\end{verbatim}

\begin{Shaded}
\begin{Highlighting}[]
\KeywordTok{svygeidec}\NormalTok{( }\OperatorTok{~}\NormalTok{eybhc0, }\OperatorTok{~}\NormalTok{wkstatus, z91, }\DataTypeTok{epsilon =} \DecValTok{2}\NormalTok{ )}
\end{Highlighting}
\end{Shaded}

\begin{verbatim}
##         total   within between
## coef 0.274325 0.245067  0.0293
## SE   0.016694 0.017831  0.0038
\end{verbatim}

For additional usage examples of \texttt{svygei} or \texttt{svygeidec},
type \texttt{?convey::svygei} or \texttt{?convey::svygeidec} in the R
console.

\section{Rényi Divergence (svyrenyi)}\label{renyi-divergence-svyrenyi}

Another measure used in areas like ecology, statistics and information
theory is Rényi divergence measure. Using the formula defined in
\citep{langel2012}, the estimator can be defined as:

\[
\widehat{R}_\alpha =
\begin{cases}
\frac{1}{\alpha - 1} \log \bigg[ \widehat{N}^{\alpha - 1} \sum_{i \in S} w_i \cdot \bigg( \frac{y_k}{ \widehat{Y} } \bigg) \bigg], &\text{if } \alpha \neq 1, \\
\sum_{i \in S} \frac{w_i y_i}{ \widehat{Y}} \log \frac{\widehat{N} y_i}{\widehat{Y}}, &\text{if } \alpha = 1,
\end{cases}
\]

where \(\alpha\) is a parameter with a similar economic interpretation
to that of the \(GE_\alpha\) index.

For additional usage examples of \texttt{svyrenyi}, type
\texttt{?convey::svyrenyi} in the R console.

\section{J-Divergence and Decomposition (svyjdiv,
svyjdivdec)}\label{j-divergence-and-decomposition-svyjdiv-svyjdivdec}

Proposed by \citep{rohde2016}, the J-divergence measure can be seen as
the sum of \(GE_0\) and \(GE_1\), satisfying axioms that, individually,
those two indices do not. Using \(U_\gamma\) and \(T_\gamma\) functions
defined in \ref{subsection.3.3.1}, the estimator can be defined as:

\[
\begin{aligned}
\widehat{J} &= \frac{1}{\widehat{N}} \sum_{i \in S} w_i \bigg( \frac{ y_i - \widehat{\mu} }{ \widehat{\mu} } \bigg) \log \bigg( \frac{y_i}{\widehat{\mu}} \bigg) \\
\therefore \widehat{J} &= \frac{\widehat{T}_1}{\widehat{U}_1} - \frac{ \widehat{T}_0 }{ \widehat{U}_0 }
\end{aligned}
\]

Since it is a sum of two additive decomposable measures, \(J\) itself is
decomposable.

For additional usage examples of \texttt{svyjdiv} or
\texttt{svyjdivdec}, type \texttt{?convey::svyjdiv} or
\texttt{?convey::svyjdivdec} in the R console.

\section{Atkinson index (svyatk)}\label{atkinson-index-svyatk}

Although the original formula was proposed in \citep{atkinson1970}, the
estimator used here comes from \citep{biewen2003}:

\[
\widehat{A}_\epsilon =
\begin{cases}
 1 - \widehat{U}_0^{ - \epsilon/(1 - \epsilon) } \widehat{U}_1^{ -1 } \widehat{U}_{1 - \epsilon}^{ 1/(1 - \epsilon) } , &\text{if } \epsilon \in \mathbb{R}_+ \setminus\{ 1 \} \\
1 - \widehat{U}_0 \widehat{U}_0^{-1} exp( \widehat{T}_0 \widehat{U}_0^{-1} ), &\text{if } \epsilon \rightarrow1
\end{cases}
\]

The \(\epsilon\) is an inequality aversion parameter: as it approaches
infinity, more weight is given to incomes in bottom of the distribution.

\begin{center}\rule{0.5\linewidth}{\linethickness}\end{center}

\textbf{A replication example}

In July 2006, \citep{jenkins2006} presented at the North American Stata
Users' Group Meetings on the stata Atkinson Index command. The example
below reproduces those statistics.

Load and prepare the same data set:

\begin{Shaded}
\begin{Highlighting}[]
\CommentTok{# load the convey package}
\KeywordTok{library}\NormalTok{(convey)}

\CommentTok{# load the survey library}
\KeywordTok{library}\NormalTok{(survey)}

\CommentTok{# load the foreign library}
\KeywordTok{library}\NormalTok{(foreign)}

\CommentTok{# create a temporary file on the local disk}
\NormalTok{tf <-}\StringTok{ }\KeywordTok{tempfile}\NormalTok{()}

\CommentTok{# store the location of the presentation file}
\NormalTok{presentation_zip <-}\StringTok{ "http://repec.org/nasug2006/nasug2006_jenkins.zip"}

\CommentTok{# download jenkins' presentation to the temporary file}
\KeywordTok{download.file}\NormalTok{( presentation_zip , tf , }\DataTypeTok{mode =} \StringTok{'wb'}\NormalTok{ )}

\CommentTok{# unzip the contents of the archive}
\NormalTok{presentation_files <-}\StringTok{ }\KeywordTok{unzip}\NormalTok{( tf , }\DataTypeTok{exdir =} \KeywordTok{tempdir}\NormalTok{() )}

\CommentTok{# load the institute for fiscal studies' 1981, 1985, and 1991 data.frame objects}
\NormalTok{x81 <-}\StringTok{ }\KeywordTok{read.dta}\NormalTok{( }\KeywordTok{grep}\NormalTok{( }\StringTok{"ifs81"}\NormalTok{ , presentation_files , }\DataTypeTok{value =} \OtherTok{TRUE}\NormalTok{ ) )}
\NormalTok{x85 <-}\StringTok{ }\KeywordTok{read.dta}\NormalTok{( }\KeywordTok{grep}\NormalTok{( }\StringTok{"ifs85"}\NormalTok{ , presentation_files , }\DataTypeTok{value =} \OtherTok{TRUE}\NormalTok{ ) )}
\NormalTok{x91 <-}\StringTok{ }\KeywordTok{read.dta}\NormalTok{( }\KeywordTok{grep}\NormalTok{( }\StringTok{"ifs91"}\NormalTok{ , presentation_files , }\DataTypeTok{value =} \OtherTok{TRUE}\NormalTok{ ) )}

\CommentTok{# stack each of these three years of data into a single data.frame}
\NormalTok{x <-}\StringTok{ }\KeywordTok{rbind}\NormalTok{( x81 , x85 , x91 )}
\end{Highlighting}
\end{Shaded}

Replicate the author's survey design statement from stata code..

\begin{verbatim}
. * account for clustering within HHs 
. version 8: svyset [pweight = wgt], psu(hrn)
pweight is wgt
psu is hrn
construct an
\end{verbatim}

.. into R code:

\begin{Shaded}
\begin{Highlighting}[]
\CommentTok{# initiate a linearized survey design object}
\NormalTok{y <-}\StringTok{ }\KeywordTok{svydesign}\NormalTok{( }\OperatorTok{~}\StringTok{ }\NormalTok{hrn , }\DataTypeTok{data =}\NormalTok{ x , }\DataTypeTok{weights =} \OperatorTok{~}\StringTok{ }\NormalTok{wgt )}

\CommentTok{# immediately run the `convey_prep` function on the survey design}
\NormalTok{z <-}\StringTok{ }\KeywordTok{convey_prep}\NormalTok{( y )}
\end{Highlighting}
\end{Shaded}

Replicate the author's subset statement and each of his svyatk results
with stata..

\begin{verbatim}
. svyatk x if year == 1981
 
Warning: x has 20 values = 0. Not used in calculations

Complex survey estimates of Atkinson inequality indices
 
pweight: wgt                                   Number of obs    = 9752
Strata: <one>                                  Number of strata = 1
PSU: hrn                                       Number of PSUs   = 7459
                                               Population size  = 54766261
---------------------------------------------------------------------------
Index    |  Estimate   Std. Err.      z      P>|z|     [95% Conf. Interval]
---------+-----------------------------------------------------------------
A(0.5)   |  .0543239   .00107583    50.49    0.000      .0522153   .0564324
A(1)     |  .1079964   .00245424    44.00    0.000      .1031862   .1128066
A(1.5)   |  .1701794   .0066943    25.42    0.000       .1570588      .1833
A(2)     |  .2755788   .02597608    10.61    0.000      .2246666    .326491
A(2.5)   |  .4992701   .06754311     7.39    0.000       .366888   .6316522
---------------------------------------------------------------------------
\end{verbatim}

..using R code:

\begin{Shaded}
\begin{Highlighting}[]
\NormalTok{z81 <-}\StringTok{ }\KeywordTok{subset}\NormalTok{( z , year }\OperatorTok{==}\StringTok{ }\DecValTok{1981}\NormalTok{ )}

\KeywordTok{svyatk}\NormalTok{( }\OperatorTok{~}\StringTok{ }\NormalTok{eybhc0 , }\KeywordTok{subset}\NormalTok{( z81 , eybhc0 }\OperatorTok{>}\StringTok{ }\DecValTok{0}\NormalTok{ ) , }\DataTypeTok{epsilon =} \FloatTok{0.5}\NormalTok{ )}
\end{Highlighting}
\end{Shaded}

\begin{verbatim}
##        atkinson     SE
## eybhc0 0.054324 0.0011
\end{verbatim}

\begin{Shaded}
\begin{Highlighting}[]
\KeywordTok{svyatk}\NormalTok{( }\OperatorTok{~}\StringTok{ }\NormalTok{eybhc0 , }\KeywordTok{subset}\NormalTok{( z81 , eybhc0 }\OperatorTok{>}\StringTok{ }\DecValTok{0}\NormalTok{ ) )}
\end{Highlighting}
\end{Shaded}

\begin{verbatim}
##        atkinson     SE
## eybhc0    0.108 0.0025
\end{verbatim}

\begin{Shaded}
\begin{Highlighting}[]
\KeywordTok{svyatk}\NormalTok{( }\OperatorTok{~}\StringTok{ }\NormalTok{eybhc0 , }\KeywordTok{subset}\NormalTok{( z81 , eybhc0 }\OperatorTok{>}\StringTok{ }\DecValTok{0}\NormalTok{ ) , }\DataTypeTok{epsilon =} \FloatTok{1.5}\NormalTok{ )}
\end{Highlighting}
\end{Shaded}

\begin{verbatim}
##        atkinson     SE
## eybhc0  0.17018 0.0067
\end{verbatim}

\begin{Shaded}
\begin{Highlighting}[]
\KeywordTok{svyatk}\NormalTok{( }\OperatorTok{~}\StringTok{ }\NormalTok{eybhc0 , }\KeywordTok{subset}\NormalTok{( z81 , eybhc0 }\OperatorTok{>}\StringTok{ }\DecValTok{0}\NormalTok{ ) , }\DataTypeTok{epsilon =} \DecValTok{2}\NormalTok{ )}
\end{Highlighting}
\end{Shaded}

\begin{verbatim}
##        atkinson    SE
## eybhc0  0.27558 0.026
\end{verbatim}

\begin{Shaded}
\begin{Highlighting}[]
\KeywordTok{svyatk}\NormalTok{( }\OperatorTok{~}\StringTok{ }\NormalTok{eybhc0 , }\KeywordTok{subset}\NormalTok{( z81 , eybhc0 }\OperatorTok{>}\StringTok{ }\DecValTok{0}\NormalTok{ ) , }\DataTypeTok{epsilon =} \FloatTok{2.5}\NormalTok{ )}
\end{Highlighting}
\end{Shaded}

\begin{verbatim}
##        atkinson     SE
## eybhc0  0.49927 0.0675
\end{verbatim}

Confirm this replication applies for subsetted objects as well,
comparing stata code..

\begin{verbatim}
. svyatk x if year == 1981 & x >= 1

Complex survey estimates of Atkinson inequality indices
 
pweight: wgt                                   Number of obs    = 9748
Strata: <one>                                  Number of strata = 1
PSU: hrn                                       Number of PSUs   = 7457
                                               Population size  = 54744234
---------------------------------------------------------------------------
Index    |  Estimate   Std. Err.      z      P>|z|     [95% Conf. Interval]
---------+-----------------------------------------------------------------
A(0.5)   |  .0540059   .00105011    51.43    0.000      .0519477   .0560641
A(1)     |  .1066082   .00223318    47.74    0.000      .1022313   .1109852
A(1.5)   |  .1638299   .00483069    33.91    0.000       .154362   .1732979
A(2)     |  .2443206   .01425258    17.14    0.000      .2163861   .2722552
A(2.5)   |   .394787   .04155221     9.50    0.000      .3133461   .4762278
---------------------------------------------------------------------------
\end{verbatim}

..to R code:

\begin{Shaded}
\begin{Highlighting}[]
\NormalTok{z81_two <-}\StringTok{ }\KeywordTok{subset}\NormalTok{( z , year }\OperatorTok{==}\StringTok{ }\DecValTok{1981} \OperatorTok{&}\StringTok{ }\NormalTok{eybhc0 }\OperatorTok{>}\StringTok{ }\DecValTok{1}\NormalTok{ )}

\KeywordTok{svyatk}\NormalTok{( }\OperatorTok{~}\StringTok{ }\NormalTok{eybhc0 , z81_two , }\DataTypeTok{epsilon =} \FloatTok{0.5}\NormalTok{ )}
\end{Highlighting}
\end{Shaded}

\begin{verbatim}
##        atkinson     SE
## eybhc0 0.054006 0.0011
\end{verbatim}

\begin{Shaded}
\begin{Highlighting}[]
\KeywordTok{svyatk}\NormalTok{( }\OperatorTok{~}\StringTok{ }\NormalTok{eybhc0 , z81_two )}
\end{Highlighting}
\end{Shaded}

\begin{verbatim}
##        atkinson     SE
## eybhc0  0.10661 0.0022
\end{verbatim}

\begin{Shaded}
\begin{Highlighting}[]
\KeywordTok{svyatk}\NormalTok{( }\OperatorTok{~}\StringTok{ }\NormalTok{eybhc0 , z81_two , }\DataTypeTok{epsilon =} \FloatTok{1.5}\NormalTok{ )}
\end{Highlighting}
\end{Shaded}

\begin{verbatim}
##        atkinson     SE
## eybhc0  0.16383 0.0048
\end{verbatim}

\begin{Shaded}
\begin{Highlighting}[]
\KeywordTok{svyatk}\NormalTok{( }\OperatorTok{~}\StringTok{ }\NormalTok{eybhc0 , z81_two , }\DataTypeTok{epsilon =} \DecValTok{2}\NormalTok{ )}
\end{Highlighting}
\end{Shaded}

\begin{verbatim}
##        atkinson     SE
## eybhc0  0.24432 0.0143
\end{verbatim}

\begin{Shaded}
\begin{Highlighting}[]
\KeywordTok{svyatk}\NormalTok{( }\OperatorTok{~}\StringTok{ }\NormalTok{eybhc0 , z81_two , }\DataTypeTok{epsilon =} \FloatTok{2.5}\NormalTok{ )}
\end{Highlighting}
\end{Shaded}

\begin{verbatim}
##        atkinson     SE
## eybhc0  0.39479 0.0416
\end{verbatim}

For additional usage examples of \texttt{svyatk}, type
\texttt{?convey::svyatk} in the R console.

\chapter{Multidimensional Indices}\label{multidimensional}

Inequality and poverty can be seen as multidimensional concepts,
combining several livelihood characteristics. Usual approaches take into
account income, housing, sanitation, etc.

In order to transform these different measures from into meaningful
numbers, economic theory builds on the idea of utility functions.
Utility is a measure of well-being, assigning a ``well-being score'' to
a vector of characteristics. Depending on the utility function, the
analyst may allow for substitutions among characteristics: for instance,
someone with a slightly lower income, but with access to sanitation, can
have a higher wellbeing than someone with a higher income, but without
access to sanitation. This depends on the set of weights given to the
set of attributes.

Most measures below follow from this kind of two-step procedure: (1)
estimating individual scores from an individual's set of
characteristics; then (2) aggregating those individual scores into a
single measure for the population.

The following section will present measures of multidimensional poverty
and inequality, describing the main aspects of the theory and estimation
procedures of each.

\section{Alkire-Foster Class and Decomposition (svyafc,
svyafcdec)}\label{alkire-foster-class-and-decomposition-svyafc-svyafcdec}

This class of measures are defined in \citet{alkire2011}, using what is
called the ``dual cutoff'' approach. This method applies a cutoffs to
define dimensional deprivations and another cutoff for multidimensional
deprivation.

To analyze a population of \(n\) individuals across \(d\) achievement
dimensions, the first step of the method is applying a FGT-like
transformation to each dimension, defined as

\[
g_{ij}^\alpha = \bigg( \frac{ z_j - x_{ij} }{ z_j } \bigg)^{\alpha}
\]

where \(i\) is an observation index, \(j\) is a dimension index and
\(\alpha\) is an exponent weighting the deprivation intensity. If
\(\alpha=0\), then \(g_{ij}^0\) becomes a binary variable, assuming
value \(1\) if person \(i\) is deprived in dimension \(j\) and \(0\)
otherwise. The \(n \times d\) matrix \(G^\alpha\) will be referred to as
\emph{deprivation matrix}.

Each dimension receives a weight \(w_j\), so that the weighted sum of
multidimensional deprivation is the matrix multiplication of
\(G^\alpha\) by the \(j \times 1\) vector \(W = [w_j]\). The
\(n \times 1\) vector \(C^\alpha = [c^\alpha_i]\) is the weighted sum of
dimensional deprivation scores, i.e.,

\[
c^\alpha_{i} = \sum_{j \in d} w_j g_{ij}^\alpha
\]

The second cutoff is defining those considered to be multidimensionally
poor. Assuming that \(\sum_{j \in d} w_j = 1\), the multidimensional
cutoff \(k\) belongs to the interval \((0,1]\). If
\(c^0_{i} \geqslant k\), then this person is considered
multidimensionally poor. The \emph{censored vector of deprivation sums}
\(C^\alpha(k)\) is defined as

\[
C^\alpha (k) = \bigg[ c_{ij}^\alpha \cdot \delta \big( c_{ij}^0 \geqslant k \big) \bigg] \text{,}
\]

where \(\delta(A)\) is an indicator function, taking value \(1\) if
condition \(A\) is true and \(0\) otherwise. If
\(k \geqslant \min{ w_j }\), this is called the ``union approach'',
where a person is considered poor if she is poor in at least one
dimension. On the other extreme, the ``intersection approach'' happens
when \(k = 1\), meaning that a person is considered poor if she is poor
in all dimensions.

The average of vector \(C^0 (k)\) returns the multidimensional headcount
ratio. For the multidimensional FGT class, a general measure can be
defined as

\[
M^\alpha = \frac{1}{n} \sum_{i \in n} \sum_{j \in d} w_j g_{ij}^{\alpha}(k) \text{, } \alpha \geq 0 \text{,}
\]

where
\(g_{ij}^{\alpha}(k) = g_{ij}^\alpha \cdot \delta \big( c^0_i \geqslant k \big)\).

For inferential purposes, since this variable is actually the average of
scores \(\sum_{j \in d} w_j g_{ij}^{\alpha}(k)\), the linearization is
straightforward.

The Alkire-Foster index is both dimensional and subgroup decomposable.
This way, it is possible to analyze how much each dimension or group
contribute to the general result. The overall poverty measure can be
seen as the weighted sum of each group's poverty measure, as in the
formula below:

\[
M^\alpha = \sum_{l \in L} \frac{ n_l }{ n } M^\alpha_{l}
\]

where \(l\) is one of \(L\) groups.

Also, the overall poverty index can be expressed across dimensions as \[
M^\alpha = \sum_{j \in d} w_j \bigg[ \frac{1}{n} \sum_{i \in n} g_{ij}^\alpha (k) \bigg] \text{.}
\]

Since those functions are linear combinations of ratios and totals, it
is also possible to calculate standard errors for such measures.

\begin{center}\rule{0.5\linewidth}{\linethickness}\end{center}

\textbf{A replication example}

In November 2015, Christopher Jindra presented at the Oxford Poverty and
Human Development Initiative on the Alkire-Foster multidimensional
poverty measure. His presentation can be viewed
\href{http://www.ophi.org.uk/wp-content/uploads/Jindra_151109_OPHISeminar.pdf}{here}.
The example below reproduces those statistics.

Load and prepare the same data set:

\begin{Shaded}
\begin{Highlighting}[]
\CommentTok{# load the convey package}
\KeywordTok{library}\NormalTok{(convey)}

\CommentTok{# load the survey library}
\KeywordTok{library}\NormalTok{(survey)}

\CommentTok{# load the stata-style webuse library}
\KeywordTok{library}\NormalTok{(webuse)}

\CommentTok{# load the same microdata set used by Jindra in his presentation}
\KeywordTok{webuse}\NormalTok{(}\StringTok{"nlsw88"}\NormalTok{)}

\CommentTok{# coerce that `tbl_df` to a standard R `data.frame`}
\NormalTok{nlsw88 <-}\StringTok{ }\KeywordTok{data.frame}\NormalTok{( nlsw88 )}

\CommentTok{# create a `collgrad` column}
\NormalTok{nlsw88}\OperatorTok{$}\NormalTok{collgrad <-}
\StringTok{    }\KeywordTok{factor}\NormalTok{( }
        \KeywordTok{as.numeric}\NormalTok{( nlsw88}\OperatorTok{$}\NormalTok{collgrad ) , }
        \DataTypeTok{label =} \KeywordTok{c}\NormalTok{( }\StringTok{'not college grad'}\NormalTok{ , }\StringTok{'college grad'}\NormalTok{ ) , }
        \DataTypeTok{ordered =} \OtherTok{TRUE} 
\NormalTok{      )}

\CommentTok{# coerce `married` column to factor}
\NormalTok{nlsw88}\OperatorTok{$}\NormalTok{married <-}\StringTok{ }
\StringTok{    }\KeywordTok{factor}\NormalTok{( }
\NormalTok{        nlsw88}\OperatorTok{$}\NormalTok{married , }
        \DataTypeTok{levels =} \DecValTok{0}\OperatorTok{:}\DecValTok{1}\NormalTok{ , }
        \DataTypeTok{labels =} \KeywordTok{c}\NormalTok{( }\StringTok{"single"}\NormalTok{ , }\StringTok{"married"}\NormalTok{ ) }
\NormalTok{    )}

\CommentTok{# initiate a linearized survey design object}
\NormalTok{des_nlsw88 <-}\StringTok{ }\KeywordTok{svydesign}\NormalTok{( }\DataTypeTok{ids =} \OperatorTok{~}\DecValTok{1}\NormalTok{ , }\DataTypeTok{data =}\NormalTok{ nlsw88 )}

\CommentTok{# immediately run the `convey_prep` function on the survey design}
\NormalTok{des_nlsw88 <-}\StringTok{ }\KeywordTok{convey_prep}\NormalTok{(des_nlsw88)}
\end{Highlighting}
\end{Shaded}

Replicate PDF page 9

\begin{Shaded}
\begin{Highlighting}[]
\NormalTok{page_nine <-}
\StringTok{  }\KeywordTok{svyafc}\NormalTok{(}
    \OperatorTok{~}\StringTok{ }\NormalTok{wage }\OperatorTok{+}\StringTok{ }\NormalTok{collgrad }\OperatorTok{+}\StringTok{ }\NormalTok{hours , }
    \DataTypeTok{design =}\NormalTok{ des_nlsw88 , }
    \DataTypeTok{cutoffs =} \KeywordTok{list}\NormalTok{( }\DecValTok{4}\NormalTok{, }\StringTok{'college grad'}\NormalTok{ , }\DecValTok{26}\NormalTok{ ) , }
    \DataTypeTok{k =} \DecValTok{1}\OperatorTok{/}\DecValTok{3}\NormalTok{ , }\DataTypeTok{g =} \DecValTok{0}\NormalTok{ , }
    \DataTypeTok{na.rm =} \OtherTok{TRUE}
\NormalTok{  )}

\CommentTok{# MO and seMO}
\KeywordTok{print}\NormalTok{( page_nine )}
\end{Highlighting}
\end{Shaded}

\begin{verbatim}
##      alkire-foster     SE
## [1,]       0.36991 0.0053
\end{verbatim}

\begin{Shaded}
\begin{Highlighting}[]
\CommentTok{# H seH and A seA}
\KeywordTok{print}\NormalTok{( }\KeywordTok{attr}\NormalTok{( page_nine , }\StringTok{"extra"}\NormalTok{ ) )}
\end{Highlighting}
\end{Shaded}

\begin{verbatim}
##        coef          SE
## H 0.8082070 0.008316807
## A 0.4576895 0.004573443
\end{verbatim}

Replicate PDF page 10

\begin{Shaded}
\begin{Highlighting}[]
\NormalTok{page_ten <-}\StringTok{ }\OtherTok{NULL}

\CommentTok{# loop through every poverty cutoff `k`}
\ControlFlowTok{for}\NormalTok{( ks }\ControlFlowTok{in} \KeywordTok{seq}\NormalTok{( }\FloatTok{0.1}\NormalTok{ , }\DecValTok{1}\NormalTok{ , .}\DecValTok{1}\NormalTok{ ) )\{}
    
\NormalTok{    this_ks <-}
\StringTok{        }\KeywordTok{svyafc}\NormalTok{(}
            \OperatorTok{~}\StringTok{ }\NormalTok{wage }\OperatorTok{+}\StringTok{ }\NormalTok{collgrad }\OperatorTok{+}\StringTok{ }\NormalTok{hours , }
            \DataTypeTok{design =}\NormalTok{ des_nlsw88 , }
            \DataTypeTok{cutoffs =} \KeywordTok{list}\NormalTok{( }\DecValTok{4}\NormalTok{ , }\StringTok{'college grad'}\NormalTok{ , }\DecValTok{26}\NormalTok{ ) , }
            \DataTypeTok{k =}\NormalTok{ ks , }
            \DataTypeTok{g =} \DecValTok{0}\NormalTok{ , }
            \DataTypeTok{na.rm =} \OtherTok{TRUE} 
\NormalTok{           )}
    
\NormalTok{    page_ten <-}
\StringTok{        }\KeywordTok{rbind}\NormalTok{(}
\NormalTok{            page_ten ,}
            \KeywordTok{data.frame}\NormalTok{( }
                \DataTypeTok{k =}\NormalTok{ ks , }
                \DataTypeTok{MO =} \KeywordTok{coef}\NormalTok{( this_ks ) ,}
                \DataTypeTok{seMO =} \KeywordTok{SE}\NormalTok{( this_ks ) ,}
                \DataTypeTok{H =} \KeywordTok{attr}\NormalTok{( this_ks , }\StringTok{"extra"}\NormalTok{ )[ }\DecValTok{1}\NormalTok{ , }\DecValTok{1}\NormalTok{ ] ,}
                \DataTypeTok{seH =} \KeywordTok{attr}\NormalTok{( this_ks , }\StringTok{"extra"}\NormalTok{ )[ }\DecValTok{1}\NormalTok{ , }\DecValTok{2}\NormalTok{ ] ,}
                \DataTypeTok{A =} \KeywordTok{attr}\NormalTok{( this_ks , }\StringTok{"extra"}\NormalTok{ )[ }\DecValTok{2}\NormalTok{ , }\DecValTok{1}\NormalTok{ ] ,}
                \DataTypeTok{seA =} \KeywordTok{attr}\NormalTok{( this_ks , }\StringTok{"extra"}\NormalTok{ )[ }\DecValTok{2}\NormalTok{ , }\DecValTok{2}\NormalTok{ ]}
\NormalTok{          )}
\NormalTok{        )}
    
\NormalTok{\}}
\end{Highlighting}
\end{Shaded}

\begin{table}

\caption{\label{tab:unnamed-chunk-57}PDF Page 10 Replication}
\centering
\begin{tabular}[t]{lrrrrrrr}
\toprule
  & k & MO & seMO & H & seH & A & seA\\
\midrule
alkire-foster & 0.1 & 0.3699078 & 0.0053059 & 0.8082070 & 0.0083168 & 0.4576895 & 0.0045734\\
alkire-foster1 & 0.2 & 0.3699078 & 0.0053059 & 0.8082070 & 0.0083168 & 0.4576895 & 0.0045734\\
alkire-foster2 & 0.3 & 0.3699078 & 0.0053059 & 0.8082070 & 0.0083168 & 0.4576895 & 0.0045734\\
alkire-foster3 & 0.4 & 0.1865894 & 0.0068123 & 0.2582516 & 0.0092455 & 0.7225101 & 0.0051745\\
alkire-foster4 & 0.5 & 0.1865894 & 0.0068123 & 0.2582516 & 0.0092455 & 0.7225101 & 0.0051745\\
\addlinespace
alkire-foster5 & 0.6 & 0.1865894 & 0.0068123 & 0.2582516 & 0.0092455 & 0.7225101 & 0.0051745\\
alkire-foster6 & 0.7 & 0.0432649 & 0.0042978 & 0.0432649 & 0.0042978 & 1.0000000 & 0.0000000\\
alkire-foster7 & 0.8 & 0.0432649 & 0.0042978 & 0.0432649 & 0.0042978 & 1.0000000 & 0.0000000\\
alkire-foster8 & 0.9 & 0.0432649 & 0.0042978 & 0.0432649 & 0.0042978 & 1.0000000 & 0.0000000\\
alkire-foster9 & 1.0 & 0.0432649 & 0.0042978 & 0.0432649 & 0.0042978 & 1.0000000 & 0.0000000\\
\bottomrule
\end{tabular}
\end{table}

Replicate PDF page 13

\begin{Shaded}
\begin{Highlighting}[]
\NormalTok{page_thirteen <-}\StringTok{ }\OtherTok{NULL}

\CommentTok{# loop through every poverty cutoff `k`}
\ControlFlowTok{for}\NormalTok{( ks }\ControlFlowTok{in} \KeywordTok{c}\NormalTok{( }\FloatTok{0.5}\NormalTok{ , }\FloatTok{0.75}\NormalTok{ , }\DecValTok{1}\NormalTok{ ) )\{}
    
\NormalTok{    this_ks <-}
\StringTok{        }\KeywordTok{svyafc}\NormalTok{(}
        \OperatorTok{~}\StringTok{ }\NormalTok{wage }\OperatorTok{+}\StringTok{ }\NormalTok{collgrad }\OperatorTok{+}\StringTok{ }\NormalTok{hours , }
        \DataTypeTok{design =}\NormalTok{ des_nlsw88 , }
        \DataTypeTok{cutoffs =} \KeywordTok{list}\NormalTok{( }\DecValTok{4}\NormalTok{, }\StringTok{'college grad'}\NormalTok{ , }\DecValTok{26}\NormalTok{ ) , }
        \DataTypeTok{k =}\NormalTok{ ks , }
        \DataTypeTok{g =} \DecValTok{0}\NormalTok{ , }
        \DataTypeTok{dimw =} \KeywordTok{c}\NormalTok{( }\FloatTok{0.5}\NormalTok{ , }\FloatTok{0.25}\NormalTok{ , }\FloatTok{0.25}\NormalTok{ ) ,}
        \DataTypeTok{na.rm =} \OtherTok{TRUE}
\NormalTok{      )}
    
\NormalTok{    page_thirteen <-}
\StringTok{        }\KeywordTok{rbind}\NormalTok{(}
\NormalTok{            page_thirteen ,}
            \KeywordTok{data.frame}\NormalTok{( }
                \DataTypeTok{k =}\NormalTok{ ks , }
                \DataTypeTok{MO =} \KeywordTok{coef}\NormalTok{( this_ks ) ,}
                \DataTypeTok{seMO =} \KeywordTok{SE}\NormalTok{( this_ks ) ,}
                \DataTypeTok{H =} \KeywordTok{attr}\NormalTok{( this_ks , }\StringTok{"extra"}\NormalTok{ )[ }\DecValTok{1}\NormalTok{ , }\DecValTok{1}\NormalTok{ ] ,}
                \DataTypeTok{seH =} \KeywordTok{attr}\NormalTok{( this_ks , }\StringTok{"extra"}\NormalTok{ )[ }\DecValTok{1}\NormalTok{ , }\DecValTok{2}\NormalTok{ ] ,}
                \DataTypeTok{A =} \KeywordTok{attr}\NormalTok{( this_ks , }\StringTok{"extra"}\NormalTok{ )[ }\DecValTok{2}\NormalTok{ , }\DecValTok{1}\NormalTok{ ] ,}
                \DataTypeTok{seA =} \KeywordTok{attr}\NormalTok{( this_ks , }\StringTok{"extra"}\NormalTok{ )[ }\DecValTok{2}\NormalTok{ , }\DecValTok{2}\NormalTok{ ]}
\NormalTok{          )}
\NormalTok{        )}
\NormalTok{\}}
\end{Highlighting}
\end{Shaded}

\begin{table}

\caption{\label{tab:unnamed-chunk-59}PDF Page 13 Replication}
\centering
\begin{tabular}[t]{lrrrrrrr}
\toprule
  & k & MO & seMO & H & seH & A & seA\\
\midrule
alkire-foster & 0.50 & 0.1913470 & 0.0069137 & 0.2689563 & 0.0093668 & 0.7114428 & 0.0068474\\
alkire-foster1 & 0.75 & 0.1489741 & 0.0066918 & 0.1842105 & 0.0081889 & 0.8087167 & 0.0052160\\
alkire-foster2 & 1.00 & 0.0432649 & 0.0042978 & 0.0432649 & 0.0042978 & 1.0000000 & 0.0000000\\
\bottomrule
\end{tabular}
\end{table}

Replicate PDF page 16

\begin{Shaded}
\begin{Highlighting}[]
\NormalTok{page_sixteen <-}\StringTok{ }\OtherTok{NULL}

\CommentTok{# loop through every alpha value `g`}
\ControlFlowTok{for}\NormalTok{( gs }\ControlFlowTok{in} \DecValTok{0}\OperatorTok{:}\DecValTok{3}\NormalTok{ )\{}
    
\NormalTok{    this_gs <-}
\StringTok{        }\KeywordTok{svyafc}\NormalTok{(}
        \OperatorTok{~}\StringTok{ }\NormalTok{wage }\OperatorTok{+}\StringTok{ }\NormalTok{collgrad }\OperatorTok{+}\StringTok{ }\NormalTok{hours , }
        \DataTypeTok{design =}\NormalTok{ des_nlsw88 , }
        \DataTypeTok{cutoffs =} \KeywordTok{list}\NormalTok{( }\DecValTok{4}\NormalTok{, }\StringTok{'college grad'}\NormalTok{ , }\DecValTok{26}\NormalTok{ ) , }
        \DataTypeTok{k =} \DecValTok{1}\OperatorTok{/}\DecValTok{3}\NormalTok{ , }
        \DataTypeTok{g =}\NormalTok{ gs , }
        \DataTypeTok{na.rm =} \OtherTok{TRUE}
\NormalTok{      )}
    
\NormalTok{    page_sixteen <-}
\StringTok{        }\KeywordTok{rbind}\NormalTok{(}
\NormalTok{            page_sixteen ,}
            \KeywordTok{data.frame}\NormalTok{( }
                \DataTypeTok{g =}\NormalTok{ gs , }
                \DataTypeTok{MO =} \KeywordTok{coef}\NormalTok{( this_gs ) ,}
                \DataTypeTok{seMO =} \KeywordTok{SE}\NormalTok{( this_gs ) }
\NormalTok{          )}
\NormalTok{        )}
\NormalTok{\}}
\end{Highlighting}
\end{Shaded}

\begin{table}

\caption{\label{tab:unnamed-chunk-61}PDF Page 16 Replication}
\centering
\begin{tabular}[t]{lrrr}
\toprule
  & g & MO & seMO\\
\midrule
alkire-foster & 0 & 0.3699078 & 0.0053059\\
alkire-foster1 & 1 & 0.2859332 & 0.0033708\\
alkire-foster2 & 2 & 0.2676266 & 0.0031164\\
alkire-foster3 & 3 & 0.2616335 & 0.0030531\\
\bottomrule
\end{tabular}
\end{table}

Replicate k=1/3 rows of PDF page 17 and 19

\begin{Shaded}
\begin{Highlighting}[]
\KeywordTok{svyafcdec}\NormalTok{(}
    \OperatorTok{~}\StringTok{ }\NormalTok{wage }\OperatorTok{+}\StringTok{ }\NormalTok{collgrad }\OperatorTok{+}\StringTok{ }\NormalTok{hours , }
    \DataTypeTok{design =}\NormalTok{ des_nlsw88 , }
    \DataTypeTok{cutoffs =} \KeywordTok{list}\NormalTok{( }\DecValTok{4}\NormalTok{ , }\StringTok{'college grad'}\NormalTok{ , }\DecValTok{26}\NormalTok{ ) , }
    \DataTypeTok{k =} \DecValTok{1}\OperatorTok{/}\DecValTok{3}\NormalTok{ , }
    \DataTypeTok{g =} \DecValTok{0}\NormalTok{ ,}
    \DataTypeTok{na.rm =} \OtherTok{TRUE}
\NormalTok{)}
\end{Highlighting}
\end{Shaded}

\begin{verbatim}
## $overall
##               alkire-foster     SE
## alkire-foster       0.36991 0.0053
## 
## $`raw headcount ratio`
##          raw headcount     SE
## wage           0.19492 0.0084
## collgrad       0.76316 0.0090
## hours          0.15165 0.0076
## 
## $`censored headcount ratio`
##          cens. headcount     SE
## wage             0.19492 0.0084
## collgrad         0.76316 0.0090
## hours            0.15165 0.0076
## 
## $`percentual contribution per dimension`
##          dim. % contribution     SE
## wage                 0.17564 0.0061
## collgrad             0.68770 0.0077
## hours                0.13666 0.0059
\end{verbatim}

Replicate PDF pages 21 and 22

\begin{Shaded}
\begin{Highlighting}[]
\KeywordTok{svyafcdec}\NormalTok{(}
    \OperatorTok{~}\StringTok{ }\NormalTok{wage }\OperatorTok{+}\StringTok{ }\NormalTok{collgrad }\OperatorTok{+}\StringTok{ }\NormalTok{hours , }
    \DataTypeTok{subgroup =} \OperatorTok{~}\NormalTok{married , }
    \DataTypeTok{design =}\NormalTok{ des_nlsw88 , }
    \DataTypeTok{cutoffs =} \KeywordTok{list}\NormalTok{( }\DecValTok{4}\NormalTok{ , }\StringTok{'college grad'}\NormalTok{ , }\DecValTok{26}\NormalTok{ ) , }
    \DataTypeTok{k =} \DecValTok{1}\OperatorTok{/}\DecValTok{3}\NormalTok{ , }
    \DataTypeTok{g =} \DecValTok{0}\NormalTok{ ,}
    \DataTypeTok{na.rm =} \OtherTok{TRUE}
\NormalTok{)}
\end{Highlighting}
\end{Shaded}

\begin{verbatim}
## $overall
##               alkire-foster     SE
## alkire-foster       0.36991 0.0053
## 
## $`raw headcount ratio`
##          raw headcount     SE
## wage           0.19492 0.0084
## collgrad       0.76316 0.0090
## hours          0.15165 0.0076
## 
## $`censored headcount ratio`
##          cens. headcount     SE
## wage             0.19492 0.0084
## collgrad         0.76316 0.0090
## hours            0.15165 0.0076
## 
## $`percentual contribution per dimension`
##          dim. % contribution     SE
## wage                 0.17564 0.0061
## collgrad             0.68770 0.0077
## hours                0.13666 0.0059
## 
## $`subgroup alkire-foster estimates`
##         alkire-foster     SE
## single        0.35414 0.0088
## married       0.37867 0.0066
## 
## $`percentual contribution per subgroup`
##         grp. % contribution    SE
## single              0.34204 0.012
## married             0.65796 0.012
\end{verbatim}

For additional usage examples of \texttt{svyafc} or \texttt{svyafcdec},
type \texttt{?convey::svyafc} or \texttt{?convey::svyafcdec} in the R
console.

\citep{alkire2011} and \citep{alkire2015} and \citep{pacifico2016}

\section{Bourguignon-Chakravarty (2003) multidimensional poverty
class}\label{bourguignon-chakravarty-2003-multidimensional-poverty-class}

A class of poverty measures is proposed in \citet{bourguignon2003},
using a cross-dimensional function that assigns values to each set of
dimensionally normalized poverty gaps. It can be defined as: \[
BCh = \sum_{i \in n} \bigg[ \bigg( \sum_{j \in d} w_{j} x_{ij} \bigg)^{\frac{1}{\theta}} \bigg]^\alpha \text{, } \theta > 0 \text{, } \alpha > 0
\] where \(x_{ij}\) being the normalized poverty gap of dimension \(j\)
for observation \(i\), \(w_j\) is the weight of dimension \(j\),
\(\theta\) and \(\alpha\) are parameters of the function.

The parameter \(\theta\) is the elasticity of subsitution between the
normalized gaps. In another words, \(\theta\) defines the order of the
weighted generalized mean across achievement dimensions. For instance,
when \(\theta = 1\), the cross-dimensional aggregation becomes the
weighted average of all dimensions. As \(\theta\) increases, the
importance of the individual's most deprived dimension increases. As
\citet{vega2009} points out, it also weights the inequality among
deprivations. In its turn, \(\alpha\) works as society's
poverty-aversion measure parameter. In another words, as \(\alpha\)
increases, more weight is given to the most deprived individuals.
Similar to \(\theta\), when \(\alpha = 1\), \(BCh\) is the average of
the weighted deprivation scores.

\section{Bourguignon (1999) inequality class
(svybmi)}\label{bourguignon-1999-inequality-class-svybmi}

\citet{bourguignon1999} proposes a multidimensional inequality index
that possesses interesting properties related to the correlation among
the welfare dimensions measured. The estimator used in \texttt{convey}
comes from the formula presented in \citet{lugo2007} and is defined as:
\[
B_{I} = 1 - \frac{1}{ \widehat{N} } \frac{ \sum_{ i \in S } w_i \bigg[ \sum_{ j \in d } w_j x_{ij} \bigg]^{ \alpha / \beta } }{ 
\bigg[ \sum_{ j \in d } w_j \mu_{ij} \bigg]^{ \alpha / \beta } },
\] where \(\alpha \geqslant 0\) is an inequality-aversion parameter and
\(\beta \leqslant 1\) is a parameter defining the degree of substitution
among dimensions.

This measure is strong scale-invariant when \(\beta = 0\), although
\citet{bourguignon1999} demonstrates that strong scale-dependent
measures might be interesting in the context of multidimensional
inequality. Also, it can be shown that stronger correlation among
dimensions leads to less inequality if \(\beta > \alpha\).

For additional usage examples of \texttt{svybmi}, type
\texttt{?convey::svybmi} in the R console.

\bibliography{packages,book}


\end{document}
