\documentclass[]{book}
\usepackage{lmodern}
\usepackage{amssymb,amsmath}
\usepackage{ifxetex,ifluatex}
\usepackage{fixltx2e} % provides \textsubscript
\ifnum 0\ifxetex 1\fi\ifluatex 1\fi=0 % if pdftex
  \usepackage[T1]{fontenc}
  \usepackage[utf8]{inputenc}
\else % if luatex or xelatex
  \ifxetex
    \usepackage{mathspec}
  \else
    \usepackage{fontspec}
  \fi
  \defaultfontfeatures{Ligatures=TeX,Scale=MatchLowercase}
\fi
% use upquote if available, for straight quotes in verbatim environments
\IfFileExists{upquote.sty}{\usepackage{upquote}}{}
% use microtype if available
\IfFileExists{microtype.sty}{%
\usepackage{microtype}
\UseMicrotypeSet[protrusion]{basicmath} % disable protrusion for tt fonts
}{}
\usepackage[margin=1in]{geometry}
\usepackage{hyperref}
\hypersetup{unicode=true,
            pdftitle={Poverty and Inequality with Complex Survey Data},
            pdfauthor={Guilherme Jacob, Anthony Damico, and Djalma Pessoa},
            pdfborder={0 0 0},
            breaklinks=true}
\urlstyle{same}  % don't use monospace font for urls
\usepackage{natbib}
\bibliographystyle{apalike}
\usepackage{color}
\usepackage{fancyvrb}
\newcommand{\VerbBar}{|}
\newcommand{\VERB}{\Verb[commandchars=\\\{\}]}
\DefineVerbatimEnvironment{Highlighting}{Verbatim}{commandchars=\\\{\}}
% Add ',fontsize=\small' for more characters per line
\usepackage{framed}
\definecolor{shadecolor}{RGB}{248,248,248}
\newenvironment{Shaded}{\begin{snugshade}}{\end{snugshade}}
\newcommand{\KeywordTok}[1]{\textcolor[rgb]{0.13,0.29,0.53}{\textbf{{#1}}}}
\newcommand{\DataTypeTok}[1]{\textcolor[rgb]{0.13,0.29,0.53}{{#1}}}
\newcommand{\DecValTok}[1]{\textcolor[rgb]{0.00,0.00,0.81}{{#1}}}
\newcommand{\BaseNTok}[1]{\textcolor[rgb]{0.00,0.00,0.81}{{#1}}}
\newcommand{\FloatTok}[1]{\textcolor[rgb]{0.00,0.00,0.81}{{#1}}}
\newcommand{\ConstantTok}[1]{\textcolor[rgb]{0.00,0.00,0.00}{{#1}}}
\newcommand{\CharTok}[1]{\textcolor[rgb]{0.31,0.60,0.02}{{#1}}}
\newcommand{\SpecialCharTok}[1]{\textcolor[rgb]{0.00,0.00,0.00}{{#1}}}
\newcommand{\StringTok}[1]{\textcolor[rgb]{0.31,0.60,0.02}{{#1}}}
\newcommand{\VerbatimStringTok}[1]{\textcolor[rgb]{0.31,0.60,0.02}{{#1}}}
\newcommand{\SpecialStringTok}[1]{\textcolor[rgb]{0.31,0.60,0.02}{{#1}}}
\newcommand{\ImportTok}[1]{{#1}}
\newcommand{\CommentTok}[1]{\textcolor[rgb]{0.56,0.35,0.01}{\textit{{#1}}}}
\newcommand{\DocumentationTok}[1]{\textcolor[rgb]{0.56,0.35,0.01}{\textbf{\textit{{#1}}}}}
\newcommand{\AnnotationTok}[1]{\textcolor[rgb]{0.56,0.35,0.01}{\textbf{\textit{{#1}}}}}
\newcommand{\CommentVarTok}[1]{\textcolor[rgb]{0.56,0.35,0.01}{\textbf{\textit{{#1}}}}}
\newcommand{\OtherTok}[1]{\textcolor[rgb]{0.56,0.35,0.01}{{#1}}}
\newcommand{\FunctionTok}[1]{\textcolor[rgb]{0.00,0.00,0.00}{{#1}}}
\newcommand{\VariableTok}[1]{\textcolor[rgb]{0.00,0.00,0.00}{{#1}}}
\newcommand{\ControlFlowTok}[1]{\textcolor[rgb]{0.13,0.29,0.53}{\textbf{{#1}}}}
\newcommand{\OperatorTok}[1]{\textcolor[rgb]{0.81,0.36,0.00}{\textbf{{#1}}}}
\newcommand{\BuiltInTok}[1]{{#1}}
\newcommand{\ExtensionTok}[1]{{#1}}
\newcommand{\PreprocessorTok}[1]{\textcolor[rgb]{0.56,0.35,0.01}{\textit{{#1}}}}
\newcommand{\AttributeTok}[1]{\textcolor[rgb]{0.77,0.63,0.00}{{#1}}}
\newcommand{\RegionMarkerTok}[1]{{#1}}
\newcommand{\InformationTok}[1]{\textcolor[rgb]{0.56,0.35,0.01}{\textbf{\textit{{#1}}}}}
\newcommand{\WarningTok}[1]{\textcolor[rgb]{0.56,0.35,0.01}{\textbf{\textit{{#1}}}}}
\newcommand{\AlertTok}[1]{\textcolor[rgb]{0.94,0.16,0.16}{{#1}}}
\newcommand{\ErrorTok}[1]{\textcolor[rgb]{0.64,0.00,0.00}{\textbf{{#1}}}}
\newcommand{\NormalTok}[1]{{#1}}
\usepackage{longtable,booktabs}
\usepackage{graphicx,grffile}
\makeatletter
\def\maxwidth{\ifdim\Gin@nat@width>\linewidth\linewidth\else\Gin@nat@width\fi}
\def\maxheight{\ifdim\Gin@nat@height>\textheight\textheight\else\Gin@nat@height\fi}
\makeatother
% Scale images if necessary, so that they will not overflow the page
% margins by default, and it is still possible to overwrite the defaults
% using explicit options in \includegraphics[width, height, ...]{}
\setkeys{Gin}{width=\maxwidth,height=\maxheight,keepaspectratio}
\IfFileExists{parskip.sty}{%
\usepackage{parskip}
}{% else
\setlength{\parindent}{0pt}
\setlength{\parskip}{6pt plus 2pt minus 1pt}
}
\setlength{\emergencystretch}{3em}  % prevent overfull lines
\providecommand{\tightlist}{%
  \setlength{\itemsep}{0pt}\setlength{\parskip}{0pt}}
\setcounter{secnumdepth}{5}
% Redefines (sub)paragraphs to behave more like sections
\ifx\paragraph\undefined\else
\let\oldparagraph\paragraph
\renewcommand{\paragraph}[1]{\oldparagraph{#1}\mbox{}}
\fi
\ifx\subparagraph\undefined\else
\let\oldsubparagraph\subparagraph
\renewcommand{\subparagraph}[1]{\oldsubparagraph{#1}\mbox{}}
\fi

%%% Use protect on footnotes to avoid problems with footnotes in titles
\let\rmarkdownfootnote\footnote%
\def\footnote{\protect\rmarkdownfootnote}

%%% Change title format to be more compact
\usepackage{titling}

% Create subtitle command for use in maketitle
\newcommand{\subtitle}[1]{
  \posttitle{
    \begin{center}\large#1\end{center}
    }
}

\setlength{\droptitle}{-2em}
  \title{Poverty and Inequality with Complex Survey Data}
  \pretitle{\vspace{\droptitle}\centering\huge}
  \posttitle{\par}
  \author{Guilherme Jacob, Anthony Damico, and Djalma Pessoa}
  \preauthor{\centering\large\emph}
  \postauthor{\par}
  \predate{\centering\large\emph}
  \postdate{\par}
  \date{2016-12-09}

\usepackage{booktabs}

\begin{document}
\maketitle

{
\setcounter{tocdepth}{1}
\tableofcontents
}
\chapter{Introduction}\label{introduction}

This is a \emph{sample} book written in \textbf{Markdown}. You can use
anything that Pandoc's Markdown supports, e.g., a math equation
\(a^2 + b^2 = c^2\).

For now, you have to install the development version of
\textbf{bookdown} from Github:

\begin{Shaded}
\begin{Highlighting}[]
\NormalTok{devtools::}\KeywordTok{install_github}\NormalTok{(}\StringTok{"rstudio/bookdown"}\NormalTok{)}
\end{Highlighting}
\end{Shaded}

Remember each Rmd file contains one and only one chapter, and a chapter
is defined by the first-level heading \texttt{\#}.

To compile this example to PDF, you need to install XeLaTeX.

The library convey aims at estimating measures of poverty and income
concentration. There are already at least two libraries covering this
subject: vardpoor and Laeken. The main difference between the library
convey and these two is that the convey strongly hinges on the survey
library.

\section{Installation}\label{install}

\begin{itemize}
\item
  the latest released version from
  \href{https://CRAN.R-project.org/package=convey}{CRAN} with

\begin{Shaded}
\begin{Highlighting}[]
\KeywordTok{install.packages}\NormalTok{(}\StringTok{"convey"}\NormalTok{)}
\end{Highlighting}
\end{Shaded}
\item
  the latest development version from github with

\begin{Shaded}
\begin{Highlighting}[]
\NormalTok{devtools::}\KeywordTok{install_github}\NormalTok{(}\StringTok{"djalmapessoa/convey"}\NormalTok{)}
\end{Highlighting}
\end{Shaded}
\end{itemize}

{[}This may present how to install R, RStudio and required packages.
Providing brief information about \texttt{survey} and
\texttt{MonetDBLite} may also be recommended.{]}

You can label chapter and section titles using \texttt{\{\#label\}}
after them, e.g., we can reference Chapter \ref{install}. If you do not
manually label them, there will be automatic labels anyway, e.g.,
Chapter \ref{inequality}.

Figures and tables with captions will be placed in \texttt{figure} and
\texttt{table} environments, respectively.

\begin{Shaded}
\begin{Highlighting}[]
\KeywordTok{par}\NormalTok{(}\DataTypeTok{mar =} \KeywordTok{c}\NormalTok{(}\DecValTok{4}\NormalTok{, }\DecValTok{4}\NormalTok{, .}\DecValTok{1}\NormalTok{, .}\DecValTok{1}\NormalTok{))}
\KeywordTok{plot}\NormalTok{(pressure, }\DataTypeTok{type =} \StringTok{'b'}\NormalTok{, }\DataTypeTok{pch =} \DecValTok{19}\NormalTok{)}
\end{Highlighting}
\end{Shaded}

\begin{figure}

{\centering \includegraphics[width=0.8\linewidth]{context_files/figure-latex/nice-fig-1} 

}

\caption{Here is a nice figure!}\label{fig:nice-fig}
\end{figure}

Reference a figure by its code chunk label with the \texttt{fig:}
prefix, e.g., see Figure \ref{fig:nice-fig}. Similarly, you can
reference tables generated from \texttt{knitr::kable()}, e.g., see Table
\ref{tab:nice-tab}.

\begin{Shaded}
\begin{Highlighting}[]
\NormalTok{knitr::}\KeywordTok{kable}\NormalTok{(}
  \KeywordTok{head}\NormalTok{(iris, }\DecValTok{20}\NormalTok{), }\DataTypeTok{caption =} \StringTok{'Here is a nice table!'}\NormalTok{,}
  \DataTypeTok{booktabs =} \OtherTok{TRUE}
\NormalTok{)}
\end{Highlighting}
\end{Shaded}

\begin{table}

\caption{\label{tab:nice-tab}Here is a nice table!}
\centering
\begin{tabular}[t]{rrrrl}
\toprule
Sepal.Length & Sepal.Width & Petal.Length & Petal.Width & Species\\
\midrule
5.1 & 3.5 & 1.4 & 0.2 & setosa\\
4.9 & 3.0 & 1.4 & 0.2 & setosa\\
4.7 & 3.2 & 1.3 & 0.2 & setosa\\
4.6 & 3.1 & 1.5 & 0.2 & setosa\\
5.0 & 3.6 & 1.4 & 0.2 & setosa\\
\addlinespace
5.4 & 3.9 & 1.7 & 0.4 & setosa\\
4.6 & 3.4 & 1.4 & 0.3 & setosa\\
5.0 & 3.4 & 1.5 & 0.2 & setosa\\
4.4 & 2.9 & 1.4 & 0.2 & setosa\\
4.9 & 3.1 & 1.5 & 0.1 & setosa\\
\addlinespace
5.4 & 3.7 & 1.5 & 0.2 & setosa\\
4.8 & 3.4 & 1.6 & 0.2 & setosa\\
4.8 & 3.0 & 1.4 & 0.1 & setosa\\
4.3 & 3.0 & 1.1 & 0.1 & setosa\\
5.8 & 4.0 & 1.2 & 0.2 & setosa\\
\addlinespace
5.7 & 4.4 & 1.5 & 0.4 & setosa\\
5.4 & 3.9 & 1.3 & 0.4 & setosa\\
5.1 & 3.5 & 1.4 & 0.3 & setosa\\
5.7 & 3.8 & 1.7 & 0.3 & setosa\\
5.1 & 3.8 & 1.5 & 0.3 & setosa\\
\bottomrule
\end{tabular}
\end{table}

You can write citations, too. For example, we are using the
\textbf{bookdown} package \citep{R-bookdown} in this sample book, which
was built on top of R Markdown and \textbf{knitr} \citep{xie2015}.

\section{Complex surveys and statistical inference}\label{survey}

{[}I think we should have a discussion about what is complex survey, its
importance and so on. We can use a book Djalma wrote.{]}

\section{Linearization}\label{linearization}

Some measures of poverty and income concentration are defined by
non-differentiable functions so that it is not possible to use Taylor
linearization to estimate their variances. An alternative is to use
\textbf{Influence functions} as described in \citep{deville1999} and
\citep{osier2009}. The library convey implements this methodology to
work with \texttt{survey.design} objects and also with
\texttt{svyrep.design} objects.

Some examples of these measures are:

\begin{itemize}
\item
  At-risk-of-poverty threshold: \(arpt=.60q_{.50}\) where \(q_{.50}\) is
  the income median;
\item
  At-risk-of-poverty rate \(arpr=\frac{\sum_U 1(y_i \leq arpt)}{N}.100\)
\item
  Quintile share ratio
\end{itemize}

\(qsr=\frac{\sum_U 1(y_i>q_{.80})}{\sum_U 1(y_i\leq q_{.20})}\)

\begin{itemize}
\tightlist
\item
  Gini coefficient \(1+G=\frac{2\sum_U (r_i-1)y_i}{N\sum_Uy_i}\) where
  \(r_i\) is the rank of \(y_i\).
\end{itemize}

Note that it is not possible to use Taylor linearization for these
measures because they depend on quantiles and the Gini is defined as a
function of ranks. This could be done using the approach proposed by
Deville (1999) based upon influence functions.

\section{Influence function}\label{influence-function}

Let \(U\) be a population of size \(N\) and \(M\) be a measure that
allocates mass one to the set composed by one unit, that is
\(M(i)=M_i= 1\) if \(i\in U\) and \(M(i)=0\) if \(i\notin U\)

Now, a population parameter \(\theta\) can be expressed as a functional
of \(M\) \(\theta=T(M)\)

Examples of such parameters are:

\begin{itemize}
\item
  Total: \(Y=\sum_Uy_i=\sum_U y_iM_i=\int ydM=T(M)\)
\item
  Ratio of two totals:
  \(R=\frac{Y}{X}=\frac{\int y dM}{\int x dM}=T(M)\)
\item
  Cumulative distribution function:
  \(F(x)=\frac{\sum_U 1(y_i\leq x)}{N}=\frac{\int 1(y\leq x)dM}{\int{dM}}=T(M)\)
\end{itemize}

To estimate these parameters from the sample, we replace the measure
\(M\) by the estimated measure \(\hat{M}\) defined by:
\(\hat{M}(i)=\hat{M}_i= w_i\) if \(i\in s\) and \(\hat{M}(i)=0\) if
\(i\notin s\).

The estimators of the population parameters can then be expressed as
functional of the measure \(\hat{M}\).

\begin{itemize}
\item
  Total: \(\hat{Y}=T(\hat{M})=\int yd\hat{M}=\sum_s w_iy_i\)
\item
  Ratio of totals:
  \(\hat{R}=T(\hat{M})=\frac{\int y d\hat{M}}{\int x d\hat{M}}=\frac{\sum_s w_iy_i}{\sum_s w_ix_i}\)
\item
  Cumulative distribution function:
  \(\hat{F}(x)=T(\hat{M})=\frac{\int 1(y\leq x)d\hat{M}}{\int{d\hat{M}}}=\frac{\sum_s w_i 1(y_i\leq x)}{\sum_s w_i}\)
\end{itemize}

\section{The variance estimator}\label{the-variance-estimator}

The variance of the estimator \(T(\hat{M})\) can approximated by:

\[Var\left[T(\hat{M})\right]\cong var\left[\sum_s w_i z_i\right]\]

The \texttt{linearized} variable \(z\) is given by the derivative of the
functional:

\[
z_k=lim_{t\rightarrow0}\frac{T(M+t\delta_k)-T(M)}{t}=IT_k(M)
\]

where, \(\delta_k\) is the Dirac measure in \(k\): \(\delta_k(i)=1\) if
and only if \(i=k\).

This \textbf{derivative} is called \textbf{Influence Function} and was
introduced in the area of \textbf{Robust Statistics}.

\section{Influence functions -
Examples}\label{influence-functions---examples}

\begin{itemize}
\item
  Total: \[
  \begin{aligned}
  IT_k(M)&=lim_{t\rightarrow 0}\frac{T(M+t\delta_k)-T(M)}{t}\\
  &=lim_{t\rightarrow 0}\frac{\int y.d(M+t\delta_k)-\int y.dM}{t}\\
  &=lim_{t\rightarrow 0}\frac{\int yd(t\delta_k)}{t}=y_k  
  \end{aligned}
  \]
\item
  Ratio of two totals: \[
  \begin{aligned}
  IR_k(M)&=I\left(\frac{U}{V}\right)_k(M)=\frac{V(M)\times IU_k(M)-U(M)\times IV_k(M)}{V(M)^2}\\
  &=\frac{X y_k-Y x_k}{X^2}=\frac{1}{X}(y_k-Rx_k)
  \end{aligned}
  \]
\end{itemize}

\section{Linearization by influence function -
Examples}\label{linearization-by-influence-function---examples}

\begin{itemize}
\tightlist
\item
  At-risk-of-poverty threshold: \[
  arpt = 0.6\times m
  \] where \(m\) is the median income.
\end{itemize}

\[
z_k= -\frac{0.6}{f(m)}\times\frac{1}{N}\times\left[I(y_k\leq m-0.5) \right]
\]

\begin{itemize}
\tightlist
\item
  At-risk-of-poverty rate:
\end{itemize}

\[
 arpr=\frac{\sum_U I(y_i \leq t)}{\sum_U w_i}.100
\] \[
z_k=\frac{1}{N}\left[I(y_k\leq t)-t\right]-\frac{0.6}{N}\times\frac{f(t)}{f(m)}\left[I(y_k\leq m)-0.5\right]
\]

where:

\(N\) - population size;

\(t\) - at-risk-of-poverty threshold;

\(y_k\) - income of person \(k\);

\(m\) - median income;

\(f\) - income density function;

\section{Structure of the library}\label{structure-of-the-library}

In the library convey, there are some basic functions that produces the
linearized variables of some estimates that often enter in the
definition of measures of concentration and poverty. For example the
\texttt{quantile} which is linearized by the function
\texttt{svyiqalpha}. Other example is the function \texttt{svyisq} that
linearizes the total below a quantile of the variable.

From the linearized variables of these basic estimates it is possible by
using rules of composition, valid for influence functions, to derive the
influence function of more complex estimates. By definition the
influence function is a Gateaux derivative and the rules rules of
composition valid for Gateaux derivatives also hold for Influence
Functions.

The following property of Gateaux derivatives was often used in the
library convey. Let \(g\) be a differentible function of \(m\)
variables. Suppose we want to compute the influence function of the
estimator \(g(T_1, T_2,\ldots, T_m)\), knowing the Influence function of
the estimators \(T_i, i=1,\ldots, m\). Then the following holds:

\[
I(g(T_1, T_2,\ldots, T_m)) = \sum_{i=1}^m \frac{\partial g}{\partial T_i}I(T_i)
\]

In the library convey this rule is implemented by the function
\texttt{contrastinf} which uses the R function \texttt{deriv} to compute
the formal partial derivatives \(\frac{\partial g}{\partial T_i}\).

For example, suppose we want to linearize the
\texttt{Relative\ median\ poverty\ gap}(rmpg), defined as the difference
between the at-risk-of-poverty threshold (\texttt{arpt}) and the median
of incomes less than the \texttt{arpt} relative to the \texttt{arprt}:

\[
rmpg= \frac{arpt-medpoor} {arpt}
\]

where \texttt{medpoor} is the median of incomes less than \texttt{arpt}.

Suppose we know how to linearize \texttt{arpt} and \texttt{medpoor},
then by applying the function \texttt{contrastinf} with \[
g(T_1,T_2)= \frac{(T_1 - T_2)}{T_1}
\] we linearize the \texttt{rmpg}.

\subsection{Examples of use of the library
convey}\label{examples-of-use-of-the-library-convey}

In the following examples we will use the data set \texttt{eusilc}
contained in the libraries \texttt{vardpoor} and \texttt{Laeken}.

\begin{Shaded}
\begin{Highlighting}[]
\KeywordTok{library}\NormalTok{(vardpoor)}
\KeywordTok{data}\NormalTok{(eusilc)}
\end{Highlighting}
\end{Shaded}

Next, we create an object of class \texttt{survey.design} using the
function \texttt{svydesign} of the library survey:

\begin{Shaded}
\begin{Highlighting}[]
\KeywordTok{library}\NormalTok{(survey)}
\NormalTok{des_eusilc <-}\StringTok{ }\KeywordTok{svydesign}\NormalTok{(}\DataTypeTok{ids =} \NormalTok{~rb030, }\DataTypeTok{strata =}\NormalTok{~db040,  }\DataTypeTok{weights =} \NormalTok{~rb050, }\DataTypeTok{data =} \NormalTok{eusilc)}
\end{Highlighting}
\end{Shaded}

Right after the creation of the design object \texttt{des\_eusilc}, we
should use the function \texttt{convey\_prep} that adds an attribute to
the survey design which saves information on the design object based
upon the whole sample, needed to work with subset designs.

\begin{Shaded}
\begin{Highlighting}[]
\KeywordTok{library}\NormalTok{(convey)}
\NormalTok{des_eusilc <-}\StringTok{ }\KeywordTok{convey_prep}\NormalTok{( des_eusilc )}
\end{Highlighting}
\end{Shaded}

To estimate the \texttt{at-risk-of-poverty\ rate} we use the function
\texttt{svyarpt}:

\begin{Shaded}
\begin{Highlighting}[]
\KeywordTok{svyarpr}\NormalTok{(~eqIncome, }\DataTypeTok{design=}\NormalTok{des_eusilc)}
\end{Highlighting}
\end{Shaded}

\begin{verbatim}
            arpr     SE
eqIncome 0.14444 0.0028
\end{verbatim}

To estimate the \texttt{at-risk-of-poverty\ rate} for domains defined by
the variable \texttt{db040} we use

\begin{Shaded}
\begin{Highlighting}[]
\KeywordTok{svyby}\NormalTok{(~eqIncome, }\DataTypeTok{by =} \NormalTok{~db040, }\DataTypeTok{design =} \NormalTok{des_eusilc, }\DataTypeTok{FUN =} \NormalTok{svyarpr, }\DataTypeTok{deff =} \OtherTok{FALSE}\NormalTok{)}
\end{Highlighting}
\end{Shaded}

\begin{verbatim}
                      db040  eqIncome          se
Burgenland       Burgenland 0.1953984 0.017202243
Carinthia         Carinthia 0.1308627 0.010610622
Lower Austria Lower Austria 0.1384362 0.006517660
Salzburg           Salzburg 0.1378734 0.011579280
Styria               Styria 0.1437464 0.007452360
Tyrol                 Tyrol 0.1530819 0.009880430
Upper Austria Upper Austria 0.1088977 0.005928336
Vienna               Vienna 0.1723468 0.007682826
Vorarlberg       Vorarlberg 0.1653731 0.013754670
\end{verbatim}

Using the same data set, we estimate the
\texttt{quintile\ share\ ratio}:

\begin{Shaded}
\begin{Highlighting}[]
\CommentTok{# for the whole population}
\KeywordTok{svyqsr}\NormalTok{(~eqIncome, }\DataTypeTok{design=}\NormalTok{des_eusilc, }\DataTypeTok{alpha=} \NormalTok{.}\DecValTok{20}\NormalTok{)}
\end{Highlighting}
\end{Shaded}

\begin{verbatim}
          qsr     SE
eqIncome 3.97 0.0426
\end{verbatim}

\begin{Shaded}
\begin{Highlighting}[]
\CommentTok{# for domains}
\KeywordTok{svyby}\NormalTok{(~eqIncome, }\DataTypeTok{by =} \NormalTok{~db040, }\DataTypeTok{design =} \NormalTok{des_eusilc,}
  \DataTypeTok{FUN =} \NormalTok{svyqsr, }\DataTypeTok{alpha=} \NormalTok{.}\DecValTok{20}\NormalTok{, }\DataTypeTok{deff =} \OtherTok{FALSE}\NormalTok{)}
\end{Highlighting}
\end{Shaded}

\begin{verbatim}
                      db040 eqIncome         se
Burgenland       Burgenland 5.008486 0.32755685
Carinthia         Carinthia 3.562404 0.10909726
Lower Austria Lower Austria 3.824539 0.08783599
Salzburg           Salzburg 3.768393 0.17015086
Styria               Styria 3.464305 0.09364800
Tyrol                 Tyrol 3.586046 0.13629739
Upper Austria Upper Austria 3.668289 0.09310624
Vienna               Vienna 4.654743 0.13135731
Vorarlberg       Vorarlberg 4.366511 0.20532075
\end{verbatim}

These functions can be used as S3 methods for the classes
\texttt{survey.design} and \texttt{svyrep.design}.

Let's create a design object of class \texttt{svyrep.design} and run the
function \texttt{convey\_prep} on it:

\begin{Shaded}
\begin{Highlighting}[]
\NormalTok{des_eusilc_rep <-}\StringTok{ }\KeywordTok{as.svrepdesign}\NormalTok{(des_eusilc, }\DataTypeTok{type =} \StringTok{"bootstrap"}\NormalTok{)}
\NormalTok{des_eusilc_rep <-}\StringTok{ }\KeywordTok{convey_prep}\NormalTok{(des_eusilc_rep) }
\end{Highlighting}
\end{Shaded}

and then use the function \texttt{svyarpr}:

\begin{Shaded}
\begin{Highlighting}[]
\KeywordTok{svyarpr}\NormalTok{(~eqIncome, }\DataTypeTok{design=}\NormalTok{des_eusilc_rep)}
\end{Highlighting}
\end{Shaded}

\begin{verbatim}
            arpr     SE
eqIncome 0.14444 0.0032
\end{verbatim}

\begin{Shaded}
\begin{Highlighting}[]
\KeywordTok{svyby}\NormalTok{(~eqIncome, }\DataTypeTok{by =} \NormalTok{~db040, }\DataTypeTok{design =} \NormalTok{des_eusilc_rep, }\DataTypeTok{FUN =} \NormalTok{svyarpr, }\DataTypeTok{deff =} \OtherTok{FALSE}\NormalTok{)}
\end{Highlighting}
\end{Shaded}

\begin{verbatim}
                      db040  eqIncome se.eqIncome
Burgenland       Burgenland 0.1953984 0.016236045
Carinthia         Carinthia 0.1308627 0.011322973
Lower Austria Lower Austria 0.1384362 0.006671846
Salzburg           Salzburg 0.1378734 0.011012774
Styria               Styria 0.1437464 0.008135005
Tyrol                 Tyrol 0.1530819 0.011701101
Upper Austria Upper Austria 0.1088977 0.005926015
Vienna               Vienna 0.1723468 0.007182969
Vorarlberg       Vorarlberg 0.1653731 0.013064658
\end{verbatim}

The functions of the library convey are called in a similar way to the
functions in library survey.

It is also possible to deal with missing values by using the argument
\texttt{na.rm}.

\begin{Shaded}
\begin{Highlighting}[]
\CommentTok{# survey.design using a variable with missings}
\KeywordTok{svygini}\NormalTok{( ~}\StringTok{ }\NormalTok{py010n , }\DataTypeTok{design =} \NormalTok{des_eusilc )}
\end{Highlighting}
\end{Shaded}

\begin{verbatim}
       gini SE
py010n   NA NA
\end{verbatim}

\begin{Shaded}
\begin{Highlighting}[]
\KeywordTok{svygini}\NormalTok{( ~}\StringTok{ }\NormalTok{py010n , }\DataTypeTok{design =} \NormalTok{des_eusilc , }\DataTypeTok{na.rm =} \OtherTok{TRUE} \NormalTok{)}
\end{Highlighting}
\end{Shaded}

\begin{verbatim}
          gini     SE
py010n 0.64606 0.0036
\end{verbatim}

\begin{Shaded}
\begin{Highlighting}[]
\CommentTok{# svyrep.design using a variable with missings}
\CommentTok{# svygini( ~ py010n , design = des_eusilc_rep ) get error}
\KeywordTok{svygini}\NormalTok{( ~}\StringTok{ }\NormalTok{py010n , }\DataTypeTok{design =} \NormalTok{des_eusilc_rep , }\DataTypeTok{na.rm =} \OtherTok{TRUE} \NormalTok{)}
\end{Highlighting}
\end{Shaded}

\begin{verbatim}
          gini     SE
py010n 0.64606 0.0036
\end{verbatim}

\section{FGT indicator}\label{fgt-indicator}

\citep{foster1984} proposed a family of indicators to measure poverty.

The class of \(FGT\) measures, can be defined as

\[
p=\frac{1}{N}\sum_{k\in U}h(y_{k},\theta ), 
\]

where

\[
h(y_{k},\theta )=\left[ \frac{(\theta -y_{k})}{\theta }\right] ^{\gamma
}\delta \left\{ y_{k}\leq \theta \right\} , 
\]

where: \(\theta\) is the poverty threshold; \(\delta\) the indicator
function that assigns value 1 if the condition \(\{y_{k}\leq \theta \}\)
is satisfied and 0 otherwise, and \(\gamma\) is a non-negative constant.

When \(\gamma =0\), \(p\) can be interpreted as the ratio of poor
people, and for \(\gamma \geq 1\), the weight of poor people increases
with the value \(\gamma\), (Foster and all, 1984).

The poverty measure FGT is implemented in the library convey by the
function \texttt{svyfgt}. The argument \texttt{thresh\_type} of this
function defines the type of poverty threshold adopted. There are three
possible choices:

\begin{enumerate}
\def\labelenumi{\arabic{enumi}.}
\tightlist
\item
  \texttt{abs} -- fixed and given by the argument thresh\_value
\item
  \texttt{relq} -- a proportion of a quantile fixed by the argument
  \texttt{proportion} and the quantile is defined by the argument
  \texttt{order}.
\item
  \texttt{relm} -- a proportion of the mean fixed the argument
  \texttt{proportion}
\end{enumerate}

The quantile and the mean involved in the definition of the threshold
are estimated for the whole population. When \(\gamma=0\) and
\(\theta= .6*MED\) the measure is equal to the indicator \texttt{arpr}
computed by the function \texttt{svyarpr}.

Next, we give some examples of the function \texttt{svyfgt} to estimate
the values of the FGT poverty index.

Consider first the poverty threshold fixed (\(\gamma=0\)) in the value
\(10000\). The headcount ratio (FGT0) is

\begin{Shaded}
\begin{Highlighting}[]
\KeywordTok{svyfgt}\NormalTok{(~eqIncome, des_eusilc, }\DataTypeTok{g=}\DecValTok{0}\NormalTok{, }\DataTypeTok{abs_thresh=}\DecValTok{10000}\NormalTok{)}
\end{Highlighting}
\end{Shaded}

\begin{verbatim}
            fgt0     SE
eqIncome 0.11444 0.0027
\end{verbatim}

The poverty gap (FGT1) (\(\gamma=1\)) index for the poverty threshold
fixed at the same value is

\begin{Shaded}
\begin{Highlighting}[]
\KeywordTok{svyfgt}\NormalTok{(~eqIncome, des_eusilc, }\DataTypeTok{g=}\DecValTok{1}\NormalTok{, }\DataTypeTok{abs_thresh=}\DecValTok{10000}\NormalTok{)}
\end{Highlighting}
\end{Shaded}

\begin{verbatim}
             fgt1     SE
eqIncome 0.032085 0.0011
\end{verbatim}

To estimate the FGT0 with the poverty threshold fixed at \(0.6* MED\) we
fix the argument type\_thresh=``relq'' and use the default values for
\texttt{percent} and \texttt{order}:

\begin{Shaded}
\begin{Highlighting}[]
\KeywordTok{svyfgt}\NormalTok{(~eqIncome, des_eusilc, }\DataTypeTok{g=}\DecValTok{0}\NormalTok{, }\DataTypeTok{type_thresh=} \StringTok{"relq"}\NormalTok{)}
\end{Highlighting}
\end{Shaded}

\begin{verbatim}
            fgt0     SE
eqIncome 0.14444 0.0028
\end{verbatim}

that matches the estimate obtained by

\begin{Shaded}
\begin{Highlighting}[]
\KeywordTok{svyarpr}\NormalTok{(~eqIncome, }\DataTypeTok{design=}\NormalTok{des_eusilc, .}\DecValTok{5}\NormalTok{, .}\DecValTok{6}\NormalTok{)}
\end{Highlighting}
\end{Shaded}

\begin{verbatim}
            arpr     SE
eqIncome 0.14444 0.0028
\end{verbatim}

To estimate the poverty gap(FGT1) with the poverty threshold equal to
\(0.6*MEAN\) we use:

\begin{Shaded}
\begin{Highlighting}[]
\KeywordTok{svyfgt}\NormalTok{(~eqIncome, des_eusilc, }\DataTypeTok{g=}\DecValTok{1}\NormalTok{, }\DataTypeTok{type_thresh=} \StringTok{"relm"}\NormalTok{)}
\end{Highlighting}
\end{Shaded}

\begin{verbatim}
             fgt1     SE
eqIncome 0.051187 0.0011
\end{verbatim}

djalma, where do these references go on this page? \citep{berger2003}
and \citep{osier2009} and \citep{deville1999}

\chapter{Poverty Indices}\label{poverty}

{[}I think this is a good start. I don't think that gender pay gap,
quantiles and totals are measures of poverty. Consider another chapter
on other wellbeing measures.{]}

\section{At Risk of Poverty Ratio and Threshold (svyarpr,
svyarpt)}\label{at-risk-of-poverty-ratio-and-threshold-svyarpr-svyarpt}

\section{The Gender Pay Gap (svygpg)}\label{the-gender-pay-gap-svygpg}

\section{Quintile Share Ratio
(svyqsr)}\label{quintile-share-ratio-svyqsr}

\section{Relative Median Income Ratio
(svyrmir)}\label{relative-median-income-ratio-svyrmir}

\section{Relative Median Poverty Gap
(svyrmpg)}\label{relative-median-poverty-gap-svyrmpg}

\section{Median Income Below the At Risk of Poverty Threshold
(svypoormed)}\label{median-income-below-the-at-risk-of-poverty-threshold-svypoormed}

\section{Foster-Greer-Thorbecke class
(svyfgt)}\label{foster-greer-thorbecke-class-svyfgt}

\chapter{Inequality Measurement}\label{inequality}

{[}Present an introduction to what is inequality{]}.

\section{Theoretical aspects of
inequality}\label{theoretical-aspects-of-inequality}

\subsection{Desirable properties of inequality
measures}\label{desirable-properties-of-inequality-measures}

\section{Lorenz Curve (svylorenz)}\label{lorenz-curve-svylorenz}

here are the references

\citep{kovacevic1997} and \citep{lerman1989} and \citep{langel2012}

\section{Measures derived from the Lorenz
Curve}\label{measures-derived-from-the-lorenz-curve}

\subsection{Gini index (svygini)}\label{gini-index-svygini}

here are the references

\citep{osier2009} and \citep{deville1999}

\subsection{Amato index (svyamato)}\label{amato-index-svyamato}

here are the references

\citep{barabesi2016} and \citep{arnold2012}

\subsection{Zenga Index and Curve (svyzenga,
svyzengacurve)}\label{zenga-index-and-curve-svyzenga-svyzengacurve}

guilherme..this has three references? not just two?

here are the references

\citep{barabesi2016} and \citep{langel2012} and \citep{deville1999}

\section{Entropy-based Measures}\label{entropy-based-measures}

\subsection{Atkinson index (svyatk)}\label{atkinson-index-svyatk}

here are the references

\citep{langel2012} and \citep{biewen2003}

\subsection{Generalized Entropy and Decomposition (svygei,
svygeidec)}\label{generalized-entropy-and-decomposition-svygei-svygeidec}

guilherme..this has three references? not just two?

here are the references

\citep{langel2012} and \citep{biewen2003} and \citep{shorrocks1984}

\subsection{J-Divergence Entropy and Decomposition (svyjdiv,
svyjdivdec)}\label{j-divergence-entropy-and-decomposition-svyjdiv-svyjdivdec}

here are the references

\citep{shorrocks1984} and \citep{rohde2016} and \citep{biewen2003}

\subsection{Rényi Divergence
(svyrenyi)}\label{renyi-divergence-svyrenyi}

here are the references

\citep{langel2012}

\chapter{Multidimensional Indices}\label{multidimensional}

We have finished a nice book.

\section{Alkire-Foster Class and Decomposition (svyafc,
svyafcdec)}\label{alkire-foster-class-and-decomposition-svyafc-svyafcdec}

\section{Bourguignon (1999) inequality class
(svybmi)}\label{bourguignon-1999-inequality-class-svybmi}

\bibliography{packages,book}


\end{document}
